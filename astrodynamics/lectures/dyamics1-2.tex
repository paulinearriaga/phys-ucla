% standard packages
\usepackage{graphicx} % support the \includegraphics command and options
\usepackage{amsmath} % for nice math commands and environments

% font packages
\usepackage{amssymb} % for \mathbb, \mathfrak fonts
\usepackage{mathrsfs} % for \mathscr font
\DeclareMathAlphabet{\mathpzc}{OT1}{pzc}{m}{it} % defines \mathpzc for Zapf Chancery (standard postscript) font

% other packages
\usepackage{datetime} % allows easy formatting of dates, e.g. \formatdate{dd}{mm}{yyyy}
\usepackage{caption} % makes figure captions better, more configurable
\usepackage{enumitem} % allows for custom labels on enumerated lists, e.g. \begin{enumerate}[label=\textbf{(\alph*)}]
\usepackage[squaren]{SIunits} % for nice units formatting e.g. \unit{50}{\kilo\gram}
\usepackage{cancel} % for crossing out terms with \cancel
\usepackage{verbatim} % for verbatim and comment environments
\usepackage{tensor} % for \indices e.g. M\indices{^a_b^{cd}_e}, and \tensor e.g. \tensor[^a_b^c_d]{M}{^a_b^c_d}
\usepackage{feynmp-auto} % for Feynman diagrams. 
\usepackage{pgfplots} % for plotting in tikzpicture environment

% new commands
\newcommand{\beg}{\begin} % a few letters less for beginning environments
\newenvironment{eqn}{\begin{equation}}{\end{equation}} % a lot fewer letter for equation environment

% notational commands
\newcommand{\opname}[1]{\operatorname{#1}} % custom operator names
\newcommand{\fslash}[1]{#1\!\!\!/} % feynman slash
\newcommand{\pd}{\partial} % partial differential shortcut
\newcommand{\ket}[1]{\left| #1 \right>} % for Dirac kets
\newcommand{\bra}[1]{\left< #1 \right|} % for Dirac bras
\newcommand{\braket}[2]{\left< #1 \vphantom{#2} \right| 
	\left. #2 \vphantom{#1} \right>} % for Dirac brackets
%\let\underdot=\d % rename builtin command \d{} to \underdot{}
%\renewcommand{\d}[2]{\frac{d #1}{d #2}} % for derivatives
%\newcommand{\pd}[2]{\frac{\partial #1}{\partial #2}} % for partial derivatives
%\newcommand{\fd}[2]{\frac{\delta #1}{\delta #2}} % for functional derivatives
\let\vaccent=\v % rename builtin command \v{} to \vaccent{}
%\renewcommand{\v}[1]{\ensuremath{\mathbf{#1}}} % for vectors
\renewcommand{\v}[1]{\ensuremath{\boldsymbol{\mathbf{#1}}}} % for vectors
%\newcommand{\gv}[1]{\ensurmath{\mbox{\boldmath$ #1 $}}} % for vectors of Greek letters
\newcommand{\uv}[1]{\ensuremath{\boldsymbol{\mathbf{\widehat{#1}}}}} % for unit vectors
\newcommand{\abs}[1]{\left| #1 \right|} % for absolute value ||x||
%\newcommand{\mag}{\abs} % magnitude, just another name for \abs
\newcommand{\norm}[1]{\left\Vert #1 \right\Vert} % for norm ||v||
\newcommand{\avg}[1]{\left< #1 \right>} % for average <x>
\newcommand{\inner}[2]{\left< #1, #2 \right>} % for inner product <x,y>
\newcommand{\set}[1]{ \left\{ #1 \right\} } % for sets {a,b,c,...}
\newcommand{\tr}{\opname{tr}} % for trace
\newcommand{\Tr}{\opname{Tr}} % for Trace

% notational shortcuts
\newcommand{\reals}{\mathbb{R}} % real numbers
\newcommand{\complexes}{\mathbb{C}} % complex numbers
\newcommand{\nats}{\mathbb{N}} % natural numbers
\newcommand{\irrats}{\mathbb{Q}} % irrationals
\newcommand{\quats}{\mathbb{H}} % quaternions (a la Hamilton)
\newcommand{\euclids}{\mathbb{E}} % Euclidean space
\newcommand{\bigo}{\mathcal{O}} % big O notation
\newcommand{\Lag}{\mathcal{L}} % fancy Lagrangian
\newcommand{\Ham}{\mathcal{H}} % fancy Hamiltonian





%%%%%%%%%%%%%%%%%%%
% some templates for various things
\begin{comment}

% template for figures
\begin{figure}
\centering
\includegraphics{myfile.png}
\caption{This is a caption}
\label{fig:myfigure}
\end{figure}

% template for Feynman diagrams using feynmf/feynmp
\begin{fmfgraph*}(40,25)
\fmfleft{em,ep}
\fmf{fermion}{em,Zee,ep}
\fmf{photon,label=$Z$}{Zee,Zff}
\fmf{fermion}{fb,Zff,f}
\fmfright{fb,f}
\fmfdot{Zee,Zff}
\end{fmfgraph*}

% template for drawing plots with pgfplot
\pgfplotsset{compat=1.3,compat/path replacement=1.5.1}
\begin{tikzpicture}
\begin{axis}[
extra x ticks={-2,2},
extra y ticks={-2,2},
extra tick style={grid=major}]
\addplot {x};
\draw (axis cs:0,0) circle[radius=2];
\end{axis}
\end{tikzpicture}

\end{comment}
%%%%%%%%%%%%%%%%%%%




\begin{document}

\textit{Read Chapters 1 and Sections 2.1-2.3}

\section{Galaxy Dynamics}
Since there are so many stars, can treat the system either as a fluid
or as a system of particles. We may want to consider it as a particle
if we want to account for star-interactions or as a fluid to consider
its bulk motion. We will have a distribution function which is a
function of position, velocity and time. We typically normalize this
such that it is normalized to unity when integrated over phase space
\begin{equation}
\int f(\v{x},\v{v},\v{t}) d^3\v{x} d^3\v{v} = 1
\end{equation}
The mean free path should be small in comparison to the macroscopic
length scales if we're assuming that it's a fluid in the
\textbf{continuum approximation}. Quantities such as the density and
velocity are continuous and they are independent parameters. 


We have three regimes that we can deal with if we abandon the fluid
approximatiosn
\begin{itemize}
\item Lumpy
\item Smooth - fluid approximation
\item In-between - globular clusters
\end{itemize}
Any system that we want to describe, we can do it any of these three
ways but we have to pick which one. 

\section{Defining Basic Parameters}

We have two particles at two positions $\v{x_1}$ and $\v{x_2}$ We'll
define the difference between them $\v{x_{12}}$. The gravitational
acceleration between them is given by $\v{\ddot{r}} = \frac{G
  m_2}{|\v{x_{12}}|^3} \v{x_{12}}$

\begin{equation}
\v{\ddot{x_1}} = \sum_{j\ne 1} \frac{G
  m_j}{|\v{x_{ji}}|^3} \v{x_{ji}}
\end{equation}
WE can express this in terms of a potential
\begin{equation}
\v{\ddot{x_i}} = \pd{\Phi}{x_i} = -\grad\Phi
\end{equation}
Now let's define the gravitational potential
\begin{equation}
W = -\frac{1}{2} \sum_i \sum_{j\ne i} \frac{Gm_1 m_j}{|x_{ij}|} =
\frac{1}{2} \sum_i m_i \Phi (x_i)
\end{equation}
We can rewrite this in terms of a typical radius $r_h$
\begin{equation}
W = -\alpha \frac{GM^2}{r_h}
\end{equation}
Kinetic energy is just
\begin{equation}
T = \frac{1}{2} \sum_j m_j |\dot{x_j}|^2
\end{equation}
In an isolated system
\begin{equation}
E = T + W = constant
\end{equation}
\subsection{Virial Theorem} 
Let's take the moment of inertia of the system which is defined as
\begin{equation}
I = \sum_i m_i |x_i^2|
\end{equation}
Right now we are concerned with the second derivative of the moment of
inertia and subsituting in our previous expresion for $\v{\ddot{x_i}}$
\begin{equation}
\ddot{I} = 2\sum_i (m_i |\vec{v_i}|^2 + 2m_i \v{\ddot{x_i}} - \v{x_i} )
= 4T + 2 \sum_i m_i \v{x_i} ( \sum_{j\ne1} \frac{G m_j(\v{x_j} -
  \v{x_j})}{|\v{x_j} - \v{x_j}|})
\end{equation}
\begin{equation}
\sum_i \sum_{j\ne i} G m_i m_j \frac{\v{x_j} \cdot (\v{x_j} -
  \v{x_i})}{|\v{x_j} - \v{x_i}|^3} = \sum_j \sum_{i\ne j} G m_i m_j
\frac{x_j \cdot (\vec{x_i} - x_j)} {|\v{x_i} - \v{x_j}|^3}
\end{equation}
These are equivalent, meaning that they are equal to half of the sum,
which is
\begin{equation}
\frac{1}{2} G m_i m_j \sum_i \sum{j\ne i} (\v{x_i} - \v{x_j})(\v{x_j} - \v{x_i})
/ (|\v{x_i} - \v{x_j}|^3)
\end{equation}
Taking out the minsu sign we get our expression for W
\begin{equation}
W = -\frac{1}{2} \sum_i \sum_{j\ne i} \frac{Gm_1 m_j}{|x_{ij}|} 
\end{equation}
So oerall
\begin{equation}
\ddot{I} = 4T + 2W
\end{equation}
\begin{equation}
2T + W = 0
\end{equation}
Note that this only occurs if the system is in a
bound isolated system in equilibrium. The overal energy of the system
is T + W, we can say for any gravitational system
\begin{equation}
E = W / 2
\end{equation}
This is most used to estimate the masses of things. 

\section{Continuum}
We expressed the previous sections in terms of sums, but now we want
to take out the particle-partical interactions and put it in terms of
$\rho(\v{x})$ 
Now our potential is defined as
\begin{equation}
\Phi(\v{x}) = -G \int \frac{\rho(\v{x'})}{|\v{x} - \v{x'}|} d^3x'
\end{equation}
We can use Gauss's therem to state
\begin{equation}
\int \grad \Phi \v{d^2s} = 4\pi G M_{enclose}
\end{equation}
We can then use the divergence theorem
\begin{equation}
\int \nabla^2 \Phi \v{d^3x} = \int \grad \Phi \cdot d^2\v{s}
\end{equation}
What we get then is that the integral of the divergence over the
volume:
\begin{equation}
\int \nabla^2 \Phi d^3\v{x} = 4\pi G \int \rho(\v{x'}) d^3\v{x'}
\end{equation}
We then end up with Poissons equation because we can drop the integral
\begin{equation}
\nabla^2 \Phi = 4\pi G \rho
\end{equation}
This is an important equation that we'll come back to. Also in the
continuum approximation, we have an expression for the gravitational
potential W which is analogous to our previous definition in terms of
the summation
\begin{equation} 
W = \frac{1}{2} \int \rho(\v{x'}) \Phi(\v{x'}) d^3\v{x}
\end{equation}
With these results, we can start applying it to simple situations. The
simplest one is a spherical potential. 


\section{Spherical potential}
We can use the spherical symmetry to make two claims. A body inside a
spherical shell of matter does not experience any force from outside
of the shell. OUtisde of the shell, the force is the same as a point
mass all at the center. This is due to Gauss' theorem. The
contribution at a distance of r of an infinitesimal shell of thickess $dr$ at distance r'
\begin{equation}
\delta \Phi = - \frac{G M_{shell}}{r} = - \frac{4\pi G r'^2
  \rho(r')}{r}
\end{equation}
if r is less than r' then there is no force, but there is still a
potential, but it is a constant. We can choose what that constant is
so that we have a meaningful potential. We choose it such that the
contribution of the potential is continuous at the boundary. So to do
that, we can choose that the contribution is 
\begin{equation}
\frac{-G M_{shell}}{r'} = - \frac{4\pi G r'^2 \rho(r') dr'}{r'}
\end{equation}
That is independent of our location r.

In total, we now combine those two where the first term is for the
interior protion
\begin{equation}
\phi(r) = -G\left[ \frac{1}{r} \int^r_0 r\pi \rho(r') dr' +
    \int_r^{\infty} 4\pi \rho(r') r' dr' \right]
\end{equation}
The interior's potential is not zero, but the derivative is zero

Dark matter tends to aggregate in a spherical halo, so we will use
this formalism for a first order approximation for the potential of
what a galaxy is doing.

\section{Velocity}
One of the key observables in a galaxy or in any dynamical system are
velocities which helps us see what is happening but also how it
evolves. We use the velocity of the galaxy to measure the mass of the
galaxy. The circular velocity is 
\begin{equation}
\frac{v_c^2}{r} = \frac{GM(r)}{r^2}
\end{equation}
\begin{equation}
v_c^2 = \frac{GM(r)}{r} = -rF(r) = r(\grad{\Phi})
\end{equation}
The other velocity that is an important determinant of the evolution
of a system is the escape velocity which is defined in terms of how
much energy it takes for a particle to reach infinity (i.e. where E =
0) from where it
starts off. That makes it easy that we can write it from conservation
of energy
\begin{equation}
\frac{mv_e^2}{2} + m\Phi(r) = 0
\end{equation}
\begin{equation}
v_e^2 = -2\Phi(r)
\end{equation}
\subsubsection{Homogeneous Sphere}
Let's look at some examples of spherical systems and see what their
potentials are and how the systems might effect the behavior or
particles. The simplest is a homogeneous sphere. 
\begin{equation}
\phi = const
\end{equation}
\begin{equation}
M_{enc} = \int_0^r 4\pi r'^2 \rho dr' = \frac{4}{3} \pi r^3 \rho
\end{equation}
The total force is 
\begin{equation}
F(r) = - \frac{GM(r)}{r^2} = - \frac{4\pi G\rho}{3} r
\end{equation}
This is the differential equation for a harmonic oscillator. The core
of a cluster can be approximated as such. Meaning
\begin{equation}
\ddot{r} = -\omega^2r
\end{equation}
We get an equation for
\begin{equation}
\omega^2 = \frac{4\pi G}{3} \rho
\end{equation}
The period of this oscillation is
\begin{equation}
P = \frac{2\pi}{\omega}  = \sqrt{\frac{3\pi}{G\rho}}
\end{equation}
Each star has an oscillation. THe question now arises, if we have a
Keplerian orbit, the period of the rotation is equal to the period of
the oscillation. Are these the same? No, see homework

\section{Dynamical Time}
At this point we have this timescale of a period. Let's define an
important timescale that we encounter, which is the \textbf{dynamical time} of
a system. It is defined as 
\begin{equation}
t_{dyn} = (\frac{3\pi}{16 G\rho}) = P/4
\end{equation}
This is not dependent on our system
We can say in general
\begin{equation}
t \sim (G\rho)^{1/2} 
\end{equation}
THis corresponds to the time between the peak and the middle of an
oscillation, or the average to the furthest it'll go. This dynamical
time is the time that any given system or a particle within a stellar
system will do something dramatic like cross the system. It is related
intimately with the free-fall time. 

\subsubsection{Back to Homogeneous Sphere}
If the radius of the homogeneous sphere b is interior to our radius
(i.e. we're outside of the sphere) then it acts as a point mass
\begin{equation}
\Phi_{r <b} = -\frac{GM_{tot}}{r}
\end{equation}
If radius is less than B. The potential needs to be written in two terms
of to describe the potential, the first of which is r dependent and
the other which is r' dependent
\begin{equation}
- G \left[ \frac{1}{r} \frac{4\pi\rho r^3}{3} + \int^b_3 4\pi\rho r'dr'\right]
\end{equation}
Which can be written as 
\begin{equation}
-\frac{GM_{tot}}{b} \left[ \frac{r^2}{b^2} + \frac{3}{2} ( 1 -
  \frac{r^2}{b^2})\right] = - \frac{GM_{tot}}{b} \left[ \frac{3}{2} -
  \frac{r^2}{2b^2}\right]
\end{equation}
Now is this continuous at r=b. Yes, it's a quick check. Our choice of
constants was appropriate to ensure that the potential is
continuous. 

\subsubsection{Singular Isothermal Sphere}
Means that the energy distribution and so
\begin{equation}
\rho = \rho_0 \left(\frac{r_0}{r}\right)^2
\end{equation}
Characteristics of the isothermal sphere will not b e derived
\begin{equation}
M(r) = \int_0^r 4\pi r^2 dr \rho(r) = M_0 \frac{r}{r_0}
\end{equation}
In this case
\begin{equation}
M_0 = 4\pi \rho_0 r_0^3
\end{equation}
\begin{equation}
v_c^2 = \frac{GM(r)}{r} = const = \frac{GM_0}{r_0}
\end{equation}
This implies that the circulal velocity is a cosntant with radius. so it has a flat rotation curvce. This tells us that the isothermal sphere is a good approximation for a galaxy.

\begin{equation}
\Phi(r) = -G[\frac{1}{r} \int_0^r 4\pi \rho(r') r'^2 dr + \int^\infty_r 4\pi \rho(r') r'^2 dr']
\end{equation}
THis is not an appropriate way to treat the isothermal sphere as the integral will not covnerge. It is more useful to look at differences in potential
\begin{equation}
\Phi(r) - \Phi(r_0) = -G\int_0^r \frac{M(r')}{r'^2} dr = \frac{GM_0}{r_0} \int_{r_0}^r \frac{dr'}{r'} = v_c^2 \ln(r/r_0)
\end{equation}

\subsection{Plummer Model}
We're going to now start wiht a potential and try to work back to a density distribution. We know the potential for a point mass 
\begin{equation}
\Phi(r) = -\frac{GM}{r}
\end{equation}
Now let's generalize that
\begin{equation}
\Phi(r) = -\frac{GM}{\sqrt{r^2 + b^2}}
\end{equation}
This makes it so that the potential converges at r = 0. B is some scale radius which tells us where the model goes from being roughly constant to looking like the point mass. We get the density distribution by invoking Poisson's equation
\begin{equation}
\nabla^2\Phi = 4\phi G\rho
\end{equation}
In spherical coordinates
\begin{equation}
\frac{1}{r^2} \d{}{r} ( r^2 \d{\Phi}{r})
\end{equation}
The result is 
\begin{equation}
4\pi G\rho = \frac{3GMb^2}{(r^2 + b^2)^{3/2}}
\end{equation}
\begin{equation}
\rho(r) = \frac{3M/4\pi b^3}{(1+r^2/b^2)^{5/2}}
\end{equation}
As r goes to zero we get a constant density distrubtion
\begin{equation}
\frac{3M}{4\pi b^3}
\end{equation}
and r goes to infinity
\begin{equation}
\rho \sim r^{-5}
\end{equation}
This is nice because the density falls off 
However, orbits cannot be described analytically in this model. A different model for this is the isochrone potential. 
\begin{equation}
\Phi(r) = -\frac{GM}{b + \sqrt{b^2+r^2}}
\end{equation}


\section{Two-power density model}
\begin{equation}
\rho(r) = \frac{\rho_0}{(r/1)^{\alpha} ( 1 + r/a)^{\beta-\alpha}}
\end{equation}

Look up Dehnen models, Herquist model, Jaffe model, NFW


\subsection{Defionitions}
\subsubsection{Integrals of motion}
Function 
\begin{equation}
I(\v{x}, \v{v}) = I(\v{x_0}, \v{v_0}) 
\end{equation}
Inviariant along the orbit of a particle. For energy conservation, it means that energy is an integral of motion. This is different from a constant of motion
\subsubsection{Constant of motion}
\begin{equation}
C(\v{x},\v{v}, t) = C(\v{x_o}, \v{v_0},t)
\end{equation}
An example is that if we know the equations of motion we can determine at some specific time $t_0$ what the position must or will be in the future. So the three space cooordinates and three velocity coordinates are related to constants of motion that are determined by the constraints determined of the constants of motion. We are more interested in the integrals of motion

\section{Various Integrals of Motion}
For these purposes let us assume that the potential is static or invariant
\subsection{Energy}
Let's consider the dot product between
\begin{equation}
\dot{r} \cdot \ddot{r} = -\dot{r} \cdot \grad \Phi
\end{equation}
\begin{equation}
\d{}{t} (\frac{1}{2} |\dot{r}|^2) = \d{\Phi(r)}{t}
\end{equation}
Oops.


\subsection{Energy}
Assume spherically symmetric
\begin{equation}
\Phi = \Phi(r)
\end{equation}
\begin{equation}
\ddot{r} = F(r) \hat{r}
\end{equation}
\begin{equation}
\v{r}\times \v{F} = 0 = \v{r} \times \v{\ddot{r}}
\end{equation}
\begin{equation}
\d{}{t} (\v{r} \v{\dot{r}}) = 0
\end{equation}
So we get three conservations
\begin{equation}
\v{L} = \v{r}\times \v{v}
\end{equation}

Lessing this to axisymmetric, we know that the symmetry that we can work with is the azimuthal angle $\phi$.
\begin{equation}
\pd{\Phi}{\phi} = 0
\end{equation}
Means that phi is conserved

\section{Fuckballsacks}

\begin{equation}
\v{r} = r\hat{r}
\end{equation}
Unit vector shit
See classical notes. 



\end{document}
