% declare document class and geometry
%\documentclass[12pt]{article} % use larger type; default would be 10pt
%\usepackage[margin=1in]{geometry} % handle page geometry

% ***********************************************************
% ******************* PHYSICS HEADER ************************
% ***********************************************************
% Version 2
\documentclass[12pt]{article} 





\usepackage{datetime} % allows easy formatting of dates, e.g. \formatdate{dd}{mm}{yyyy}

\usepackage{amsmath} % AMS Math Package
\usepackage{amsthm} % Theorem Formatting
\usepackage{amssymb}	% Math symbols such as \mathbb
\usepackage{graphicx} % Allows for eps images
\usepackage{multicol} % Allows for multiple columns
\usepackage[dvips,letterpaper,margin=1in,bottom=1in]{geometry}
 % Sets margins and page size
\pagestyle{empty} % Removes page numbers
\makeatletter % Need for anything that contains an @ command 
%\renewcommand{\maketitle} % Redefine maketitle to conserve space
%{ \begingroup \vskip 10pt \begin{center} \large {\bf \@title}
%	\vskip 10pt \large \@author \hskip 20pt \@date \end{center}
%  \vskip 10pt \endgroup \setcounter{footnote}{0} }
\makeatother % End of region containing @ commands
\renewcommand{\labelenumi}{(\alph{enumi})} % Use letters for enumerate
% \DeclareMathOperator{\Sample}{Sample}
\let\vaccent=\v % rename builtin command \v{} to \vaccent{}
\renewcommand{\v}[1]{\ensuremath{\mathbf{#1}}} % for vectors
\newcommand{\gv}[1]{\ensuremath{\mbox{\boldmath$ #1 $}}} 
% for vectors of Greek letters
\newcommand{\vx}{\ensuremath{\v{x}}} 
% for vectors of Greek letters
\newcommand{\vy}{\ensuremath{\v{y}}} 
% for vectors of Greek letters
\newcommand{\xdot}{\ensuremath{\dot{x}}} 
% for vectors of Greek letters

\newcommand{\ydot}{\ensuremath{\dot{y}}} 
% for vectors of Greek letters
\usepackage{commath} % for some nice standardized syntax stuff. 
	% \dif, \Dif, \od, \pd, \md, \(abs | envert), \(norm | enVert), \(set | cbr), \sbr, \eval, \int(o | c)(o | c), etc
\newcommand{\bbar}[1]{\bar{\bar{#1}}} % for barring things twice -- use \cbar or \zbar instead of original \bbar

\newcommand{\uv}[1]{\ensuremath{\mathbf{\hat{#1}}}} % for unit vector
%\newcommand{\abs}[1]{\left| #1 \right|} % for absolute value
\newcommand{\avg}[1]{\left< #1 \right>} % for average
\let\underdot=\d % rename builtin command \d{} to \underdot{}
\renewcommand{\d}[2]{\frac{d #1}{d #2}} % for derivatives
\newcommand{\dd}[2]{\frac{d^2 #1}{d #2^2}} % for double derivatives
%\newcommand{\pd}[2]{\frac{\partial #1}{\partial #2}} 
% for partial derivatives
\newcommand{\fd}[2]{\frac{\delta #1}{\delta #2}} 
% for functional derivatives

\newcommand{\pdd}[2]{\frac{\partial^2 #1}{\partial #2^2}} 
% for double partial derivatives
\newcommand{\pdc}[3]{\left( \frac{\partial #1}{\partial #2}
 \right)_{#3}} % for thermodynamic partial derivatives
\newcommand{\ket}[1]{\left| #1 \right>} % for Dirac bras
\newcommand{\bra}[1]{\left< #1 \right|} % for Dirac kets
\newcommand{\braket}[2]{\left< #1 \vphantom{#2} \right|
 \left. #2 \vphantom{#1} \right>} % for Dirac brackets
\newcommand{\matrixel}[3]{\left< #1 \vphantom{#2#3} \right|
 #2 \left| #3 \vphantom{#1#2} \right>} % for Dirac matrix elements
\newcommand{\grad}[1]{\gv{\nabla} #1} % for gradient
\let\divsymb=\div % rename builtin command \div to \divsymb
\renewcommand{\div}[1]{\gv{\nabla} \cdot #1} % for divergence
\newcommand{\curl}[1]{\gv{\nabla} \times #1} % for curl
\let\baraccent=\= % rename builtin command \= to \baraccent
\renewcommand{\=}[1]{\stackrel{#1}{=}} % for putting numbers above =
\newtheorem{prop}{Proposition}
\newtheorem{thm}{Theorem}[section]
\newtheorem{lem}[thm]{Lemma}
\theoremstyle{definition}
\newtheorem{dfn}{Definition}
\theoremstyle{remark}
\newtheorem*{rmk}{Remark}
\newcommand{\bigO}{\mathcal{O}} % big O notation
\let \bigo = \bigO % deprecated version. keeping for now because need to update instances in older files










\makeatletter
% À droite
\renewcommand\subsection{\@startsection {subsection}{1}{\z@}%
                                   {-3.5ex \@plus -1ex \@minus -.2ex}%
                                   {2.3ex \@plus.2ex}%
                                   {\raggedright\normalfont\Large\bfseries}}
\makeatother


\makeatletter
\def\section{\@ifstar\unnumberedsection\numberedsection}
\def\numberedsection{\@ifnextchar[%]
  \numberedsectionwithtwoarguments\numberedsectionwithoneargument}
\def\unnumberedsection{\@ifnextchar[%]
  \unnumberedsectionwithtwoarguments\unnumberedsectionwithoneargument}
\def\numberedsectionwithoneargument#1{\numberedsectionwithtwoarguments[#1]{#1}}
\def\unnumberedsectionwithoneargument#1{\unnumberedsectionwithtwoarguments[#1]{#1}}
\def\numberedsectionwithtwoarguments[#1]#2{%
  \ifhmode\par\fi
  \removelastskip
  \vskip 5ex\goodbreak
  \refstepcounter{section}%
  \hbox to \hsize{\vbox{%
      \noindent
      \leavevmode
      \begingroup
      \Large\bfseries\raggedleft
      \thesection.\ 
      #2\par
      \endgroup
      \vskip -2ex
      \noindent\hrulefill
      \vskip -2.2ex\nobreak
      \noindent\hrulefill
      }}\nobreak
  \vskip 2ex\nobreak
  \addcontentsline{toc}{section}{%
    \protect\numberline{\thesection}%
    #1}%
  }
\def\unnumberedsectionwithtwoarguments[#1]#2{%
  \ifhmode\par\fi
  \removelastskip
  \vskip 5ex\goodbreak
%  \refstepcounter{section}%
  \hbox to \hsize{\vbox{%
      \noindent
      \leavevmode
      \begingroup
      \Large\bfseries\raggedleft
%      \thesection.\ 
      #2\par
      \endgroup
      \vskip -2ex
      \noindent\hrulefill
      \vskip -2.2ex\nobreak
      \noindent\hrulefill
      }}\nobreak
  \vskip 2ex\nobreak
  \addcontentsline{toc}{section}{%
%    \protect\numberline{\thesection}%
    #1}%
  }
\makeatother
\pagestyle{empty}




% ***********************************************************
% ********************** END HEADER *************************
% ***********************************************************


\title{Astro 270 -- Astrophysical Dynamics -- The Orbits of Stars}
\author{UCLA, Fall 2014}
%\date{\formatdate{02}{10}{2014}} % Activate to display a given date or no date (if empty),
         % otherwise the current date is printed 
%\date{\formatdate{07}{10}{2014}} 

\begin{document}
\setlength{\unitlength}{1mm}
\maketitle

\section{Orbits in static spherical potentials}
Our fundamental equation in this chapters are the equations of motion which are derived from the Lagrangian
\begin{equation}
L = r^2 \dot{\phi}
\end{equation}
Our equation of motion is:
\begin{equation}
u^2 + \frac{2[\Phi (1/u) - E]}{L^2} = 0
\end{equation}
where u is 1/r

The time that it takes a star to go from apocenter to pericenter back
to apocenter is defined as the \ref{radial period}. Since this is a
spherical symmetric potential, pericenter and apocenter are always the
same for any orbit, though it may precess. The azimuthal period is the
amount of time a particle will take to complete one orbit (traverse
2$\pi$ of $\phi$ if it travelled at the average speed. This will only
be a rational number if the orbit is closed, i.e. the beginning of an
orbit ends at the same position. There are two cases where all bound
orbits are closed. Note that \textit{all} potentials have at least one
orbit that is bound - that of a circular orbit. 
\subsection{Examples of Spherical Potentials}
\subsubsection{Spherical Harmonic Oscillator}
In this case, the radial period is $\pi /\Omega$. In order, the
particle only needs to traverse half of one orbit to reach back to the
same distance. This is because all of the orbits are ellipses centered
on the center of the potential. This is in contrast to the Kepler
potential which has radial period of $2\pi / \Omega$, where the
ellipse has its \textit{focus} at the center of the potential

\subsubsection{Kepler Potential}
\subsubsection{Isochrone Potential}
\subsubsection{Hyperbolic Encounters}



\subsection{Constants and integrals of motion}
A \textbf{constant of motion} is a quantity that 
\begin{equation}
C(\v{x}(t_1) \v{v}(t_1); t_1) = C(\v x(t_2) \v{v}(t_2); t_2)
\end{equation}
This is different from an integral of motion because you need an
initial condition in order to calculate one of these quantities,
i.e. you need to know the value at some particular time in order to be
able to get it at a different time. Generally any initial conditions
or boundary conditions can be considered constants of
motion. Therefore, any orbit, which has six coordinates in phase space
has six constants of motion. From the current time and the six coordinates and
phase space it is possible to determine the position in phase space. 

Integrals of motion are determined solely by the potential, and are
invariant in time. They limit the amount of phase space that a
particle can inhabit. Thus the number of integrals of motion describe
how many coordinates are required to specify the location in phase
space. In other words, having n integrals of motion means that any
particle's orbit will lie on a 6-n dimensional subspace of x. A
non-isolating integral is one that does not limit the number of
dimensions that the orbit can lie on. They're fucking useless. 

\section{Orbits in Axisymmetric Potentials}
Effective potential
\begin{equation}
\Phi_{eff} = \Phi(R, z) + \frac{L_z^2}{2R^2}
\end{equation}


\end{document}
