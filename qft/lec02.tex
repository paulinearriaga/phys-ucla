% declare document class and geometry
\documentclass[12pt]{article} % use larger type; default would be 10pt
\usepackage[margin=1in]{geometry} % handle page geometry

% standard packages
\usepackage{graphicx} % support the \includegraphics command and options
\usepackage{amsmath} % for nice math commands and environments

% font packages
\usepackage{amssymb} % for \mathbb, \mathfrak fonts
\usepackage{mathrsfs} % for \mathscr font
\DeclareMathAlphabet{\mathpzc}{OT1}{pzc}{m}{it} % defines \mathpzc for Zapf Chancery (standard postscript) font

% other packages
\usepackage{datetime} % allows easy formatting of dates, e.g. \formatdate{dd}{mm}{yyyy}
\usepackage{caption} % makes figure captions better, more configurable
\usepackage{enumitem} % allows for custom labels on enumerated lists, e.g. \begin{enumerate}[label=\textbf{(\alph*)}]
\usepackage[squaren]{SIunits} % for nice units formatting e.g. \unit{50}{\kilo\gram}
\usepackage{cancel} % for crossing out terms with \cancel
\usepackage{verbatim} % for verbatim and comment environments
\usepackage{tensor} % for \indices e.g. M\indices{^a_b^{cd}_e}, and \tensor e.g. \tensor[^a_b^c_d]{M}{^a_b^c_d}
\usepackage{feynmp-auto} % for Feynman diagrams. 
\usepackage{pgfplots} % for plotting in tikzpicture environment

% new commands
\newcommand{\beg}{\begin} % a few letters less for beginning environments
\newenvironment{eqn}{\begin{equation}}{\end{equation}} % a lot fewer letter for equation environment

% notational commands
\newcommand{\opname}[1]{\operatorname{#1}} % custom operator names
\newcommand{\fslash}[1]{#1\!\!\!/} % feynman slash
\newcommand{\pd}{\partial} % partial differential shortcut
\newcommand{\ket}[1]{\left| #1 \right>} % for Dirac kets
\newcommand{\bra}[1]{\left< #1 \right|} % for Dirac bras
\newcommand{\braket}[2]{\left< #1 \vphantom{#2} \right| 
	\left. #2 \vphantom{#1} \right>} % for Dirac brackets
%\let\underdot=\d % rename builtin command \d{} to \underdot{}
%\renewcommand{\d}[2]{\frac{d #1}{d #2}} % for derivatives
%\newcommand{\pd}[2]{\frac{\partial #1}{\partial #2}} % for partial derivatives
%\newcommand{\fd}[2]{\frac{\delta #1}{\delta #2}} % for functional derivatives
\let\vaccent=\v % rename builtin command \v{} to \vaccent{}
%\renewcommand{\v}[1]{\ensuremath{\mathbf{#1}}} % for vectors
\renewcommand{\v}[1]{\ensuremath{\boldsymbol{\mathbf{#1}}}} % for vectors
%\newcommand{\gv}[1]{\ensurmath{\mbox{\boldmath$ #1 $}}} % for vectors of Greek letters
\newcommand{\uv}[1]{\ensuremath{\boldsymbol{\mathbf{\widehat{#1}}}}} % for unit vectors
\newcommand{\abs}[1]{\left| #1 \right|} % for absolute value ||x||
%\newcommand{\mag}{\abs} % magnitude, just another name for \abs
\newcommand{\norm}[1]{\left\Vert #1 \right\Vert} % for norm ||v||
\newcommand{\avg}[1]{\left< #1 \right>} % for average <x>
\newcommand{\inner}[2]{\left< #1, #2 \right>} % for inner product <x,y>
\newcommand{\set}[1]{ \left\{ #1 \right\} } % for sets {a,b,c,...}
\newcommand{\tr}{\opname{tr}} % for trace
\newcommand{\Tr}{\opname{Tr}} % for Trace

% notational shortcuts
\newcommand{\reals}{\mathbb{R}} % real numbers
\newcommand{\complexes}{\mathbb{C}} % complex numbers
\newcommand{\nats}{\mathbb{N}} % natural numbers
\newcommand{\irrats}{\mathbb{Q}} % irrationals
\newcommand{\quats}{\mathbb{H}} % quaternions (a la Hamilton)
\newcommand{\euclids}{\mathbb{E}} % Euclidean space
\newcommand{\bigo}{\mathcal{O}} % big O notation
\newcommand{\Lag}{\mathcal{L}} % fancy Lagrangian
\newcommand{\Ham}{\mathcal{H}} % fancy Hamiltonian





%%%%%%%%%%%%%%%%%%%
% some templates for various things
\begin{comment}

% template for figures
\begin{figure}
\centering
\includegraphics{myfile.png}
\caption{This is a caption}
\label{fig:myfigure}
\end{figure}

% template for Feynman diagrams using feynmf/feynmp
\begin{fmfgraph*}(40,25)
\fmfleft{em,ep}
\fmf{fermion}{em,Zee,ep}
\fmf{photon,label=$Z$}{Zee,Zff}
\fmf{fermion}{fb,Zff,f}
\fmfright{fb,f}
\fmfdot{Zee,Zff}
\end{fmfgraph*}

% template for drawing plots with pgfplot
\pgfplotsset{compat=1.3,compat/path replacement=1.5.1}
\begin{tikzpicture}
\begin{axis}[
extra x ticks={-2,2},
extra y ticks={-2,2},
extra tick style={grid=major}]
\addplot {x};
\draw (axis cs:0,0) circle[radius=2];
\end{axis}
\end{tikzpicture}

\end{comment}
%%%%%%%%%%%%%%%%%%%


\title{Phys 230A -- QFT -- Lec02}
\author{UCLA, Fall 2014}
\date{\formatdate{08}{10}{2014}} % Activate to display a given date or no date (if empty),
         % otherwise the current date is printed 

\begin{document}
\maketitle


\section{More on scalar fields}

Take our action
\begin{equation}
S = \int d^4 x \mathcal{L}(\phi, \pd_\mu \phi)
\end{equation}
then the variation $\delta S = 0$ of the action implies the Euler-Lagrange equation 
\begin{equation}
\pd_\mu \frac{\pd \mathcal{L}}{\pd(\pd_\mu \phi)} - \frac{\pd \mathcal{L}}{\pd \phi} = 0.
\end{equation}
For free scalar field we have the Lagrangian
\begin{equation}
\mathcal{L} = \frac{1}{2}(\pd_\mu \phi \pd^\mu \phi - m^2 \phi^2)
\end{equation}
giving the KG equation $\box \phi + m^2 \phi = 0$. 

\subsection{Canonical (Hamiltonian) formulation}

The canonical momentum is given by $\Pi_\phi = \pd \mathcal{L} / \pd \dot\phi$. The Hamiltonian is given by the integral of the Hamiltonian density, 
\begin{equation}
H = \int d^3 x \mathcal{H}(\phi, \nabla\phi, \Pi_\phi).
\end{equation}
The Hamiltonian density is given by $\mathcal{H} = \Pi_\phi \dot\phi - \mathcal{L}$ where $\dot\phi = \dot\phi(\phi, \Pi_\phi)$. More fully written as functions we have
\begin{equation}
\mathcal{H} (\phi, \nabla \phi, \Pi_\phi) = \Pi_\phi \dot\phi(\phi, \Pi_\phi) - \mathcal{L}(\phi, \nabla\phi, \dot\phi(\phi, \Pi_\phi)).
\end{equation}
Of course we have Hamilton's equations 
\begin{equation}
\dot\phi = \frac{\delta H}{\delta \Pi_\phi}, \qquad \dot{\Pi}_\phi = - \frac{\delta H}{\delta \phi}
\end{equation}
Let's check equivalence to Euler-Lagrange. We have
\begin{equation}
\frac{\delta H}{\delta \Pi_\phi} = \dot\phi + \Pi_\phi \frac{\pd \dot\phi}{\pd \Pi_\phi} - \frac{\pd \mathcal{L}}{\pd \dot\phi} \frac{\pd \dot\phi}{\pd \Pi_\phi} = \dot\phi
\end{equation}
where the second and third terms cancel by definitions. Furthermore, we find
\begin{equation}
\frac{\delta H}{\delta \phi} = \Pi_\phi \frac{\pd \dot\phi}{\pd \phi} - \frac{\pd \mathcal{L}}{\pd \phi} - \frac{\pd \mathcal{L}}{\pd \dot\phi} \frac{\pd \dot\phi}{\pd \phi} - \int \frac{\pd \mathcal{L}}{\pd(\pd_i \phi)} \frac{\delta (\pd_i \phi)}{\delta \phi} d^3 x
\end{equation}
and we see that the first and third terms cancel and we can rewrite the last term as 
\begin{equation}
+ \int \pd_i \frac{\pd \mathcal{L}}{\pd(\pd_i\phi)} \frac{\delta \phi}{\delta \phi} d^3 x = \pd_i \frac{\delta \mathcal{L}}{\pd(\pd_i \phi)}
\end{equation}
which is just $-\dot{\Pi}_\phi$. So we see the equivalence of Hamiltonian and Lagrangian mechanics in this formalism:
\begin{align}
0 &= \dot{\Pi}_\phi + \frac{\delta H}{\delta \phi} \\
	&= \dot{\Pi}_\phi - \frac{\pd \mathcal{L}}{\pd \phi} + \pd_i \frac{\pd \mathcal{L}}{\pd (\pd_i \phi)} \\
	&= \frac{\pd}{\pd t} (\frac{\pd \mathcal{L}}{\pd \dot\phi}) + \pd_i \frac{\pd \mathcal{L}}{\pd (\pd_i \phi)} - \frac{\delta \mathcal{L}}{\pd \phi} \\
	&= \pd_\mu \frac{\pd \mathcal{L}}{\pd(\pd_\mu \phi)} - \frac{\pd \mathcal{L}}{\pd \phi}.
\end{align}

Now if we have the action
\begin{equation}
S = \int d^4 x (\frac{1}{2} \pd_\mu \phi \pd^\mu \phi - V(\phi))
\end{equation}
we can write the Hamiltonian
\begin{equation}
H = \int d^3x (\frac{1}{2} \Pi_\phi^2 + \frac{1}{2} \nabla\phi \cdot \nabla \phi + V(\phi)).
\end{equation}
Then we can write the variation
\begin{align}
\delta H(\phi, \Pi_\phi) &= \int d^3x (\frac{\delta H}{\delta \phi} \delta \phi + \frac{\delta H}{\delta \Pi_\phi} \delta Pi_\phi) \\
	&= \int d^3x (\frac{\delta H}{\delta \phi} \dot\phi + \frac{\delta H}{\delta \Pi_\phi} \dot{\Pi}_\phi) dt
\end{align}
and thus
\begin{align}
\frac{dH}{dt} &= \int d^3x (\frac{\delta H}{\delta \phi} \dot\phi + \frac{\delta H}{\delta \Pi_\phi} \dot{\Pi}_\phi) \\
	&= -\dot{\Pi}_\phi \dot\phi + \dot\phi \dot{\Pi}_\phi \\
	&= 0.
\end{align}


\subsection{Canonical quantization}

In QM, we start with a canonical pair $(p,q)$ and impose the commutation relation $[q(t), p(t)] = i$. Then the time derivatives are given by 
\begin{equation}
\dot{q}(t) = i[H, q(t)], \qquad \dot{p}(t) = i[H, p(t)]
\end{equation}
in fact this works for any observable. 

For the simple harmonic oscillation, we have
\begin{equation}
S = \int dt (\frac{1}{2} m \dot{q}^2 - \frac{1}{2} m \omega^2 q^2)
\end{equation}
so that
\begin{equation}
p = \frac{\pd L}{\pd \dot q} = m \dot q, \qquad H = p \dot q - L = \frac{p^2}{2m} + \frac{1}{2} m \omega^2 q^2.
\end{equation}
Then we have
\begin{equation}
\dot q = i[H, q] = p/m, \qquad \dot p = i[H, p] = -m \omega^2 q
\end{equation}
We will be working in the Heisenberg picture where the operators evolve in time rather than the Schroedinger picture where the wavefunctions evolve in time. So we will have coordinate $\psi = \psi(q)$, $p(0) = -i \pd / \pd q$. In QFT will have $\psi[\phi]$.

We also have of course the ladder operators
\begin{align}
a &= \frac{ip}{\sqrt{2m\omega}} + \sqrt{m \omega / 2} q \\
a^\dagger &= -\frac{ip}{\sqrt{2m\omega}} + \sqrt{m\omega / 2} q
\end{align}
where $[a, a^\dagger] = 1$. We write the Hamiltonian
\begin{equation}
H = \frac{1}{2} \omega (a^\dagger a + a a^\dagger) = (a^\dagger a + \frac{1}{2}) \omega.
\end{equation}
So we have ground state $a\ket{0} = 0$ with $H\ket{0} = \frac{1}{2} \omega \ket{0}$. The excited states are written 
\begin{align}
\ket{n} &= N (a^\dagger)^n \ket{0} \\
H \ket{n} &= (n+1/2) \omega \ket{n}.
\end{align}
Furthermore we have $\dot{a} = i[H, a] = -i \omega a$ so that
\begin{equation}
a(t) = a(0) e^{-i\omega t}, \qquad a^\dagger(t) = a^\dagger(0) e^{i \omega t}.
\end{equation}
Finally note that we can write
\begin{align}
q(t) &= \frac{1}{\sqrt{2m \omega}} (a(t) + a^\dagger(t)) \\
	&= \frac{1}{\sqrt{2m \omega}} (a(0) e^{-i\omega t} + a^\dagger(0) e^{i\omega t}).
\end{align}
So we can now port this all over to scalar fields where $q \rightarrow \phi$ and $p \rightarrow \Pi_\phi$. 


\subsection{Quantizing the scalar field}

Given the action 
\begin{equation}
S = \int d^4x (\frac{1}{2} (\pd \phi)^2 - \frac{1}{2} m^2 \phi^2)
\end{equation}
we have $\Pi_\phi = \pd \mathcal{L} / \pd \dot\phi = \dot\phi$. Then the hamiltonian is written
\begin{equation}
\int d^3x (\frac{1}{2} \Pi_\phi^2 + \frac{1}{2} (\nabla \phi)^2 + \frac{1}{2} m^2 \phi^2)
\end{equation}
and we impose the commutation relation 
\begin{equation}
[\phi(\v{x},t), \Pi_\phi(\v{y},t)] = i \delta^3 (\v{x} - \v{y}).
\end{equation}
(A key idea to note is that the Hamiltonian is manifestly unitary but not Lorentz invariant, while the Lagrangian is manifestly Lorentz invariant but not unitary.) 

Now we can compute 
\begin{equation}
\dot\phi = i[H, \phi], \quad \text{and} \quad \dot{\Pi}_\phi = i[H, \Pi_\phi].
\end{equation}
We see that the first equation reproduces $\dot\phi = \Pi_\phi$. For the second equation, we have
\begin{align}
\dot{\Pi}_\phi &= i \int d^3 y [\frac{1}{2} (\nabla \phi(y))^2 + \frac{1}{2} m^2 \phi^2(\v{y}, t), \Pi_\phi (\v{x},t) ] \\
	&= i \int d^3y \left\{ \nabla \phi(\v{y}) \cdot \nabla_y [ \phi(\v{y},t), \Pi_\phi(\v{x},t)] + m^2 \phi(\v{y},t) [\phi(\v{y},t), \Pi_\phi(\v{x},t)] \right\} \\
	&= - \int d^3y \left\{ \nabla \phi(\v{y}, t) \cdot \nabla_y \delta^3 (\v{y} - \v{x}) + m^2 \phi(\v{y},t) \delta^3(\v{y} - \v{x}) \right\} \\
	&= \nabla^2 \phi(\v{x},t) - m^2 \phi(\v{x},t).
\end{align}
Notice that since $\dot{\Pi}_\phi = \ddot\phi$, this just reproduces the KG equation for a free field, $\Box \phi + m^2 \phi = 0$. 


\subsection{Mode expansion}

First we'll do a spatial Fourier expansion. 
\begin{equation}
\phi(\v{x},t) = \int \frac{d^3p}{(2\pi)^3} e^{i \v{p} \cdot \v{x}} \tilde\phi (\v{p}, t).
\end{equation}
We can rewrite the KG equation in momentum space
\begin{equation}
\ddot{\tilde \phi} (\v{p}, t) + (\v{p}^2 + m^2) \tilde{\phi} (\v{p},t) = 0
\end{equation}
which has solutions
\begin{equation}
\tilde{\phi}(\v{p}, t) = \tilde\phi (\v p, 0) e^{\pm i \omega_p t}, \qquad \omega_p = \sqrt{\v{p}^2 + m^2}.
\end{equation}
Thus we can write $\phi$ in spatial coordinates
\begin{equation}
\phi(\v x, t) = \int \frac{d^3p}{(2\pi)^3} \frac{1}{\sqrt{2 \omega_p}} \left( a_{\v{p}} e^{-i \omega_p t + i \v{p} \cdot \v{x}} + a_p^\dagger e^{i \omega_p t - i \v{p} \cdot \v{x}} \right).
\end{equation}
Notice that this is a real field: $\phi^\dagger = \phi$. Then we have
\begin{equation}
\Pi_\phi = \dot\phi = - \int \frac{d^3p}{(2\pi)^3} i \sqrt{\omega_p/2} \left( a_{\v{p}} e^{-i \omega_p t + i \v{p} \cdot \v{x}} - a_p^\dagger e^{i \omega_p t - i \v{p} \cdot \v{x}} \right). 
\end{equation}
Now we want to know, what is $[a_{\v{p}}, a_{\v{p}'}^\dagger]$ ? We have
\begin{align}
a_{\v{p}} &= \int d^3 x e^{-i \v{p} \cdot \v{x}} \left( \sqrt{\omega_p / 2} \phi(\v{x}, 0) + \frac{i}{\sqrt{2 \omega_p}} \Pi_\phi (\v{x}, 0) \right) \\
a_{\v{p}}^\dagger &= \int d^3 x e^{+i \v{p} \cdot \v{x}} \left( \sqrt{\omega_p / 2} \Pi_\phi(\v{x}, 0) - \frac{i}{\sqrt{2 \omega_p}} \phi (\v{x}, 0) \right). 
\end{align}
Then we have the commutator
\begin{align}
[a_{\v{p}}, a_{\v{p}'}^\dagger] &= - \int d^3x d^3x' e^{-i\v{p} \cdot \v{x} + i\v{p}' \cdot \v{x}'} \frac{i}{2} \left( [\phi(\v{x},0), \Pi_\phi(\v{x}', 0)] + [\phi(\v{x},0), \Pi_\phi(\v{x}, 0)] \right) \\
	&= \int d^3x e^{-i(\v{p} - \v{p}') \cdot \v{x}} \\
	&= (2\pi)^3 \delta^3 (\v{p} - \v{p}').
\end{align}
and of course $[a_{\v{p}}, a_{\v{p}'}] = [a_{\v{p}}^\dagger, a_{\v{p}'}^\dagger] = 0$. 

As an exercise, show:
\begin{equation}
H = \int \frac{d^3p}{(2\pi)^3} \omega_{\v{p}} ( a_{\v{p}}^\dagger a_{\v{p}} + \frac{1}{2} \underbrace{[a_{\v{p}}, a_{\v{p}}^\dagger]}_\text{$(2\pi)^3 \delta^3(0) = \infty$ ???} )
\end{equation}


\subsection{Spectrum}

In the vacuum, $a_{\v p} \ket{0} = 0$ for all $\v{p}$ so $H \ket{0} = 0 \ket{0}$. But we have
\begin{equation}
H a_{\v{p}}^\dagger \ket{0} = [H, a_{\v{p}}^\dagger] \ket{0} = \omega_{\v{p}} a_{\v{p}}^\dagger \ket{0}
\end{equation}
where again $\omega_{\v p} = \sqrt{ \v{p}^2 + m^2}$. This is what we really mean when we say that particle are the quanta of the fields in QFT. Furthermore, we find
\begin{equation}
\ket{\v{p}, \v{p}'} = a_{\v p}^\dagger a_{\v{p}'}^\dagger \ket{0} = \ket{\v{p}', \v{p}}
\end{equation}
so we find that scalar fields obey Bose statistics. We also have
\begin{equation}
H \ket{\v{p}, \v{p}'} = (\omega_{\v p} + \omega_{\v{p}'}) \ket{\v{p}, \v{p}'}.
\end{equation}
A general state can be written
\begin{equation}
\ket{\psi} = \ket{0} + \int \frac{d^3p}{(2\pi)^3} f_1(\v p) a_{\v p}^\dagger \ket{0} + \int \frac{d^3p d^3p'}{(2\pi^3)(2\pi)^3} f_2(\v p, \v p') a_{\v p}^\dagger a_{\v p'}^\dagger \ket{0}.
\end{equation}
This is referred to as Fock space. 

What is the energy of states in the CM frame $\v{p}_\text{tot} = 0$? 
\begin{itemize}
\item in vacuum: $E = 0$
\item 1 particle: $E = m$
\item 2 identical particles: $E = 2\sqrt{\v{p}^2 + m^2}$
\end{itemize}
Gapped vs. gapless? $E < E_gap = m$ trivial. nontrivial IR dynamics? [???]







\begin{comment}
\begin{figure}
\centering
\includegraphics{3a.pdf}
\caption{Half the diagrams for photon-photon scattering.}
\label{fig:3a}
\end{figure}
\end{comment}



\end{document}
