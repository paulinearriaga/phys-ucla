% declare document class and geometry
%\documentclass[12pt]{article} % use larger type; default would be 10pt
%\usepackage[margin=1in]{geometry} % handle page geometry

% ***********************************************************
% ******************* PHYSICS HEADER ************************
% ***********************************************************
% Version 2
\documentclass[12pt]{article} 





\usepackage{datetime} % allows easy formatting of dates, e.g. \formatdate{dd}{mm}{yyyy}

\usepackage{amsmath} % AMS Math Package
\usepackage{amsthm} % Theorem Formatting
\usepackage{amssymb}	% Math symbols such as \mathbb
\usepackage{graphicx} % Allows for eps images
\usepackage{multicol} % Allows for multiple columns
\usepackage[dvips,letterpaper,margin=1in,bottom=1in]{geometry}
 % Sets margins and page size
\pagestyle{empty} % Removes page numbers
\makeatletter % Need for anything that contains an @ command 
%\renewcommand{\maketitle} % Redefine maketitle to conserve space
%{ \begingroup \vskip 10pt \begin{center} \large {\bf \@title}
%	\vskip 10pt \large \@author \hskip 20pt \@date \end{center}
%  \vskip 10pt \endgroup \setcounter{footnote}{0} }
\makeatother % End of region containing @ commands
\renewcommand{\labelenumi}{(\alph{enumi})} % Use letters for enumerate
% \DeclareMathOperator{\Sample}{Sample}
\let\vaccent=\v % rename builtin command \v{} to \vaccent{}
\renewcommand{\v}[1]{\ensuremath{\mathbf{#1}}} % for vectors
\newcommand{\gv}[1]{\ensuremath{\mbox{\boldmath$ #1 $}}} 
% for vectors of Greek letters
\newcommand{\vx}{\ensuremath{\v{x}}} 
% for vectors of Greek letters
\newcommand{\vy}{\ensuremath{\v{y}}} 
% for vectors of Greek letters
\newcommand{\xdot}{\ensuremath{\dot{x}}} 
% for vectors of Greek letters

\newcommand{\ydot}{\ensuremath{\dot{y}}} 
% for vectors of Greek letters
\usepackage{commath} % for some nice standardized syntax stuff. 
	% \dif, \Dif, \od, \pd, \md, \(abs | envert), \(norm | enVert), \(set | cbr), \sbr, \eval, \int(o | c)(o | c), etc
\newcommand{\bbar}[1]{\bar{\bar{#1}}} % for barring things twice -- use \cbar or \zbar instead of original \bbar

\newcommand{\uv}[1]{\ensuremath{\mathbf{\hat{#1}}}} % for unit vector
%\newcommand{\abs}[1]{\left| #1 \right|} % for absolute value
\newcommand{\avg}[1]{\left< #1 \right>} % for average
\let\underdot=\d % rename builtin command \d{} to \underdot{}
\renewcommand{\d}[2]{\frac{d #1}{d #2}} % for derivatives
\newcommand{\dd}[2]{\frac{d^2 #1}{d #2^2}} % for double derivatives
%\newcommand{\pd}[2]{\frac{\partial #1}{\partial #2}} 
% for partial derivatives
\newcommand{\fd}[2]{\frac{\delta #1}{\delta #2}} 
% for functional derivatives

\newcommand{\pdd}[2]{\frac{\partial^2 #1}{\partial #2^2}} 
% for double partial derivatives
\newcommand{\pdc}[3]{\left( \frac{\partial #1}{\partial #2}
 \right)_{#3}} % for thermodynamic partial derivatives
\newcommand{\ket}[1]{\left| #1 \right>} % for Dirac bras
\newcommand{\bra}[1]{\left< #1 \right|} % for Dirac kets
\newcommand{\braket}[2]{\left< #1 \vphantom{#2} \right|
 \left. #2 \vphantom{#1} \right>} % for Dirac brackets
\newcommand{\matrixel}[3]{\left< #1 \vphantom{#2#3} \right|
 #2 \left| #3 \vphantom{#1#2} \right>} % for Dirac matrix elements
\newcommand{\grad}[1]{\gv{\nabla} #1} % for gradient
\let\divsymb=\div % rename builtin command \div to \divsymb
\renewcommand{\div}[1]{\gv{\nabla} \cdot #1} % for divergence
\newcommand{\curl}[1]{\gv{\nabla} \times #1} % for curl
\let\baraccent=\= % rename builtin command \= to \baraccent
\renewcommand{\=}[1]{\stackrel{#1}{=}} % for putting numbers above =
\newtheorem{prop}{Proposition}
\newtheorem{thm}{Theorem}[section]
\newtheorem{lem}[thm]{Lemma}
\theoremstyle{definition}
\newtheorem{dfn}{Definition}
\theoremstyle{remark}
\newtheorem*{rmk}{Remark}
\newcommand{\bigO}{\mathcal{O}} % big O notation
\let \bigo = \bigO % deprecated version. keeping for now because need to update instances in older files










\makeatletter
% À droite
\renewcommand\subsection{\@startsection {subsection}{1}{\z@}%
                                   {-3.5ex \@plus -1ex \@minus -.2ex}%
                                   {2.3ex \@plus.2ex}%
                                   {\raggedright\normalfont\Large\bfseries}}
\makeatother


\makeatletter
\def\section{\@ifstar\unnumberedsection\numberedsection}
\def\numberedsection{\@ifnextchar[%]
  \numberedsectionwithtwoarguments\numberedsectionwithoneargument}
\def\unnumberedsection{\@ifnextchar[%]
  \unnumberedsectionwithtwoarguments\unnumberedsectionwithoneargument}
\def\numberedsectionwithoneargument#1{\numberedsectionwithtwoarguments[#1]{#1}}
\def\unnumberedsectionwithoneargument#1{\unnumberedsectionwithtwoarguments[#1]{#1}}
\def\numberedsectionwithtwoarguments[#1]#2{%
  \ifhmode\par\fi
  \removelastskip
  \vskip 5ex\goodbreak
  \refstepcounter{section}%
  \hbox to \hsize{\vbox{%
      \noindent
      \leavevmode
      \begingroup
      \Large\bfseries\raggedleft
      \thesection.\ 
      #2\par
      \endgroup
      \vskip -2ex
      \noindent\hrulefill
      \vskip -2.2ex\nobreak
      \noindent\hrulefill
      }}\nobreak
  \vskip 2ex\nobreak
  \addcontentsline{toc}{section}{%
    \protect\numberline{\thesection}%
    #1}%
  }
\def\unnumberedsectionwithtwoarguments[#1]#2{%
  \ifhmode\par\fi
  \removelastskip
  \vskip 5ex\goodbreak
%  \refstepcounter{section}%
  \hbox to \hsize{\vbox{%
      \noindent
      \leavevmode
      \begingroup
      \Large\bfseries\raggedleft
%      \thesection.\ 
      #2\par
      \endgroup
      \vskip -2ex
      \noindent\hrulefill
      \vskip -2.2ex\nobreak
      \noindent\hrulefill
      }}\nobreak
  \vskip 2ex\nobreak
  \addcontentsline{toc}{section}{%
%    \protect\numberline{\thesection}%
    #1}%
  }
\makeatother
\pagestyle{empty}




% ***********************************************************
% ********************** END HEADER *************************
% ***********************************************************


\title{Phys 220A -- Classical Mechanics -- Lec02}
\author{UCLA, Fall 2014}
%\date{\formatdate{07}{10}{2014}} % Activate to display a given date or no date (if empty),
         % otherwise the current date is printed 

\begin{document}
\setlength{\unitlength}{1mm}
\maketitle


\section{Introduction}

A couple of general comments about Newton's laws. 

\subsection{Something funny about Lorentz force and magnetic fields}

Consider the Lorentz force between two charged particles, given by $\v{F}_{12} = q \v{v}_1 \times \v{B}_2$. Consider particle 1 to lie on the positive $y$ axis with velocity in the positive $y$ direction and particle 2 to lie on the positive $z$ axis with velocity in the positive $x$ axis. Then we find that $\v{F}_{12} \neq 0$ points in the positive $x$ direction yet $\v{F}_{21} = 0$. 

What's going on here? Is this a case of violation of angular momentum conservation? No, in fact the EM field itself carries angular momentum. 


\subsection{Virial Theorem (useful in stat. mech)}

Take some function $f$ and write the time average
\begin{equation}
\avg{f} = \lim_{T \rightarrow \infty} \frac{1}{T} \int_{t_0}^{t_0 + T} f(t) dt.
\end{equation}
Note that there is no dependence on $t_0$ for a bounded quantity,
\begin{equation}
\pd{}{t_0} \avg{f_0} = \lim_{T \rightarrow \infty} \frac{f(t_0 + T) - f(t_0)}{T} = 0
\end{equation}
which vanishes because the numerator in the limit is bounded. 

Next, we hope to obtain an expression for $\avg{T}$ for conservative forces. Notice that 
\begin{equation}
\sum_i m_i \ddot{\v{x}}_i \cdot \v{x} = \sum_i \v{F}_i \cdot \v{x}_i,
\end{equation}
which we can rewrite as 
\begin{equation}
\frac{d}{dt} \left( \sum_i m_i \dot{\v{x}}_i \cdot \v{x} \right) - \underbrace{\sum_i m_i \dot{\v{x}}_i^2}_{2T} = - \sum_i \v{\nabla}_i V \cdot \v{x}_i.
\end{equation}
If the motion is bounded then the derivative goes away when we average, so averaging over time we find
\begin{equation}
2 \avg{T} = \avg{ \sum_i \v{\nabla}_i V \cdot \v{x}_i }.
\end{equation}
This is the Virial theorem. It is especially useful if $V$ is homogeneous, i.e. 
\begin{equation}
V(\lambda \v{x}_1, \dots \lambda \v{x}_n) = \lambda^k V(\v{x}_1, \dots, \v{x}_n)
\end{equation}
for some constant $k$ and any value of $\lambda$. Taking the derivative of this equation w.r.t. $\lambda$ we have
\begin{equation}
\sum_i \v{\nabla}_i V(\lambda \v{x}_1, \dots, \lambda \v{x}_n) \cdot \v{x}_i = k \lambda^{k-1} V(\v{x}_1, \dots, \v{x}_n)
\end{equation}
and setting $\lambda = 1$ we find $\sum_i \v{\nabla} V \cdot \v{x}_i = k V$. Thus the Virial theorem for a homogeneous potential gives us
\begin{equation}
2 \avg{T} = k \avg{V}.
\end{equation}

Let's look at some simple systems. For the harmonic oscillator
\begin{equation}
V = \frac{1}{2} kx^2
\end{equation}
So our parameter from above $k$ is 2, so virial theorem for this system is:
\begin{equation}
\avg{T} = \avg{V}
\end{equation}
For a Keplerian potential it's:
\begin{equation}
V = \frac{\alpha}{r}
\end{equation}
So k = 1 and Virial theorem is
\begin{equation}
\avg{T} = -\frac{1}{2} \avg{V}
\end{equation}
\textbf{Note that Virial theorem in general only holds for bound motion with no scattering}


\section{Simple systems}
\subsection{1-D Harmonic Oscillator}
Let's look at the solutions to some systems

Hookes law:
\begin{equation}
F = -kx 
\end{equation}
\begin{equation}
m\ddot{x} = -kx
\end{equation}
\begin{equation}
\ddot{x} + \frac{k}{m} x = 0
\end{equation}
This system has a very general ansatz where $\omega = \frac{k}{m}$
\begin{equation}
x(t) = c_1 \sin\omega t + c_2 \cos\omega t
\end{equation}
\begin{equation}
x(t) = c\sin(\omega t + \phi_0)
\end{equation}
\begin{equation}
c_1 e^{i\omega t} + c_2^* e^{-i \omega t}
\end{equation}
There are 2 integration constants that determine x and $\dot{x}$ at t = 0. We can also write the momentum of this system
\begin{equation}
p = m\dot{x} = mc_1 \omega \cos(\omega t + \phi_0)
\end{equation}
\begin{equation}
p^2 = m^2 \omega^2 (c_1^2 - x^2)
\end{equation}
\begin{equation}
p^2 + \omega x^2 = m^2 \omega^2
\end{equation}
Looking at this in phase space, we get closed circles because this is a conservative field. The origin is the equilibrium point and the circles do not intersect. This changes if there is friction, and for driven harmonic oscillator

\subsection{Particle in Constant Magnetic Field}
Let's set up a constant magnetic field in the $\hat{z}$ direction. Then our forces are 
\begin{equation}
\v{F} = e\v{v} \times \v{B}
\end{equation}
In tensor notation
\begin{equation}
F_i = e \epsilon_{ijk} v_j B_k
\end{equation}
Which results in 
\begin{equation}
m\ddot{z} = -0
\end{equation}
\begin{equation}
m\ddot{x} = B e \dot{y}
\end{equation}
\begin{equation}
m\ddot{y} = -B e\dot{x}
\end{equation}
This is a systme of two first order differential equations. Let's make a couple definitions
\begin{equation}
\omega_B = \frac{Be}{m}
\end{equation}
\begin{equation}
\dot{x} = S_1
\end{equation}
\begin{equation}
\dot{y} = S_2
\end{equation}
So now our equations become
\begin{equation}
\dot{S_1} = \omega_B S_2
\end{equation}
\begin{equation}
\dot{S_2} = -\omega_B S_1
\end{equation}
Let's write this in matrix form
\begin{equation}
\d{}{t} \left(\begin{array}{c} S_1 \\S_2\end{array}\right) = M \left(\begin{array}{c} S_1 \\S_2\end{array}\right)
\end{equation}
where
\begin{equation}
M = \left(\begin{array}{cc} 0 & \omega_B \\ -\omega_B & 0 \end{array}\right)
\end{equation}
We can solve this by exponentiating and doing a Taylor expansion



\end{document}
