% declare document class and geometry
\documentclass[12pt]{article} % use larger type; default would be 10pt
\usepackage[margin=1in]{geometry} % handle page geometry

% import packages and commands
% standard packages
\usepackage{graphicx} % support the \includegraphics command and options
\usepackage{amsmath} % for nice math commands and environments

% font packages
\usepackage{amssymb} % for \mathbb, \mathfrak fonts
\usepackage{mathrsfs} % for \mathscr font
\DeclareMathAlphabet{\mathpzc}{OT1}{pzc}{m}{it} % defines \mathpzc for Zapf Chancery (standard postscript) font

% other packages
\usepackage{datetime} % allows easy formatting of dates, e.g. \formatdate{dd}{mm}{yyyy}
\usepackage{caption} % makes figure captions better, more configurable
\usepackage{enumitem} % allows for custom labels on enumerated lists, e.g. \begin{enumerate}[label=\textbf{(\alph*)}]
\usepackage[squaren]{SIunits} % for nice units formatting e.g. \unit{50}{\kilo\gram}
\usepackage{cancel} % for crossing out terms with \cancel
\usepackage{verbatim} % for verbatim and comment environments
\usepackage{tensor} % for \indices e.g. M\indices{^a_b^{cd}_e}, and \tensor e.g. \tensor[^a_b^c_d]{M}{^a_b^c_d}
\usepackage{feynmp-auto} % for Feynman diagrams. 
\usepackage{pgfplots} % for plotting in tikzpicture environment

% new commands
\newcommand{\beg}{\begin} % a few letters less for beginning environments
\newenvironment{eqn}{\begin{equation}}{\end{equation}} % a lot fewer letter for equation environment

% notational commands
\newcommand{\opname}[1]{\operatorname{#1}} % custom operator names
\newcommand{\fslash}[1]{#1\!\!\!/} % feynman slash
\newcommand{\pd}{\partial} % partial differential shortcut
\newcommand{\ket}[1]{\left| #1 \right>} % for Dirac kets
\newcommand{\bra}[1]{\left< #1 \right|} % for Dirac bras
\newcommand{\braket}[2]{\left< #1 \vphantom{#2} \right| 
	\left. #2 \vphantom{#1} \right>} % for Dirac brackets
%\let\underdot=\d % rename builtin command \d{} to \underdot{}
%\renewcommand{\d}[2]{\frac{d #1}{d #2}} % for derivatives
%\newcommand{\pd}[2]{\frac{\partial #1}{\partial #2}} % for partial derivatives
%\newcommand{\fd}[2]{\frac{\delta #1}{\delta #2}} % for functional derivatives
\let\vaccent=\v % rename builtin command \v{} to \vaccent{}
%\renewcommand{\v}[1]{\ensuremath{\mathbf{#1}}} % for vectors
\renewcommand{\v}[1]{\ensuremath{\boldsymbol{\mathbf{#1}}}} % for vectors
%\newcommand{\gv}[1]{\ensurmath{\mbox{\boldmath$ #1 $}}} % for vectors of Greek letters
\newcommand{\uv}[1]{\ensuremath{\boldsymbol{\mathbf{\widehat{#1}}}}} % for unit vectors
\newcommand{\abs}[1]{\left| #1 \right|} % for absolute value ||x||
%\newcommand{\mag}{\abs} % magnitude, just another name for \abs
\newcommand{\norm}[1]{\left\Vert #1 \right\Vert} % for norm ||v||
\newcommand{\avg}[1]{\left< #1 \right>} % for average <x>
\newcommand{\inner}[2]{\left< #1, #2 \right>} % for inner product <x,y>
\newcommand{\set}[1]{ \left\{ #1 \right\} } % for sets {a,b,c,...}
\newcommand{\tr}{\opname{tr}} % for trace
\newcommand{\Tr}{\opname{Tr}} % for Trace

% notational shortcuts
\newcommand{\reals}{\mathbb{R}} % real numbers
\newcommand{\complexes}{\mathbb{C}} % complex numbers
\newcommand{\nats}{\mathbb{N}} % natural numbers
\newcommand{\irrats}{\mathbb{Q}} % irrationals
\newcommand{\quats}{\mathbb{H}} % quaternions (a la Hamilton)
\newcommand{\euclids}{\mathbb{E}} % Euclidean space
\newcommand{\bigo}{\mathcal{O}} % big O notation
\newcommand{\Lag}{\mathcal{L}} % fancy Lagrangian
\newcommand{\Ham}{\mathcal{H}} % fancy Hamiltonian





%%%%%%%%%%%%%%%%%%%
% some templates for various things
\begin{comment}

% template for figures
\begin{figure}
\centering
\includegraphics{myfile.png}
\caption{This is a caption}
\label{fig:myfigure}
\end{figure}

% template for Feynman diagrams using feynmf/feynmp
\begin{fmfgraph*}(40,25)
\fmfleft{em,ep}
\fmf{fermion}{em,Zee,ep}
\fmf{photon,label=$Z$}{Zee,Zff}
\fmf{fermion}{fb,Zff,f}
\fmfright{fb,f}
\fmfdot{Zee,Zff}
\end{fmfgraph*}

% template for drawing plots with pgfplot
\pgfplotsset{compat=1.3,compat/path replacement=1.5.1}
\begin{tikzpicture}
\begin{axis}[
extra x ticks={-2,2},
extra y ticks={-2,2},
extra tick style={grid=major}]
\addplot {x};
\draw (axis cs:0,0) circle[radius=2];
\end{axis}
\end{tikzpicture}

\end{comment}
%%%%%%%%%%%%%%%%%%%



\title{Phys 220A -- Classical Mechanics -- HW01}
\author{UCLA, Fall 2014}
\date{\formatdate{09}{10}{2014}} % Activate to display a given date or no date (if empty),
         % otherwise the current date is printed 

\begin{document}
\maketitle


\section*{Problem 1: Math recap (15 pts)}
\begin{description}

\item[(a)]
\textit{
Show that
\begin{equation}
(\v{a} \times \v{b} ) \times \v{c} = (\v{a} \cdot \v{c}) \v{b} - (\v{b} \cdot \v{c}) \v{a}.
\end{equation}
}


\item[(b)]
\textit{
Spherical coordinates $r, \theta, \phi$ are defined as 
\begin{equation}
x = r \sin\theta \cos\phi, \qquad y = r \sin\theta \sin\phi, \qquad z = r \cos\theta.
\end{equation}
Express the cartesian differential operators 
\begin{equation}
\v{\nabla} = (\pd_x, \pd_y, \pd_z), \qquad \v{\nabla}^2 = \pd_x^2 + \pd_y^2 + \pd_z^2
\end{equation}
in spherical coordinates. 
}


\item[(c)]
\textit{
Show the following identities:
\begin{align}
\v{\nabla} \times (\v{\nabla} f(\v{x})) &= 0, \text{ and} \\
\v{\nabla} \cdot (\v{\nabla} \times \v{a}) &= 0.
\end{align}
}


\end{description}



\section*{Problem 2: Conservation is important in a drought! (15 pts)}

\begin{description}

\item[(a)]
\textit{
Show that in an isolated system of $n$ point masses $m_i$ (no external forces), the momentum $\v{p}_\text{tot} = \sum_{i=1}^n \v{p}_i$ is conserved. 
}


\item[(b)]
\textit{
Show that in an isolated system of $n$ point masses $m_i$ (no external forces) the total angular momentum $\v{L}_\text{tot} = \sum_{i=1}^n \v{r}_i \times \v{p}_i$ is conserved. 
}


\item[(c)]
\textit{
What about the total energy of such an isolated system. Is it necessarily conserved? 
}


\end{description}



\section*{Problem 3: Work, work work (15 pts)}

\textit{
Check whether the following forces are conservative by calculating
\begin{equation}
W_{if} = \int_{i \rightarrow f} d\v{x} \cdot \v{F}
\end{equation}
for two different paths. (If you believe the force is non-conservative try to find a path which makes this clear.) 
}

\begin{description}

\item[(a)]
\textit{
For a central force $\v{F} = r^2 \uv{r}$.
}


\item[(b)]
\textit{
For the Lorentz force $\v{F} = -e\v{v} \times \v{B}$ with a constant magnetic field $\v{B}$. 
}


\item[(c)]
\textit{
For a friction force $\v{F} = -b\v{v}$.
}


\end{description}


\section*{Problem 4: Central motion (30 pts)}
\textit{
A planet of mass $m$ is moving around a Sun of mass $M$ subject to Newtonian gravitational force. 
}

\begin{description}

\item[(a)]
\textit{
Write down the Newtonian equation of motion for the planet and the sun, and show that they can be separated into a free motion of the center of mass and a motion in the central potential of a mass point with reduced mass $\mu$.
}


\item[(b)]
\textit{
Show that conservation of angular momentum implies that the motion of the mass point lies in a plane with coordinates $\rho, \phi$.
}


\item[(c)]
\textit{
Use the substitution $u = 1 / \rho$ to write down a differential equation for the trajectory $u = u(\phi)$.
}


\item[(d)]
\textit{
What is the equilibrium solution of this equation? What does it represent?
}


\item[(e)]
\textit{
If the planet is not initially on the equilibrium orbit, there will be small oscillations around the equilibrium point. What is the period of these oscillations?
}


\item[(f)]
\textit{
Assume there is a perturbing potential $V = -B / \rho^2$, calculate the effect of this perturbation on the orbit $u(\phi)$.
}


\end{description}





\end{document}
