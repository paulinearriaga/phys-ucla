% declare document class and geometry
\documentclass[12pt]{article} % use larger type; default would be 10pt
\usepackage[margin=1in]{geometry} % handle page geometry

% standard packages
\usepackage{graphicx} % support the \includegraphics command and options
\usepackage{amsmath} % for nice math commands and environments

% font packages
\usepackage{amssymb} % for \mathbb, \mathfrak fonts
\usepackage{mathrsfs} % for \mathscr font
\DeclareMathAlphabet{\mathpzc}{OT1}{pzc}{m}{it} % defines \mathpzc for Zapf Chancery (standard postscript) font

% other packages
\usepackage{datetime} % allows easy formatting of dates, e.g. \formatdate{dd}{mm}{yyyy}
\usepackage{caption} % makes figure captions better, more configurable
\usepackage{enumitem} % allows for custom labels on enumerated lists, e.g. \begin{enumerate}[label=\textbf{(\alph*)}]
\usepackage[squaren]{SIunits} % for nice units formatting e.g. \unit{50}{\kilo\gram}
\usepackage{cancel} % for crossing out terms with \cancel
\usepackage{verbatim} % for verbatim and comment environments
\usepackage{tensor} % for \indices e.g. M\indices{^a_b^{cd}_e}, and \tensor e.g. \tensor[^a_b^c_d]{M}{^a_b^c_d}
\usepackage{feynmp-auto} % for Feynman diagrams. 
\usepackage{pgfplots} % for plotting in tikzpicture environment

% new commands
\newcommand{\beg}{\begin} % a few letters less for beginning environments
\newenvironment{eqn}{\begin{equation}}{\end{equation}} % a lot fewer letter for equation environment

% notational commands
\newcommand{\opname}[1]{\operatorname{#1}} % custom operator names
\newcommand{\fslash}[1]{#1\!\!\!/} % feynman slash
\newcommand{\pd}{\partial} % partial differential shortcut
\newcommand{\ket}[1]{\left| #1 \right>} % for Dirac kets
\newcommand{\bra}[1]{\left< #1 \right|} % for Dirac bras
\newcommand{\braket}[2]{\left< #1 \vphantom{#2} \right| 
	\left. #2 \vphantom{#1} \right>} % for Dirac brackets
%\let\underdot=\d % rename builtin command \d{} to \underdot{}
%\renewcommand{\d}[2]{\frac{d #1}{d #2}} % for derivatives
%\newcommand{\pd}[2]{\frac{\partial #1}{\partial #2}} % for partial derivatives
%\newcommand{\fd}[2]{\frac{\delta #1}{\delta #2}} % for functional derivatives
\let\vaccent=\v % rename builtin command \v{} to \vaccent{}
%\renewcommand{\v}[1]{\ensuremath{\mathbf{#1}}} % for vectors
\renewcommand{\v}[1]{\ensuremath{\boldsymbol{\mathbf{#1}}}} % for vectors
%\newcommand{\gv}[1]{\ensurmath{\mbox{\boldmath$ #1 $}}} % for vectors of Greek letters
\newcommand{\uv}[1]{\ensuremath{\boldsymbol{\mathbf{\widehat{#1}}}}} % for unit vectors
\newcommand{\abs}[1]{\left| #1 \right|} % for absolute value ||x||
%\newcommand{\mag}{\abs} % magnitude, just another name for \abs
\newcommand{\norm}[1]{\left\Vert #1 \right\Vert} % for norm ||v||
\newcommand{\avg}[1]{\left< #1 \right>} % for average <x>
\newcommand{\inner}[2]{\left< #1, #2 \right>} % for inner product <x,y>
\newcommand{\set}[1]{ \left\{ #1 \right\} } % for sets {a,b,c,...}
\newcommand{\tr}{\opname{tr}} % for trace
\newcommand{\Tr}{\opname{Tr}} % for Trace

% notational shortcuts
\newcommand{\reals}{\mathbb{R}} % real numbers
\newcommand{\complexes}{\mathbb{C}} % complex numbers
\newcommand{\nats}{\mathbb{N}} % natural numbers
\newcommand{\irrats}{\mathbb{Q}} % irrationals
\newcommand{\quats}{\mathbb{H}} % quaternions (a la Hamilton)
\newcommand{\euclids}{\mathbb{E}} % Euclidean space
\newcommand{\bigo}{\mathcal{O}} % big O notation
\newcommand{\Lag}{\mathcal{L}} % fancy Lagrangian
\newcommand{\Ham}{\mathcal{H}} % fancy Hamiltonian





%%%%%%%%%%%%%%%%%%%
% some templates for various things
\begin{comment}

% template for figures
\begin{figure}
\centering
\includegraphics{myfile.png}
\caption{This is a caption}
\label{fig:myfigure}
\end{figure}

% template for Feynman diagrams using feynmf/feynmp
\begin{fmfgraph*}(40,25)
\fmfleft{em,ep}
\fmf{fermion}{em,Zee,ep}
\fmf{photon,label=$Z$}{Zee,Zff}
\fmf{fermion}{fb,Zff,f}
\fmfright{fb,f}
\fmfdot{Zee,Zff}
\end{fmfgraph*}

% template for drawing plots with pgfplot
\pgfplotsset{compat=1.3,compat/path replacement=1.5.1}
\begin{tikzpicture}
\begin{axis}[
extra x ticks={-2,2},
extra y ticks={-2,2},
extra tick style={grid=major}]
\addplot {x};
\draw (axis cs:0,0) circle[radius=2];
\end{axis}
\end{tikzpicture}

\end{comment}
%%%%%%%%%%%%%%%%%%%


\title{Phys 220A -- Classical Mechanics -- Lec01}
\author{UCLA, Fall 2014}
\date{\formatdate{02}{10}{2014}} % Activate to display a given date or no date (if empty),
         % otherwise the current date is printed 

\begin{document}
\setlength{\unitlength}{1mm}
\maketitle


\section{Introduction}

Lecture will be Tuesday -- Thursday
\begin{itemize}
\item Ask questions! I will try to make this as interactive as possible
\item I post scans of my notes before class
\end{itemize}

\paragraph{Homework}
\begin{itemize}
\item Problem solving is the most important part of learning
\item Invest serious amount of time in solving the problem yourself
\item It's okay to discuss problems with other students but I would advise on trying to solve them yourself first
\end{itemize}

\paragraph{Exams}
\begin{itemize}
\item Checks and gives feedback on your understanding
\item Preparation for comps
\end{itemize}

\paragraph{Help}
\begin{itemize}
\item During lecture
\item Office hours
\item TA office hours
\item email
\item your peers
\end{itemize}

\paragraph{Goal of course}
\begin{itemize}
\item Review of basic undergraduate mechanics
\item A more mathematically sophisticated look at Lagrangian and Hamiltonian mechanics
\item Provide conceptual underpinning for stat mech and QM
\item Solving more complicated problems
\end{itemize}

[Need to fix: Time of midterm, time for office hours]

\paragraph{Topics}
\begin{itemize}
\item Review of Newtonian mechanics
\item \textbf{Lagrangian mechanics}
\item small oscillations
\item \textbf{Hamiltonian mechanics}
\item Rigid bodies
\item \textbf{Relativistic mechanics}
\item Fluid mechanics
\item Integrability and chaos
\end{itemize}


\section{Review of Newtonian mechanics}

Newtonian mechanics is an idealization.
\begin{enumerate}
\item Newtonian dynamics deals with \underline{point like} masses $m_1, \dots, m_N$. Their positions are given by vectors $\v{x}_i (t)$ in Euclidean space $\reals^3$ (as functions of time)
\item One defines a special class of coordinate systems, so-called inertial frames of reference, in which Newton's laws take the same form. 

The Galilean principle of relativity is that two inertial frames of references are related by 
	\begin{enumerate}
	\item Translation of the origin in $\reals^3 \times \reals$
	\begin{equation}
	\v{x} \rightarrow \v{x} + \v{x}_0, \qquad t \rightarrow t + t_0
	\end{equation}
	\item Rotation (time independent)
	\begin{equation}
	\v{x} \rightarrow R \v{x}, \qquad R \in SO(3)
	\end{equation}
	\item Galilean boost --- relates 2 frames of reference which move with relative constant velocity
	\begin{equation}
	\v{x}' = \v{x} + \v{v} t, \qquad t' = t.
	\end{equation}
	\end{enumerate}
	Note that these three operations form a group, the so called Galilean group (which happens to be the Poincare group from special relativity in the limit $c \rightarrow \infty$. Principle of relativity says that physical laws take the same form in all inertial frames.) Note also that this is a 10 dimensional group!

\item Newton's force law
\begin{itemize}
\item Point particles are acted upon by other point particles
\item The motion of point particles in the future is determined by the knowledge of the position and velocity of all particles at a fixed time $t$ (not acceleration $a$, or jerk, snap, crackle, or pop). 
\item Define the momentum $\v{p} = m\dot{\v{x}}$. (Note that the definition of $\v{p}$ holds in all frames, but needs to be modified in relativistc mechanics.) The theory is defined by the system of ordinary differential equations:
\begin{equation}
\dot{\v{p}}_i = \v{F}_i(\v{x}_j, \dot{\v{x}}_j, t).
\end{equation}
This is a system of $3N$ 2nd order differential equations. We also need $2 \times 3N$ initial conditions $\v{x}_i (t_0)$, $\dot{\v{x}}_i (t_0)$. 
\item Newton's 3rd law (action and reaction): If $m_1$ exerts a force $\v{F}_{21}$ on $m_2$ then $m_2$ exerts a force $\v{F}_{12} = -\v{F}_{21}$ on $m_1$. 
\item Newton's law of gravity: point masses $m_1$ and $m_2$ exert a gravitational force
\begin{equation}
\v{F}_{12} = - G_N m_1 m_2 \frac{\v{x}_1 - \v{x}_2}{\abs{\v{x}_1 - \v{x}_2}^3}
\end{equation}
where $G_N \approx \unit{6.673 \times 10^{-11}}{\meter^3 \per (\kilo\gram \cdot \second^3)}$ is the Newtonian gravitational constant. 
\end{itemize}

\end{enumerate}


\paragraph{Comments}
\begin{itemize}
\item Forces are instantaneous ($c \rightarrow \infty$) --- idealization not true for E\&M, relativity. 
\item In classical mechanics $\v{x}$ and $\v{p}$ can both be measured to arbitrary precision ($\hbar \rightarrow 0$) --- not true in QM.
\end{itemize}

\paragraph{Applications}
\begin{itemize}
\item Consequences of Newton's laws
\item Conservative forces
\item Important examples of physical systems
	\begin{enumerate}
	\item Harmonic oscillator
	\item motion in central potential
	\item particle in electromagnetic field
	\end{enumerate}
\end{itemize}


\section{Some simple consequences of Newton's laws}

(Should all be familiar from undergraduate.)


\subsection{Work and kinetic energy}
\begin{itemize}
\item For a single particle trajectory along path $C_{12}$ going from points $1 \rightarrow 2$ subject to force $\v{F}$, the work is given by
\begin{align}
W_{12} &= \int_{C_{12}} \v{F} \cdot d\v{x} \\
	&= \int_{C_{12}} m \dot{\v{v}} \cdot \frac{d\v{x}}{dt} dt \\
	&= \int_{C_{12}} m \dot{\v{v}} \cdot \v{v} dt \\
	&= \int_{C_{12}} \frac{d}{dt} \left( \frac{1}{2} m\v{v}^2 \right) dt \\
	&= T_2 - T_1
\end{align}
where $T = \frac{1}{2} m \v{v}^2$ is the kinetic energy. 
\item For $N$ particle system: same expression, just have
\begin{equation}
W_{12} = \sum_i \int \v{F}_i \cdot d\v{x}_i, \qquad T = \sum_i \frac{1}{2} m_i \v{v}_i^2
\end{equation}
\end{itemize}


\subsection{Conservative Forces \& energy conservation}

\begin{itemize}
\item For dissipative forces like friction $\v{F} = -k \v{v}$ the work $W_{12}$ depends on the path $C_{12}$. For a longer path $C_{12}'$ the work will be larger, $W_{12}' > W_{12}$.
\item For a \underline{conservative force} the work only depends on the initial and final points. So a conservative force should not explicitly depend on time. 
\end{itemize}

For the time being, let's assume that $\v{F}$ is independent of $\v{v}$. Take two distinct paths $C$ and $C' \neq C$ from point 1 to point 2. If $W_{12} = W_{12}'$ then $W_{12} - W_{12}' = 0$ so we have
\begin{equation}
\oint_C \v{F}(\v{x}) \cdot d\v{x} = 0
\end{equation}
where $C$ is the closed path $C - C'$. In other words, work along a closed path is zero. Using Stoke's theorem we have
\begin{equation}
\oint_C \v{F}(\v{x}) \cdot d\v{x} = \int_D (\v{\nabla} \times \v{F}) \cdot d^2 \v{S}
\end{equation}
for any surface $D$ with boundary $\pd D = C$, so that we have $\v{\nabla} \times \v{F} = 0$. Thus for an $\v{F}$ with a simply connected domain, we have 
\begin{equation}
\v{\nabla} \times \v{F} = 0 \quad \text{for all $\v{x}$} \quad \implies \quad \v{F} = - \v{\nabla} V.
\end{equation}
Furthermore, we find
\begin{equation}
W_{12} = \int_{C_{12}} \v{F} \cdot d\v{x} = - \int_{C_{12}} d\v{x} \cdot \v{\nabla} V = -(V_2 - V_1)
\end{equation}
thus we have finally
\begin{equation}
T_2 - T_1 = -(V_2 - V_1) \qquad \text{or} \qquad T_1 + V_1 = T_2 + V_2
\end{equation}
so the total energy $E = T + V$ is conserved. 

Note that we excluded velocity dependent forces (like friction) since they are generally non-conservative. Any important exception is the Lorentz force $\v{F} = e \v{v} \times \v{B}$. Since $\v{F} \perp \v{v} = d\v{x} / dt$, the work done by the Lorentz force is zero so the force is conservative. However it seems that the potential is zero --- one has to generalize to a velocity dependent potential, so we have
\begin{equation}
U = e \phi - e \v{v} \times \v{A} \quad \text{and} \quad \v{F} = -\v{\nabla}_{\v{x}} U - \frac{d}{dt} \v{\nabla}_{\dot{\v{x}}} U.
\end{equation}
(See D'Hoker for a more details discussion.) 


\section{System of Particles}

Consider a system of $N$ particles with mass $m_i$. We have Newton's equation for each particle, $m_i \ddot{\v{x}}_i = \v{F}_i$, where the forces $\v{F}_i = \v{F}_i (\v{x}_1, \dots, \v{x}_N)$ are functions of the positions (and possibly the velocities). We can again split the forces into interparticle and external forces,
\begin{equation}
\v{F}_i = \sum_{j \neq i} \v{F}_{ij} + \v{F}_i^\text{ext}.
\end{equation}
We define the center of mass
\begin{equation}
\v{X} = \frac{1}{M} \sum_i m_i \v{x}_i, \qquad M = \sum_i m_i.
\end{equation}
Then Newton's equation for the center of mass is given by
\begin{align}
M \ddot{\v{X}} &= \sum_i m_i \ddot{\v{x}}_i = \sum_i \v{F}_i = \sum_i \v{F}_i^\text{ext} + \sum_i \sum_{j \neq i} \v{F}_{ij} \\
	&= \sum_i \v{F}_i^\text{ext} + \frac{1}{2} \sum_i \sum_{j < i} (\v{F}_{ij} + \v{F}_{ji}) \\
	&= 0
\end{align}
where the last equality comes from Newton's 3rd law. 

For an isolated (or closed) system we have $\sum_i \v{F}_i^\text{ext} = 0$. Then the center of mass behaves like a free particle,
\begin{equation}
\v{X}(t) = \frac{\v{P}}{M} t + \v{X}(0), \qquad \v{P} = \sum_i \v{p}_i.
\end{equation}
This gives us 3 conserved quantities,
\begin{equation}
\v{X}(0) = \v{X}(t) - \frac{\v{P}}{M} t.
\end{equation}
Similarly we can show that the total momentum
\begin{equation}
\v{P}_\text{tot} = \sum_i \v{p}_i
\end{equation}
is conserved, giving us 3 more conserved quantities, and the total angular momentum
\begin{equation}
\v{L}_\text{tot} = \sum_i \v{x}_i \times \v{p}_i
\end{equation}
is conserved, giving yet 3 more conserved quantities. 


\subsection{Work and energy}

Notice that if we write $\v{\widetilde{x}}_i = \v{x}_i - \v{X}$, the total kinetic energy is given by
\begin{equation}
T = \sum_i \frac{1}{2} m \v{x}_i^2 = \frac{1}{2} \sum_i m_i \dot{\v{\widetilde{x}}}_i^2 + \frac{1}{2} M \dot{\v{X}}^2.
\end{equation}
The work energy theorem works as before,
\begin{equation}
T_2 - T_1 = \sum_i \int \v{F}_i \cdot d\v{x}_i.
\label{eq:work-energy}
\end{equation}
For conservative forces we have
\begin{equation}
\v{F}_i^\text{ext} = - \v{\nabla}_i V^\text{ext}.
\end{equation}
The contribution to the RHS of \eqref{eq:work-energy} from external forces is given by
\begin{equation}
\sum_i V_{i,(1)}^\text{ext} - \sum_i V_{i,(2)}^\text{ext}.
\end{equation}
Internal forces contribute to the RHS too. We have 
\begin{equation}
\v{F}_{ij} = -\v{\nabla}_i V_{ij}.
\end{equation}
If $V_{ij}$ is of the form $V_{ij} = V_{ij} (\abs{\v{x}_i - \v{x}_j})$ then we find
\begin{equation}
\v{F}_{ij} = - \frac{\v{x}_i - \v{x}_j}{\abs{\v{x}_i - \v{x}_j}} V'_{ij} (\abs{\v{x}_i - \v{x}_j})
\end{equation}
and hence $\v{F}_{ij} = -\v{F}_{ji}$, as expected from Newton's third law, and we also know that the force is in the direction of ${x}_i - \v{x}_j$. Then we find that the interparticle contribution to the RHS of \eqref{eq:work-energy} is given by
\begin{align}
\sum_i \sum_{j \neq i} \int \v{F}_i \cdot d\v{x}_i &= - \sum_i \sum_{j \neq i} \int \v{\nabla}_i V_{ij} \cdot d\v{x}_i \\
	&= -\frac{1}{2} \sum_i \sum_{j \neq i} \int \v{\nabla}_i V_{ij} \cdot (d\v{x}_i - d\v{x}_j) \\
	&= -\frac{1}{2} \sum_i \sum_{j \neq i} \int \v{\nabla}_i V_{ij} \cdot d\v{r}_{ij} \\
	&= - \sum_i \left( -\frac{1}{2} \sum_{j \neq i} V_{ij}^{(2)} - \frac{1}{2} \sum_{j \neq i} V_{ij}^{(1)} \right).
\end{align}
Thus combining the kinetic and potential terms for initial and final states we find the total energy
\begin{equation}
E = T + \sum_i V_i^\text{ext} + \frac{1}{2} \sum_i \sum_{j \neq i} V_{ij}
\end{equation}
is yet another conserved quantity. 

We have found a lot of conserved quantities for isolated systems:
\begin{itemize}
\item 3 for the total momentum $\v{P}_\text{tot}$, 
\item 3 for the total angular momentum $\v{L}_\text{tot}$,
\item 1 for the total energy $E$,
\item 3 for $\v{X} - (\v{P} / M) t$,
\end{itemize}
giving us a total of \textit{10 conserved quantities}. This is no accident: remember that the Galilean group has 10 dimensions --- this is the essence of Noether's theorem!






\end{document}
