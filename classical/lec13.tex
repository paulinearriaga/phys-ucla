% declare document class and geometry
\documentclass[12pt]{article} % use larger type; default would be 10pt
\usepackage[margin=1in]{geometry} % handle page geometry

% standard packages
\usepackage{graphicx} % support the \includegraphics command and options
\usepackage{amsmath} % for nice math commands and environments
\usepackage{mathtools} % extends amsmath with bug fixes and useful commands e.g. \shortintertext
\usepackage{amsthm} % for theorem and proof environments

% font packages
\usepackage{amssymb} % for \mathbb, \mathfrak fonts
\usepackage{mathrsfs} % for \mathscr font
\DeclareMathAlphabet{\mathpzc}{OT1}{pzc}{m}{it} % defines \mathpzc for Zapf Chancery (standard postscript) font

% other packages
\usepackage{datetime} % allows easy formatting of dates, e.g. \formatdate{dd}{mm}{yyyy}
\usepackage{caption} % makes figure captions better, more configurable
\usepackage{enumitem} % allows for custom labels on enumerated lists, e.g. \begin{enumerate}[label=\textbf{(\alph*)}]
\usepackage[squaren]{SIunits} % for nice units formatting e.g. \unit{50}{\kilo\gram}
\usepackage{cancel} % for crossing out terms with \cancel
\usepackage{verbatim} % for verbatim and comment environments
\usepackage{tensor} % for \indices e.g. M\indices{^a_b^{cd}_e}, and \tensor e.g. \tensor[^a_b^c_d]{M}{^a_b^c_d}
\usepackage{feynmp-auto} % for Feynman diagrams. 
\usepackage{pgfplots} % for plotting in tikzpicture environment
\usepackage{commath} % for some nice standardized syntax stuff. \dif, \Dif \od, \pd, \md, \(abs | envert), \(norm | enVert), \(set | cbr), \sbr, \eval, \int(o | c)(o | c), etc
\usepackage{slashed} % provides a command \slashed[1] for Feynman slash notation
%\newcommand{\fslash}[1]{#1\!\!\!/} % feynman slash
%\newcommand{\fsl}[1]{\ensuremath{\mathrlap{\!\not{\phantom{#1}}}#1}}% \fsl{<symbol>}
	% alternative feynman slash

% new commands
\newcommand{\beg}{\begin} % a few letters less for beginning environments
\newenvironment{eqn}{\begin{equation}}{\end{equation}} % a lot fewer letter for equation environment

% rotate stuff
\usepackage{rotating}
	% provides environments for rotating arbitrary objects, e.g. sideways, turn[ang], rotate[ang]
	% also provides macro \turnbox{ang}{stuff}
%\newcommand{\sideways}[1]{\begin{sideways} #1 \end{sideways}} % turn things 90 degrees CCW
%\newcommand{\turn}[2][]{\begin{turn}{#2} #1 \end{turn}} % \turn[ang]{stuff} turns things arbitrary +/- angle

% notational commands
\newcommand{\opname}[1]{\operatorname{#1}} % custom operator names
%\newcommand{\pd}{\partial} % partial differential shortcut
\newcommand{\ket}[1]{\left| #1 \right>} % for Dirac kets
%\newcommand{\ket}[1]{| #1 \rangle}
\newcommand{\bra}[1]{\left< #1 \right|} % for Dirac bras
%\newcommand{\bra}[1]{\langle #1 |}
\newcommand{\braket}[2]{\left< #1 \vphantom{#2} \right| \left. #2 \vphantom{#1} \right>} 
	% for Dirac bra-kets \braket{bra}{ket}
%\newcommand{\braket}[2]{\langle #1 | #2 \rangle} 
\newcommand{\matrixel}[3]{\left< #1 \vphantom{#2#3} \right| #2 \left| #3 \vphantom{#1#2} \right>} 
	% for Dirac matrix elements \matrixel{bra}{op}{ket}
%\newcommand{\matrixel}[3]{\langle #1 | #2 | #3 \rangle} 

%\newcommand{\pd}[2]{\frac{\partial #1}{\partial #2}} % for partial derivatives
%\newcommand{\fd}[2]{\frac{\delta #1}{\delta #2}} % for functional derivatives
\let \vaccent = \v % rename builtin command \v{} to \vaccent{}
%\renewcommand{\v}[1]{\ensuremath{\mathbf{#1}}} % for vectors
\renewcommand{\v}[1]{\ensuremath{\boldsymbol{\mathbf{#1}}}} % for vectors
%\newcommand{\gv}[1]{\ensurmath{\mbox{\boldmath$ #1 $}}} % for vectors of Greek letters
\newcommand{\uv}[1]{\ensuremath{\boldsymbol{\mathbf{\widehat{#1}}}}} % for unit vectors
%\newcommand{\abs}[1]{\left| #1 \right|} % for absolute value ||x||
%\newcommand{\mag}{\abs} % magnitude, just another name for \abs
%\newcommand{\norm}[1]{\left\Vert #1 \right\Vert} % for norm ||v||
\newcommand{\vd}[1]{\v{\dot{#1}}} % for dotted vectors
\newcommand{\vdd}[1]{\v{\ddot{#1}}} % for ddotted vectors
\newcommand{\vddd}[1]{\v{\dddot{#1}}} % for dddotted vectors
\newcommand{\vdddd}[1]{\v{\ddddot{#1}}} % for ddddotted vectors
\newcommand{\avg}[1]{\left< #1 \right>} % for average <x>
\newcommand{\inner}[2]{\left< #1, #2 \right>} % for inner product <x,y>
%\newcommand{\set}[1]{ \left\{ #1 \right\} } % for sets {a,b,c,...}
\newcommand{\tr}{\opname{tr}} % for trace
\newcommand{\Tr}{\opname{Tr}} % for Trace
\newcommand{\rank}{\opname{rank}} % for rank
\let \fancyre = \Re
\let \fancyim = \Im
\newcommand{\Res}{\opname{Res}\limits} % for residue function -- change to put limits on bottom
\renewcommand{\Re}{\opname{Re}}
\renewcommand{\Im}{\opname{Im}}
\renewcommand{\bbar}[1]{\bar{\bar{#1}}} 
	% for barring things twice -- use \cbar or \zbar instead of original \bbar
\newcommand{\bbbar}[1]{\bar{\bbar{#1}}}
\newcommand{\bbbbar}[1]{\bar{\bbbar{#1}}}

\newcommand{\inv}{^{-1}}

% temporary fixes -- commath's versions are bad for powers, like $\dif^3 x$
\renewcommand{\dif}{\mathrm{d}} % \opname{d} better maybe?
\renewcommand{\Dif}{\mathrm{D}}

% notational shortcuts
\newcommand{\bigO}{\mathcal{O}} % big O notation
\let \bigo = \bigO % keep for now, need to update instances in older files
\newcommand{\Lag}{\mathcal{L}} % fancy Lagrangian
\newcommand{\Ham}{\mathcal{H}} % fancy Hamiltonian
\newcommand{\reals}{\mathbb{R}} % real numbers
\newcommand{\complexes}{\mathbb{C}} % complex numbers
\newcommand{\ints}{\mathbb{Z}} % integers
\newcommand{\nats}{\mathbb{N}} % natural numbers
\newcommand{\irrats}{\mathbb{Q}} % irrationals
\newcommand{\quats}{\mathbb{H}} % quaternions (a la Hamilton)
\newcommand{\euclids}{\mathbb{E}} % Euclidean space
\newcommand{\R}{\reals}
\newcommand{\C}{\complexes}
\newcommand{\Z}{\ints}
\newcommand{\Q}{\irrats}
\newcommand{\N}{\nats}
\newcommand{\E}{\euclids}
\newcommand{\RP}{\mathbb{RP}} % real projective space
\newcommand{\CP}{\mathbb{CP}} % complex projective space

% matrix shortcuts!
\newcommand{\pmat}[1]{\begin{pmatrix} #1 \end{pmatrix}}
\newcommand{\bmat}[1]{\begin{bmatrix} #1 \end{bmatrix}}
\newcommand{\Bmat}[1]{\begin{Bmatrix} #1 \end{Bmatrix}}
\newcommand{\vmat}[1]{\begin{vmatrix} #1 \end{vmatrix}}
\newcommand{\Vmat}[1]{\begin{Vmatrix} #1 \end{Vmatrix}}


% more stuff
\newenvironment{enumproblem}{\begin{enumerate}[label=\textbf{(\alph*)}]}{\end{enumerate}}
	% for easily enumerating letters in problems
\newcommand{\grad}[1]{\v{\nabla} #1} % for gradient
\let \divsymb = \div % rename builtin command \div to \divsymb
\renewcommand{\div}[1]{\v{\nabla} \cdot #1} % for divergence
\newcommand{\curl}[1]{\v{\nabla} \times #1} % for curl
\let \baraccent = \= % rename builtin command \= to \baraccent
\renewcommand{\=}[1]{\stackrel{#1}{=}} % for putting numbers above =


% theorem-style environments. note amsthm builtin proof environment: \begin{proof}[title]
% appending [section] resets counter and prepends section number
% use \setcounter{counter}{0} to reset counter
% typical use cases:
% plain: Theorem, Lemma, Corollary, Proposition, Conjecture, Criterion, Algorithm
% definition: Definition, Condition, Problem, Example
% remark: Remark, Note, Notation, Claim, Summary, Acknowledgment, Case, Conclusion
\theoremstyle{plain} % default
\newtheorem{theorem}{Theorem}[section]
\newtheorem{lemma}[theorem]{Lemma}
\newtheorem{corollary}[theorem]{Corollary}
\newtheorem{proposition}[theorem]{Proposition}
\newtheorem{conjecture}[theorem]{Conjecture}
% definition style
\theoremstyle{definition}
\newtheorem{definition}{Definition}
\newtheorem{problem}{Problem}
\newtheorem{exercise}{Exercise}
\newtheorem{example}{Example}
% remark style
\theoremstyle{remark}
\newtheorem{remark}{Remark}
\newtheorem{note}{Note}
\newtheorem{claim}{Claim}
\newtheorem{conclusion}{Conclusion}
% to-do: add problem/subproblem/answer environments for homeworks









%%%%% derivatives


\let \underdot = \d % rename builtin command \d{} to \underdot{}
\let \d = \od % for derivatives

% BUG: derivatives revert to text mode often when in smaller environments in math mode?


% Command for functional derivatives. The first argument denotes the function and the second argument denotes the variable with respect to which the derivative is taken. The optional argument denotes the order of differentiation. The style (text style/display style) is determined automatically
\providecommand{\fd}[3][]{\ensuremath{
\ifinner
\tfrac{\delta{^{#1}}#2}{\delta{#3^{#1}}}
\else
\dfrac{\delta{^{#1}}#2}{\delta{#3^{#1}}}
\fi
}}

% \tfd[2]{f}{k} denotes the second functional derivative of f with respect to k
% The first letter t means "text style"
\providecommand{\tfd}[3][]{\ensuremath{\mathinner{
\tfrac{\delta{^{#1}}#2}{\delta{#3^{#1}}}
}}}
% \dfd[2]{f}{k} denotes the second functional derivative of f with respect to k
% The first letter d means "display style"
\providecommand{\dfd}[3][]{\ensuremath{\mathinner{
\dfrac{\delta{^{#1}}#2}{\delta{#3^{#1}}}
}}}

% mixed functional derivative - analogous to the functional derivative command
% \mfd{F}{5}{x}{2}{y}{3}
\providecommand{\mfd}[6]{\ensuremath{
\ifinner
\tfrac{\delta{^{#2}}#1}{\delta{#3^{#4}}\delta{#5^{#6}}}
\else
\dfrac{\delta{^{#2}}#1}{\delta{#3^{#4}}\delta{#5^{#6}}}
\fi
}}


% Command for thermodynamic (chemistry?) partial derivatives. The first argument denotes the function and the second argument denotes the variable with respect to which the derivative is taken. The optional argument denotes the order of differentiation. The style (text style/display style) is determined automatically
\providecommand{\pdc}[4][]{\ensuremath{
\ifinner
\left( \tfrac{\partial{^{#1}}#2}{\partial{#3^{#1}}} \right)_{#4}
\else
\left( \dfrac{\partial{^{#1}}#2}{\partial{#3^{#1}}} \right)_{#4}
\fi
}}

% \tpd[2]{f}{k} denotes the second thermo partial derivative of f with respect to k
% The first letter t means "text style"
\providecommand{\tpdc}[4][]{\ensuremath{\mathinner{
\left( \tfrac{\partial{^{#1}}#2}{\partial{#3^{#1}}} \right)_{#4}
}}}
% \dpd[2]{f}{k} denotes the second thermo partial derivative of f with respect to k
% The first letter d means "display style"
\providecommand{\dpdc}[4][]{\ensuremath{\mathinner{
\left( \dfrac{\partial{^{#1}}#2}{\partial{#3^{#1}}} \right)_{#4}
}}}


%%%%%%





%%%%%%%%%%%%%%%%%%%
% some templates for various things
\begin{comment}

% template for figures
\begin{figure}
\centering
\includegraphics{myfile.png}
\caption{This is a caption}
\label{fig:myfigure}
\end{figure}

% template for Feynman diagrams using feynmf/feynmp
\begin{fmfgraph*}(40,25)
\fmfleft{em,ep}
\fmf{fermion}{em,Zee,ep}
\fmf{photon,label=$Z$}{Zee,Zff}
\fmf{fermion}{fb,Zff,f}
\fmfright{fb,f}
\fmfdot{Zee,Zff}
\end{fmfgraph*}

% template for drawing plots with pgfplot
\pgfplotsset{compat=1.3,compat/path replacement=1.5.1}
\begin{tikzpicture}
\begin{axis}[
extra x ticks={-2,2},
extra y ticks={-2,2},
extra tick style={grid=major}]
\addplot {x};
\draw (axis cs:0,0) circle[radius=2];
\end{axis}
\end{tikzpicture}

%% find package for easily drawing mapping / algebraic / commutative diagrams..

\end{comment}
%%%%%%%%%%%%%%%%%%%



%%%%% A note on spacing
% 5) \qquad
% 4) \quad
% 3) \thickspace = \;
% 2) \medspace = \:
% 1) \thinspace = \,
% -1) \negthinspace = \!
% -2) \negmedspace
% -3) \negthickspace




\title{Phys 220A -- Classical Mechanics -- Lec13}
\author{UCLA, Fall 2014}
\date{\formatdate{20}{11}{2014}} % Activate to display a given date or no date (if empty),
         % otherwise the current date is printed 

\begin{document}
\setlength{\unitlength}{1mm}
\maketitle


\section{Rigid bodies}

Rigid bodies are great because they give us interesting differential equations and provide a good example of when coordinate transformations are useful. What is a rigid body? It's a set of $3N$ degrees of freedom $\v x^1, \dots, \v x^N$ ($N$ mass points) where all the distances are constant, i.e.
\begin{eqn}
\abs{\v x^i - \v x^j} = \text{const}
\end{eqn}
for all $i,j$. We can think of rigid bodies from a different perspective by considering translations of the center of mass $\v X$,
\begin{eqn}
\v x^i \rightarrow \v x^i + \v X,
\end{eqn}
which gives us three degrees of freedom, and rotations of the body
\begin{eqn}
\v x^i \rightarrow R \cdot \v x^i, \qquad
R \in SO(3),
\end{eqn}
which give us another 3 degrees of freedom. So the $3N$ degrees of freedom reduce down to 6. 

If we look at the rotations in terms of the Lie algebra, we have
\begin{eqn}
R = 1 + \epsilon T
\end{eqn}
where $T$ is a generator of rotations. We find that
\begin{eqn}
R^\top R =1 \qquad
\implies \qquad
T^\top + T = 0,
\end{eqn}
so the generators $T$ are $3 \times 3$ antisymmetric matrices,
\begin{eqn}
T = 
\begin{pmatrix}
0 & a & b \\
-a & 0 & c \\
-b & -c & 0
\end{pmatrix}.
%\begin{pmatrix}
%0 & \omega_3 & -\omega_2 \\
%-\omega_3 & 0 & \omega_1 \\
%\omega_2 & -\omega_1 & 0
%\end{pmatrix}.
\end{eqn}
For $b=c=0$ we get rotations about the $z$-axis, similarly about the $x$-axis for $a=b=0$ and about the $y$-axis for $a=c=0$. So we can rewrite the generators as
\begin{eqn}
T_{ij} = \sum_k \epsilon_{ijk} \omega_k
\end{eqn}
which gives us rotation about the vector $\v \omega_k$. 

Generally it will be useful to change from the lab frame $\Gamma$ into the body frame $\Gamma'$. The relation between points $\v x$ and $\v x'$ in these frames can be written
\begin{eqn}
x_i(t) = X_i(t) + R_{ij}(t) x'_j(t),
\label{eq:changeFrame}
\end{eqn}
where $\v x'$ is a constant vector and we need to find the rotation $R_{ij}(t)$. 


\subsection{Euler Angles}

It turns out that for an arbitrary rotation $R \in SO(3)$ we can write 
\begin{eqn}
R = R_3(\psi) R_1(\theta) R_3(\phi)
\end{eqn}
where $R_1$ and $R_3$ are rotations
\begin{eqn}
R_3(\phi) = 
\begin{pmatrix}
\cos \phi & \sin\phi & 0 \\
-\sin\phi & \cos\phi & 0 \\
0 & 0 & 1,
\end{pmatrix}, \qquad
R_1(\theta) = 
\begin{pmatrix}
1 & 0 & 0 \\
0 & \cos\theta & \sin\theta \\
0 & -\sin\theta & \cos\theta
\end{pmatrix}.
\end{eqn}
Note that we perform these rotations in right-left order, i.e. first perform the rotation $R_3(\phi)$, next the rotation $R_1(\theta)$, and finally the rotation $R_3(\psi)$. 

\subsection{Velocity}

If we take the time derivative of our coordinate transformation \eqref{eq:changeFrame}, we obtain
\begin{eqn}
\dot x_i = \dot X_i + \dot R_{ij} x'_j.
\label{eq:velocity0}
\end{eqn}
Now consider for a moment the derivative of $R^\top R = 1$,
\begin{eqn}
\dot R^\top R + R^\top \dot R = 0,
\end{eqn}
which if we write $M = R^\top \dot R$ is just
\begin{eqn}
M^\top + M = 0.
\end{eqn}
Then we can write $M$ as
\begin{eqn}
M_{ij} = (R^\top \dot R)_{ij} = \sum_k \epsilon_{ijk} \omega_k
\end{eqn}
where $\v \omega$ is the angular velocity, so we can rewrite the velocity equation \eqref{eq:velocity0} as 
\begin{eqn}
\vd x = \vd X + R \cdot (\v \omega \times \v x'),
\label{eq:velocity1}
\end{eqn}
which is our major result. One more thing to note is that if we shift coordinates in the body frame $\v x' \rightarrow \v x' + \v a$, our velocity changes by
\begin{eqn}
\vd x \rightarrow \vd x + R \cdot (\v \omega \times \v a).
\end{eqn}

Now we can work out the kinetic energy with our new velocity equation \eqref{eq:velocity1},
\begin{align}
T &= \frac{1}{2} \sum_i m_i \vd x^{i\top} \vd x^i \\
	&= \frac{1}{2} \sum_i m_i \vd X^\top \vd X 
		\; + \; \vd X^\top R \cdot (\v \omega \times \cancelto{0}{\sum_i m_i  \v x'^i}) \\
		&\qquad + \; \frac{1}{2} \sum_i m_i (\v \omega \times \v x'^i)^\top \underbrace{R^\top R}_{=1} \cdot (\v \omega \times \v x'^i),
\end{align}
where the second term cancels because it is just the center of mass in the body frame, which is zero. So we can rewrite the remaining two terms as
\begin{eqn}
T = T_\mathrm{CM} + \frac{1}{2} \sum_{ij} \omega_i I_{ij} \omega_j,
\end{eqn}
where $I_{ij}$ is the inertia tensor
\begin{eqn}
I_{ij} = \sum_k m_k \left[ \delta_{ij} (\v x'^k)^2 - x'^k_i x'^k_j \right]. 
\end{eqn}
We've been working with discrete point masses, what if we more generally have a mass density $\rho(\v x)$? Then the sum over masses becomes an integral over the mass density. Explicitly, the matrix takes the form
\begin{eqn}
I = \int \dif^3{x} \, \rho(\v x) 
\begin{pmatrix}
y^ + z^2 & -xy & -xz \\
-xy & x^2 + z^2 & -yz \\
-xz & -yz & x^2 + y^2
\end{pmatrix}.
\end{eqn}

We can immediately notice some nice properties of $I_{ij}$:
\begin{itemize}
\item it's positive definite, 
\item it's symmetric, and thus
\item it can be diagonalized, i.e. we can rotate so that $I = \diag\set{I_1, I_2, I_3}$.
\end{itemize}
The inertia tensor's eigenvectors define a set of ``principal axes'' determined by the eigenvalues $I_k$. We can summarize the symmetry of an object by
\begin{itemize}
\item $I_1 = I_2 = I_3$ --- completely spherically symmetric object, e.g. point mass;
\item $I_1 = I_2 \neq I_3$ --- radially symmetric, e.g. symmetric top;
\item $I_1 \neq I_2 \neq I_3$ --- asymmetric, e.g. asymmetric top.
\end{itemize}
Finally we have the following theorem.
\begin{theorem}[Parallel Axis theorem]
Suppose we shift the axis from the center of mass to the point $\v a$, then the inertia tensor shifts by
\begin{eqn}
I_{ij}' = I_{ij}^\mathrm{CM} + M_\mathrm{tot} (\v a^2 \delta_{ij} - a_i a_j).
\end{eqn}
\end{theorem}

How can we write the angular momentum in this notation? We have
\begin{align}
\v L_\mathrm{tot} &= \sum_k m_k \v x x^k \times \vd x^k \\
	&= \sum_k m_k (\v X + R \v x'^k) \times (\vd X + R \cdot (\omega \times \v x'^k)) \\
	&= \underbrace{\sum_k m_k \v X \times \vd X}_{\v L_\mathrm{CM}} \; + \; \cancelto{0}{\sum_k m_k \v x'^k} \\
		&\qquad\quad + \; \sum_k m_k (R \v x'^k) \times R (\v \omega \times \v x'^k).
\end{align}
Using the identity
\begin{eqn}
\sum_{ijk} \epsilon_{ijk} R_{ii'} R_{jj'} R_{kk'} = \epsilon_{i'j'k'},
\end{eqn}
one can show that this results in 
\begin{eqn}
\v L_\mathrm{tot} = \v L_\mathrm{CM} + R \cdot (I \cdot \v \omega).
\end{eqn}
This informs us that the angular momentum in the body frame is $\v L' = I \cdot \v \omega$. 


\subsection{Free rigid body}

Suppose there are no external forces and therefore no torques. Then we have $\vd L_\mathrm{inertial} = 0$ so that
\begin{eqn}
0 = \od{}{t} (R I \v \omega) = \dot R I \v \omega + R I \vd \omega.
\end{eqn}
If we multiply the first term on the left by $R R^\top = 1$ we find
\begin{eqn}
R \underbrace{(\v \omega \times (I \v \omega) + I \vd \omega)}_{=0} = 0.
\end{eqn}
Explicitly, working in the diagonalized frame where our axes are the principal axes, we obtain \textit{Euler's equations},
\begin{align}
0 &= I_1 \dot \omega_1 + \omega_2 \omega_3 (I_3 - I_2), \\
0 &= I_2 \dot \omega_2 + \omega_3 \omega_1 (I_1 - I_3), \\
0 &= I_3 \dot \omega_3 + \omega_1 \omega_2 (I_2 - I_1).
\end{align}
If we further assume that $\v \omega$ is constant and we have a symmetric top, $I_1 = I_2$, we find 
\begin{eqn}
\dot \omega_1 = \omega_2 \Omega, \qquad
\dot \omega_2 = - \omega_1 \Omega,
\end{eqn}
where
\begin{eqn}
\Omega = \omega_3 \, \frac{I_1 - I_3}{I_1}.
\end{eqn}
The general case is very difficult and must be solved using elliptic functions. 







\end{document}
