% declare document class and geometry
\documentclass[12pt]{article} % use larger type; default would be 10pt
\usepackage[margin=1in]{geometry} % handle page geometry

% import packages and commands
% standard packages
\usepackage{graphicx} % support the \includegraphics command and options
\usepackage{amsmath} % for nice math commands and environments

% font packages
\usepackage{amssymb} % for \mathbb, \mathfrak fonts
\usepackage{mathrsfs} % for \mathscr font
\DeclareMathAlphabet{\mathpzc}{OT1}{pzc}{m}{it} % defines \mathpzc for Zapf Chancery (standard postscript) font

% other packages
\usepackage{datetime} % allows easy formatting of dates, e.g. \formatdate{dd}{mm}{yyyy}
\usepackage{caption} % makes figure captions better, more configurable
\usepackage{enumitem} % allows for custom labels on enumerated lists, e.g. \begin{enumerate}[label=\textbf{(\alph*)}]
\usepackage[squaren]{SIunits} % for nice units formatting e.g. \unit{50}{\kilo\gram}
\usepackage{cancel} % for crossing out terms with \cancel
\usepackage{verbatim} % for verbatim and comment environments
\usepackage{tensor} % for \indices e.g. M\indices{^a_b^{cd}_e}, and \tensor e.g. \tensor[^a_b^c_d]{M}{^a_b^c_d}
\usepackage{feynmp-auto} % for Feynman diagrams. 
\usepackage{pgfplots} % for plotting in tikzpicture environment

% new commands
\newcommand{\beg}{\begin} % a few letters less for beginning environments
\newenvironment{eqn}{\begin{equation}}{\end{equation}} % a lot fewer letter for equation environment

% notational commands
\newcommand{\opname}[1]{\operatorname{#1}} % custom operator names
\newcommand{\fslash}[1]{#1\!\!\!/} % feynman slash
\newcommand{\pd}{\partial} % partial differential shortcut
\newcommand{\ket}[1]{\left| #1 \right>} % for Dirac kets
\newcommand{\bra}[1]{\left< #1 \right|} % for Dirac bras
\newcommand{\braket}[2]{\left< #1 \vphantom{#2} \right| 
	\left. #2 \vphantom{#1} \right>} % for Dirac brackets
%\let\underdot=\d % rename builtin command \d{} to \underdot{}
%\renewcommand{\d}[2]{\frac{d #1}{d #2}} % for derivatives
%\newcommand{\pd}[2]{\frac{\partial #1}{\partial #2}} % for partial derivatives
%\newcommand{\fd}[2]{\frac{\delta #1}{\delta #2}} % for functional derivatives
\let\vaccent=\v % rename builtin command \v{} to \vaccent{}
%\renewcommand{\v}[1]{\ensuremath{\mathbf{#1}}} % for vectors
\renewcommand{\v}[1]{\ensuremath{\boldsymbol{\mathbf{#1}}}} % for vectors
%\newcommand{\gv}[1]{\ensurmath{\mbox{\boldmath$ #1 $}}} % for vectors of Greek letters
\newcommand{\uv}[1]{\ensuremath{\boldsymbol{\mathbf{\widehat{#1}}}}} % for unit vectors
\newcommand{\abs}[1]{\left| #1 \right|} % for absolute value ||x||
%\newcommand{\mag}{\abs} % magnitude, just another name for \abs
\newcommand{\norm}[1]{\left\Vert #1 \right\Vert} % for norm ||v||
\newcommand{\avg}[1]{\left< #1 \right>} % for average <x>
\newcommand{\inner}[2]{\left< #1, #2 \right>} % for inner product <x,y>
\newcommand{\set}[1]{ \left\{ #1 \right\} } % for sets {a,b,c,...}
\newcommand{\tr}{\opname{tr}} % for trace
\newcommand{\Tr}{\opname{Tr}} % for Trace

% notational shortcuts
\newcommand{\reals}{\mathbb{R}} % real numbers
\newcommand{\complexes}{\mathbb{C}} % complex numbers
\newcommand{\nats}{\mathbb{N}} % natural numbers
\newcommand{\irrats}{\mathbb{Q}} % irrationals
\newcommand{\quats}{\mathbb{H}} % quaternions (a la Hamilton)
\newcommand{\euclids}{\mathbb{E}} % Euclidean space
\newcommand{\bigo}{\mathcal{O}} % big O notation
\newcommand{\Lag}{\mathcal{L}} % fancy Lagrangian
\newcommand{\Ham}{\mathcal{H}} % fancy Hamiltonian





%%%%%%%%%%%%%%%%%%%
% some templates for various things
\begin{comment}

% template for figures
\begin{figure}
\centering
\includegraphics{myfile.png}
\caption{This is a caption}
\label{fig:myfigure}
\end{figure}

% template for Feynman diagrams using feynmf/feynmp
\begin{fmfgraph*}(40,25)
\fmfleft{em,ep}
\fmf{fermion}{em,Zee,ep}
\fmf{photon,label=$Z$}{Zee,Zff}
\fmf{fermion}{fb,Zff,f}
\fmfright{fb,f}
\fmfdot{Zee,Zff}
\end{fmfgraph*}

% template for drawing plots with pgfplot
\pgfplotsset{compat=1.3,compat/path replacement=1.5.1}
\begin{tikzpicture}
\begin{axis}[
extra x ticks={-2,2},
extra y ticks={-2,2},
extra tick style={grid=major}]
\addplot {x};
\draw (axis cs:0,0) circle[radius=2];
\end{axis}
\end{tikzpicture}

\end{comment}
%%%%%%%%%%%%%%%%%%%



\title{Math 217 -- Geometry and Physics -- Lec01}
\author{UCLA, Fall 2014}
\date{\formatdate{03}{10}{2014}} % Activate to display a given date or no date (if empty),
         % otherwise the current date is printed 

\begin{document}
\maketitle


\section{Introduction}


\subsection{Books for reference}

\begin{enumerate}
\item T. Frankel -- The Geometry of Physics
\item Bott, Tu -- Differential Forms in Algebraic Topology
\item Lawson, Michelson -- Spin Geometry
\item J. Roe -- Elliptic Operators, Topology, and Asymptotic Methods
\item Berline, Getzler, Vergne -- Heat Kernel \& Dirac operators
\end{enumerate}


\subsection{History}

\begin{align}
\int_M \text{geometry} &\neq \text{Topological} \\
\varepsilon(TM) &\qquad \chi(M) \notag
\end{align}

\begin{enumerate}
\item Chern, 1946: Gauss-Bonnet theorem.
\item Hodge Theory (Analysis of elliptic PDEs)
\item Hirzebruch, 1950: Riemann-Roch Signature (Algebraic geometry, topology)
\item Grothendieck (GRR), 1958-59: K-theory
\item Atiyah-Hirzebruch: topological K-theory, topological Riemann-Roch
\item Atiyah-Singer: index formula, Dirac operator ($\hat{A}(M) = \operatorname{Ind} D$)
\item Mckean-Singer formula: heat-Kernel of elliptic operators
\item Atiyah-Boti-Patodi, 1978
\item Witten, Alveraz, Gaume, 1980: Heat kernel proof (Getzler), QFT (adiabatic limit)
\item Applications
\end{enumerate}


\section{Manifolds}

\subsection{Basics}

Denote the \textbf{manifold} $M = \cup_{i \in I} U_i$ with \textbf{coordinate maps} $\varphi_i: U_i \rightarrow \mathbb{R}^n$ which are homeomorphisms / coordinate covers. Recall that a manifold is a topological space (Hausdorff) and locally Euclidean. 
\begin{equation}
\text{(insert canonical diagram of coordinate maps)}%. see e.g. https://commons.wikimedia.org/wiki/File:Two\_coordinate\_charts\_on\_a\_manifold.svg)}
\end{equation}
If all the $\varphi_j \circ \varphi_i^{-1} \in C^\infty$, then it is a \textbf{smooth manifold}. 


\subsection{Examples}

\begin{enumerate}
\item The circle $M = S^1 = \set{ (x,y)\in \mathbb{R}^2 : x^2 + y^2 = 1 }$
\begin{equation}
\text{(insert diagram of $S^1$ with four charts)}
\end{equation}
with charts
\begin{align}
U_1 &= \set{ x>0 } \overset{\varphi_1}{\longrightarrow} \mathbb{R}^1, \qquad \varphi_1(p) = y, \\
U_2 &= \set{ y>0 } \overset{\varphi_2}{\longrightarrow} \mathbb{R}^1, \qquad \varphi_2(p) = x, \\
U_3 &= \set{ x<0 } \overset{\varphi_3}{\longrightarrow} \mathbb{R}^1, \qquad \varphi_3(p) = y, \\
U_4 &= \set{ y<0 } \overset{\varphi_4}{\longrightarrow} \mathbb{R}^1, \qquad \varphi_4(p) = x.
\end{align}
On $U_1 \cap U_2$, we have the coordinate change $y = \sqrt{1-x^2}$, $x = \sqrt{1-y^2}$.

\item The 2-sphere $S^2 = \set{ (x,y,z) \in \mathbb{R}^3 : x^2 + y^2 + z^2 = 1 }$
\begin{equation}
\text{(insert diagram of $S^2$)}
\end{equation}
with charts
\begin{align}
U_+ &= \set{ z \neq -1 } \overset{\varphi_+}{\longrightarrow} \mathbb{R}^2 \\
U_- &= \set{ z \neq 1 } \overset{\varphi_-}{\longrightarrow} \mathbb{R}^2
\end{align}
\begin{align}
\varphi_+(x,y,z) &= (\frac{x}{1+z}, \frac{y}{1+z}) \\
\varphi_-(x,y,z) &= (\frac{x}{1-z}, \frac{y}{1-z}).
\end{align}
The inverse of the coordinate maps can be written
\begin{align}
\varphi_+^{-1} : \mathbb{R}^2 &\longrightarrow U_+ \\
	(x_1, x_2) &\mapsto (\frac{2x_1}{1+\rho^2}, \frac{2x_2}{1+\rho^2}, \frac{1-\rho^2}{1+\rho^2}) 
\end{align}
\begin{align}
\varphi_-^{-1} : \mathbb{R}^2 &\longrightarrow U_- \\
	(x_1, x_2) &\mapsto (\frac{2x_1}{1+\rho^2}, \frac{2x_2}{1+\rho^2}, \frac{\rho^2-1}{1+\rho^2}) 
\end{align}
where $\rho^2 = x^2 + y^2$. Then
\begin{align}
\varphi_- \circ \varphi_+^{-1} : \varphi_+(U_+ \cap U_-) &\rightarrow \varphi_-(U_+ \cap U_p) \\
	(x_1, x_2) &\mapsto (x_1 / \rho, x_2 / \rho)
\end{align}


\item The real projective space $\mathbb{RP}^n = (\mathbb{R}^{n+1} - \set{ 0 }) / \mathbb{R}^* = S^n / \mathbb{Z}_2$, i.e. $(x_0, \dots, x_n) \sim (\lambda x_0, \dots, \lambda x_n)$, $\lambda \in \mathbb{R}^*$

\item The complex projective space $\mathbb{CP}^n = (\mathbb{C}^{n+1} - \set{ 0 }) / \mathbb{C}^* = S^{2n+1} / S^1$, i.e. $z_0, \dots, z_n) \sim (\lambda z_0, \dots, \lambda z_n)$, $\lambda \in \mathbb{C}^*$, say with charts: $U_i = \set{ z_i \neq 0 }$,
\begin{align}
\varphi_i : U_i &\longrightarrow \mathbb{C}^n \\
	(z_0, \dots, z_n) &\mapsto (z_0/z_i, \dots, z_n/z_i)
\end{align}
For example, $\mathbb{CP}^1 \cong S^2$.

%(more stuff on $\mathbb{CP}^n$ missing here? (got more from Yeou))


\subsection{More on manifolds}

Denote a \textbf{tangent bundle} $TM = \cup_{x \in M} T_x M$ with \textbf{canonical map} $\pi: TM \rightarrow M$. $TM$ has a manifold structure.

$f: M \rightarrow N$ is a \textbf{smooth map} if $f \circ \varphi_i^{-1}$ is smooth for all $i$. If $f$ is smooth, then define the differential or pushforward $df = f_* : TM \rightarrow TN$ such that for all $x \in M$, 
\begin{align}
f_*(x) : T_x M &\longrightarrow T_{f(x)}M \\
	v &\mapsto f_*(v) = \frac{d}{dt} f(\alpha(t)) \Big|_{t=0}
\end{align}
where $\alpha : (-\epsilon, \epsilon) \rightarrow M$, $\alpha(0) = x$, and $\alpha'(0) = v$. 

We define a \textbf{smooth section} as a map $X : M \rightarrow TM$ such that $\pi \circ X = \operatorname{id}_M$. For example, a vector field on $M$ is a standard example of a smooth section. 

%(missed board on lie brackets here (got more from Yeou))

Using coordinates on chart $U$, $X = \sum_{i=1}^n a_i \frac{\pd}{\pd x_i}$ for all $f \in C^\infty(M)$ such that $f : M \rightarrow \mathbb{R}$. Then $\mathcal{L}_X f = X(f) = f_*(X)$.

Lie bracket: $[X,Y]f = X(Yf) - Y(Xf)$. Antisymmetric, Jacobi (???)


Given a vector field $X$, let $\varphi : (a_x, b_x) \rightarrow M$ be a curve s.t. $\varphi(0) = x$, $\pd \varphi / \pd t = X \circ \varphi$ on $(a_x, b_x)$ maximal interval

If $(a_x, b_x) = (-\infty, \infty) = \mathbb{R}$, $\varphi$ or $X$ is complete

%(more on a board here on vector field? (got more from Yeou))

\end{enumerate}



\begin{comment}
\begin{figure}
\centering
\includegraphics{3a.pdf}
\caption{Half the diagrams for photon-photon scattering.}
\label{fig:3a}
\end{figure}
\end{comment}



\end{document}
