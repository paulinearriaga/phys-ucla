% declare document class and geometry
\documentclass[12pt]{article} % use larger type; default would be 10pt
\usepackage[margin=1in]{geometry} % handle page geometry

% import packages and commands
% standard packages
\usepackage{graphicx} % support the \includegraphics command and options
\usepackage{amsmath} % for nice math commands and environments

% font packages
\usepackage{amssymb} % for \mathbb, \mathfrak fonts
\usepackage{mathrsfs} % for \mathscr font
\DeclareMathAlphabet{\mathpzc}{OT1}{pzc}{m}{it} % defines \mathpzc for Zapf Chancery (standard postscript) font

% other packages
\usepackage{datetime} % allows easy formatting of dates, e.g. \formatdate{dd}{mm}{yyyy}
\usepackage{caption} % makes figure captions better, more configurable
\usepackage{enumitem} % allows for custom labels on enumerated lists, e.g. \begin{enumerate}[label=\textbf{(\alph*)}]
\usepackage[squaren]{SIunits} % for nice units formatting e.g. \unit{50}{\kilo\gram}
\usepackage{cancel} % for crossing out terms with \cancel
\usepackage{verbatim} % for verbatim and comment environments
\usepackage{tensor} % for \indices e.g. M\indices{^a_b^{cd}_e}, and \tensor e.g. \tensor[^a_b^c_d]{M}{^a_b^c_d}
\usepackage{feynmp-auto} % for Feynman diagrams. 
\usepackage{pgfplots} % for plotting in tikzpicture environment

% new commands
\newcommand{\beg}{\begin} % a few letters less for beginning environments
\newenvironment{eqn}{\begin{equation}}{\end{equation}} % a lot fewer letter for equation environment

% notational commands
\newcommand{\opname}[1]{\operatorname{#1}} % custom operator names
\newcommand{\fslash}[1]{#1\!\!\!/} % feynman slash
\newcommand{\pd}{\partial} % partial differential shortcut
\newcommand{\ket}[1]{\left| #1 \right>} % for Dirac kets
\newcommand{\bra}[1]{\left< #1 \right|} % for Dirac bras
\newcommand{\braket}[2]{\left< #1 \vphantom{#2} \right| 
	\left. #2 \vphantom{#1} \right>} % for Dirac brackets
%\let\underdot=\d % rename builtin command \d{} to \underdot{}
%\renewcommand{\d}[2]{\frac{d #1}{d #2}} % for derivatives
%\newcommand{\pd}[2]{\frac{\partial #1}{\partial #2}} % for partial derivatives
%\newcommand{\fd}[2]{\frac{\delta #1}{\delta #2}} % for functional derivatives
\let\vaccent=\v % rename builtin command \v{} to \vaccent{}
%\renewcommand{\v}[1]{\ensuremath{\mathbf{#1}}} % for vectors
\renewcommand{\v}[1]{\ensuremath{\boldsymbol{\mathbf{#1}}}} % for vectors
%\newcommand{\gv}[1]{\ensurmath{\mbox{\boldmath$ #1 $}}} % for vectors of Greek letters
\newcommand{\uv}[1]{\ensuremath{\boldsymbol{\mathbf{\widehat{#1}}}}} % for unit vectors
\newcommand{\abs}[1]{\left| #1 \right|} % for absolute value ||x||
%\newcommand{\mag}{\abs} % magnitude, just another name for \abs
\newcommand{\norm}[1]{\left\Vert #1 \right\Vert} % for norm ||v||
\newcommand{\avg}[1]{\left< #1 \right>} % for average <x>
\newcommand{\inner}[2]{\left< #1, #2 \right>} % for inner product <x,y>
\newcommand{\set}[1]{ \left\{ #1 \right\} } % for sets {a,b,c,...}
\newcommand{\tr}{\opname{tr}} % for trace
\newcommand{\Tr}{\opname{Tr}} % for Trace

% notational shortcuts
\newcommand{\reals}{\mathbb{R}} % real numbers
\newcommand{\complexes}{\mathbb{C}} % complex numbers
\newcommand{\nats}{\mathbb{N}} % natural numbers
\newcommand{\irrats}{\mathbb{Q}} % irrationals
\newcommand{\quats}{\mathbb{H}} % quaternions (a la Hamilton)
\newcommand{\euclids}{\mathbb{E}} % Euclidean space
\newcommand{\bigo}{\mathcal{O}} % big O notation
\newcommand{\Lag}{\mathcal{L}} % fancy Lagrangian
\newcommand{\Ham}{\mathcal{H}} % fancy Hamiltonian





%%%%%%%%%%%%%%%%%%%
% some templates for various things
\begin{comment}

% template for figures
\begin{figure}
\centering
\includegraphics{myfile.png}
\caption{This is a caption}
\label{fig:myfigure}
\end{figure}

% template for Feynman diagrams using feynmf/feynmp
\begin{fmfgraph*}(40,25)
\fmfleft{em,ep}
\fmf{fermion}{em,Zee,ep}
\fmf{photon,label=$Z$}{Zee,Zff}
\fmf{fermion}{fb,Zff,f}
\fmfright{fb,f}
\fmfdot{Zee,Zff}
\end{fmfgraph*}

% template for drawing plots with pgfplot
\pgfplotsset{compat=1.3,compat/path replacement=1.5.1}
\begin{tikzpicture}
\begin{axis}[
extra x ticks={-2,2},
extra y ticks={-2,2},
extra tick style={grid=major}]
\addplot {x};
\draw (axis cs:0,0) circle[radius=2];
\end{axis}
\end{tikzpicture}

\end{comment}
%%%%%%%%%%%%%%%%%%%



\title{Math 217 -- Geometry and Physics -- Lec03}
\author{UCLA, Fall 2014}
\date{\formatdate{08}{10}{2014}} % Activate to display a given date or no date (if empty),
         % otherwise the current date is printed 

\begin{document}
\maketitle


\section{Stuff}

Seminal paper: Witten --- Supersymmetry and Morse Theory (J. Diff. Geometry, 1982)

\subsection{Chern conjecture}

Let's go over the Chern conjecture from last time. If $M = M^{2n}$ is a compact closed manifold, then
\begin{equation}
\chi(M) = \int_M \varepsilon(TM).
\end{equation}
Given the metric $g$ on $TM$, we have the Levi-Civita connection $\nabla^g$. Then we have $R^{\nabla^g} = (\nabla^g)^2$ where $R$ is skew-symmetric, and $\abs{Pf(R^{\nabla^g})} = \varepsilon(TM) \in H^{2n}(M)$. [not this this is right, it was fast and messy on the board.] If $R^{\nabla^g} = 0$ then $\chi(M)= 0$ where $\chi(M) = \sum_{i=0}^{2n} (-1)^i b_i$, $b_i = \opname{dim}_\mathbb{R} H^i (M)$

The Chern conjecture is based on that. Conjecture: If $M^{2n}$ is an affine flat closed compact manifold, then $\chi(M) = 0$. Let's define what affine flat means. Write $M = \cup_{\alpha \in I} U_\alpha$. Then it is affine flat $\iff$ $\varphi_\beta \circ \varphi_\alpha^{-1}(x_\alpha) = A_{\alpha\beta} x_\alpha + B_{\alpha\beta}$ where $A_{\alpha\beta}, B_{\alpha\beta}$ are constant matrices $\iff$ there exists an affine connection $\nabla : \Gamma(TM) \rightarrow \Gamma(T^* M \otimes TM)$ where $\nabla(fs) = f \nabla s + f \nabla s$. 

The Chern conjecture is known in 2 dimensions (the torus). Not sure if known in any other dimensions. (See Sullivan-Kostant-Milnor for special cases.) 


\subsection{Poincare duality}

Says that $H^p \cong H^{n-p}$. Exercise (Bott-Tu):
\begin{enumerate}
\item $M = \mathbb{R}^n$. 
\begin{equation}
	H^k (\mathbb{R}^n) = 
	\begin{cases}
	0, & k = 0, \\
	\mathbb{R}, & k \neq 0
	\end{cases}
	\qquad
	H_c^k (\mathbb{R}^n) = 
	\begin{cases}
	0, & k \neq n \\
	\mathbb{R}, & k = n.
	\end{cases}
\end{equation}

\item $M = \cup_{\alpha \in I} U_\alpha$. 
\end{enumerate}

We write $[\Delta] \in H_p (M, \mathbb{R}) \cong H^{n-p}(M, \mathbb{R})$ where $\Delta \subseteq M$ is a $p$-cycle in $M$. Then we can write 
\begin{align}
\int_\Delta \omega : H_\text{dR}^p(M) &\rightarrow \mathbb{R} \\
[\omega] &\mapsto \int_\Delta \omega
\end{align}
for a $p$-form $\omega$. Write $\Delta = \sum_i a_i \underbrace{\Delta_i}_{=\varphi_i(\tilde{\Delta}_i)}$ where $\tilde{\Delta}_i \subseteq [\text{something}] \overset{\varphi_i}{\longrightarrow} M$

[missed a LOT here on whatever crap he was talking about]


\subsection{Lie groups}

Definition: A Lie group $G$ is a smooth manifold with a group structure in the following sense:
\begin{enumerate}
\item Multiplication $G \times G \rightarrow G$ taking $(g, g') \mapsto gg'$, and
\item Inverse $G \rightarrow G$ taking $g \mapsto g^{-1}$
\end{enumerate}
Examples: $\mathbb{C}, \mathbb{C}, \mathbb{Q}, \mathbb{Q}_p, \dots$

Write $M_n(K) = \text{$n \times n$ matrices with entries in $K$}$. The group with elements $g = (a_{ij})_{n \times n} \in GL_n(\mathbb{R})$ with $\det g \neq 0$ is called the general linear group. Any subgroup submanifold is also a Lie group. 

Definition: A Lie subgroup $H$ of $G$ is a subgroup which is also a submanifold. If $H = \mathbb{R}$ then $i : \mathbb{R} \rightarrow G$ defines a one-parameter subgroup. 

Examples: $SL_n(\mathbb{R}), O_n(\mathbb{R}), SO_n(\mathbb{R}), U_n, SU_n$

Definition: A Lie group homomorphism $\psi : G \rightarrow G'$ is a map satisfying $\psi(gg') = \psi(g) \psi(g')$. If $\psi$ is a diffeomorphism then it is called an isomorphism. 

Furthermore, define
\begin{itemize}
\item $R_g : G \rightarrow G$ mapping $R_g (g') = g' g$ for all $g' \in G$
\item $L_g : G \rightarrow G$ mapping $L_g (g') = g g'$ for all $g' in G$.
\end{itemize}

Definition: A vector field $X$ on $G$ is left invariant if $L_g X = X$ 






\begin{comment}
\begin{figure}
\centering
\includegraphics{3a.pdf}
\caption{Half the diagrams for photon-photon scattering.}
\label{fig:3a}
\end{figure}
\end{comment}



\end{document}
