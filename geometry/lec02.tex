% declare document class and geometry
\documentclass[12pt]{article} % use larger type; default would be 10pt
\usepackage[margin=1in]{geometry} % handle page geometry

% import packages and commands
% standard packages
\usepackage{graphicx} % support the \includegraphics command and options
\usepackage{amsmath} % for nice math commands and environments

% font packages
\usepackage{amssymb} % for \mathbb, \mathfrak fonts
\usepackage{mathrsfs} % for \mathscr font
\DeclareMathAlphabet{\mathpzc}{OT1}{pzc}{m}{it} % defines \mathpzc for Zapf Chancery (standard postscript) font

% other packages
\usepackage{datetime} % allows easy formatting of dates, e.g. \formatdate{dd}{mm}{yyyy}
\usepackage{caption} % makes figure captions better, more configurable
\usepackage{enumitem} % allows for custom labels on enumerated lists, e.g. \begin{enumerate}[label=\textbf{(\alph*)}]
\usepackage[squaren]{SIunits} % for nice units formatting e.g. \unit{50}{\kilo\gram}
\usepackage{cancel} % for crossing out terms with \cancel
\usepackage{verbatim} % for verbatim and comment environments
\usepackage{tensor} % for \indices e.g. M\indices{^a_b^{cd}_e}, and \tensor e.g. \tensor[^a_b^c_d]{M}{^a_b^c_d}
\usepackage{feynmp-auto} % for Feynman diagrams. 
\usepackage{pgfplots} % for plotting in tikzpicture environment

% new commands
\newcommand{\beg}{\begin} % a few letters less for beginning environments
\newenvironment{eqn}{\begin{equation}}{\end{equation}} % a lot fewer letter for equation environment

% notational commands
\newcommand{\opname}[1]{\operatorname{#1}} % custom operator names
\newcommand{\fslash}[1]{#1\!\!\!/} % feynman slash
\newcommand{\pd}{\partial} % partial differential shortcut
\newcommand{\ket}[1]{\left| #1 \right>} % for Dirac kets
\newcommand{\bra}[1]{\left< #1 \right|} % for Dirac bras
\newcommand{\braket}[2]{\left< #1 \vphantom{#2} \right| 
	\left. #2 \vphantom{#1} \right>} % for Dirac brackets
%\let\underdot=\d % rename builtin command \d{} to \underdot{}
%\renewcommand{\d}[2]{\frac{d #1}{d #2}} % for derivatives
%\newcommand{\pd}[2]{\frac{\partial #1}{\partial #2}} % for partial derivatives
%\newcommand{\fd}[2]{\frac{\delta #1}{\delta #2}} % for functional derivatives
\let\vaccent=\v % rename builtin command \v{} to \vaccent{}
%\renewcommand{\v}[1]{\ensuremath{\mathbf{#1}}} % for vectors
\renewcommand{\v}[1]{\ensuremath{\boldsymbol{\mathbf{#1}}}} % for vectors
%\newcommand{\gv}[1]{\ensurmath{\mbox{\boldmath$ #1 $}}} % for vectors of Greek letters
\newcommand{\uv}[1]{\ensuremath{\boldsymbol{\mathbf{\widehat{#1}}}}} % for unit vectors
\newcommand{\abs}[1]{\left| #1 \right|} % for absolute value ||x||
%\newcommand{\mag}{\abs} % magnitude, just another name for \abs
\newcommand{\norm}[1]{\left\Vert #1 \right\Vert} % for norm ||v||
\newcommand{\avg}[1]{\left< #1 \right>} % for average <x>
\newcommand{\inner}[2]{\left< #1, #2 \right>} % for inner product <x,y>
\newcommand{\set}[1]{ \left\{ #1 \right\} } % for sets {a,b,c,...}
\newcommand{\tr}{\opname{tr}} % for trace
\newcommand{\Tr}{\opname{Tr}} % for Trace

% notational shortcuts
\newcommand{\reals}{\mathbb{R}} % real numbers
\newcommand{\complexes}{\mathbb{C}} % complex numbers
\newcommand{\nats}{\mathbb{N}} % natural numbers
\newcommand{\irrats}{\mathbb{Q}} % irrationals
\newcommand{\quats}{\mathbb{H}} % quaternions (a la Hamilton)
\newcommand{\euclids}{\mathbb{E}} % Euclidean space
\newcommand{\bigo}{\mathcal{O}} % big O notation
\newcommand{\Lag}{\mathcal{L}} % fancy Lagrangian
\newcommand{\Ham}{\mathcal{H}} % fancy Hamiltonian





%%%%%%%%%%%%%%%%%%%
% some templates for various things
\begin{comment}

% template for figures
\begin{figure}
\centering
\includegraphics{myfile.png}
\caption{This is a caption}
\label{fig:myfigure}
\end{figure}

% template for Feynman diagrams using feynmf/feynmp
\begin{fmfgraph*}(40,25)
\fmfleft{em,ep}
\fmf{fermion}{em,Zee,ep}
\fmf{photon,label=$Z$}{Zee,Zff}
\fmf{fermion}{fb,Zff,f}
\fmfright{fb,f}
\fmfdot{Zee,Zff}
\end{fmfgraph*}

% template for drawing plots with pgfplot
\pgfplotsset{compat=1.3,compat/path replacement=1.5.1}
\begin{tikzpicture}
\begin{axis}[
extra x ticks={-2,2},
extra y ticks={-2,2},
extra tick style={grid=major}]
\addplot {x};
\draw (axis cs:0,0) circle[radius=2];
\end{axis}
\end{tikzpicture}

\end{comment}
%%%%%%%%%%%%%%%%%%%



\title{Math 217 -- Geometry and Physics -- Lec02}
\author{UCLA, Fall 2014}
\date{\formatdate{06}{10}{2014}} % Activate to display a given date or no date (if empty),
         % otherwise the current date is printed 

\begin{document}
\maketitle


\section{More on manifolds}


\subsection{Remarks}

\begin{enumerate}
\item Partition of unity: $M = \cup_{\alpha \in I} U_\alpha$ where $U_\alpha$ coordinate charts. There exists $\rho_\alpha : M \rightarrow \mathbb{R}$, $\rho_\alpha \geq 0$ smooth with $\opname{supp}(\rho_\alpha) \subset U_\alpha$ and $\sum_\alpha \rho_\alpha = 1$. $\omega = \sum_\alpha \rho_\alpha \omega$, $\opname{supp}(\rho_\alpha \omega) \subset U_\alpha$.
\item Sheaf theory. local $\implies$ global (Cohomology)
\item Characteristic classes. local computations $\implies$ global invariants.
\item $M ( = \cup_{\alpha \in I} U_\alpha) \overset{f}{\longrightarrow} N ( = \cup_\beta V_\beta)$ is smooth iff $\psi_\beta \circ f \circ \varphi_\alpha^{-1} : \varphi_\alpha(U_\alpha) \rightarrow \psi_\beta (V_\beta)$ [is smooth?]. Where $\varphi_\alpha : U_\alpha \rightarrow \mathbb{R}^n$, $\psi_\beta : V_\beta \rightarrow \mathbb{R}^n$. 
\item $\mathbb{CP}^n = \mathbb{C}^n \cup \mathbb{CP}^{n-1} = \dots = \mathbb{C}^n \cup \mathbb{C}^{n-1} \cup \dots \cup \mathbb{C} \cup \set{ \rho_+ }$ [rho sub what?]
\end{enumerate}

Given a tangent bundle $TM$ we have a vector field (smooth section) $X = \sum_i a_i \frac{\pd}{\pd x_i}$ and differential form $\omega = \sum_i a_i dx_i$, where $\langle dx_i, \frac{\pd}{\pd x_j} \rangle = \delta_{ij}$. Furthermore, a differential $p$-form $\eta = \sum_I a_I dx_I \in A^p (M)$, where $I = \set{ 1 \leq i_1 < \dots < i_p \leq n }$ and $1 \leq p \leq n$, is a section on $\Lambda^p T^* M$. $p = \opname{deg} \eta$.

The exterior product $\Lambda^p T^* M \overset{\pi}{\longrightarrow} M$. 

Define $A(M) = A^*(M) = \oplus_{p=0}^n A^p (M)$. Then $\eta \wedge \eta' = (-1)^{pq} \eta' \wedge \eta$. 

$M$ is orientable iff there exists a nowhere vanishing $n$-form $\omega$ on $M^n$. 

$d : A^p (M) \rightarrow A^{p+1}(M)$, $d\eta = \sum_I da_I \wedge dx_I$.

$da_I = \sum_{j=1}^n \frac{\pd a_I}{\pd x_j} dx_j$.

\subsection{Exercise}

\begin{enumerate}
\item $d^2 = 0$. $A^p(M) \overset{d}{\longrightarrow} A^{p+1}(M) \overset{d}{\longrightarrow} A^{p+2}(M)$. $\opname{\eta} = p$.
\item Leibniz rule: $d(\eta \wedge \eta') = d\eta \wedge \eta' + (-1)^p \eta \wedge d\eta'$.
\item $\int_M \omega = \sum_{\alpha \in I} \int_{U_\alpha} (\rho_\alpha \omega) = \int_M (\sum_\alpha \rho_\alpha) \omega$. 

$\int_{U_\alpha} (\rho_\alpha \omega) = \int_{\varphi(U_\alpha) \subseteq \mathbb{R}^n} (\varphi_\alpha^{-1})^* (\rho_\alpha \omega)$
\end{enumerate}

\subsection{Stokes}

$\int_M d\eta = \int_{\pd M} \eta$ where $\eta \in A^{n-1}(M)$ implies that $\int_M d\eta = 0$ if $M$ closed compact. 

Consider: $f : M \rightarrow M'$ a smooth map. Then the pullback $f^* \eta'$ of $\eta' = \sum_I b_I(y) dy_I \in A(M')$ is given by 
\begin{equation}
f^* \eta'(x) = \sum_I b_I(f(x)) df_I.
\end{equation}

\subsection{exercise}

\begin{enumerate}
\item Chain rule: $f^* \circ d_{M'} = d_M f^*$
\item $f^*(\eta_1 \wedge \eta_2) = (f^* \eta_1) \wedge (f^* \eta_2)$
\end{enumerate}

\subsection{de Rham Cohomology}

\begin{equation}
H_{\text{dR}}^p(M, \mathbb{R}) = \frac{\opname{Ker} d}{\opname{Im} d} \Big|_{A^p(M)}
\end{equation}
where of course $\opname{Im} d \subseteq \opname{Ker} d$. Then 
\begin{equation}
0 \rightarrow A^0(M) \overset{d}{\longrightarrow} A^1(M) \rightarrow \dots \overset{d}{\longrightarrow} A^p(M) \overset{d}{\longrightarrow} \dots \overset{d}{\longrightarrow} A^n(M) \rightarrow 0
\end{equation}

\subsection{Singular homology, Cohomology}

Famous theorem of de Rham: $H_{\text{dR}}^p(M, \mathbb{R}) \cong H_{\text{sing}}^p(M, \mathbb{R})$ when $M$ is a compact, closed manifold. We will also see later that it is $\cong \opname{Ker} \Delta_d$, where $\Delta_d$ is the elliptic operator. 

Cohomology ring: $H_\text{dR}^* (M) = \oplus_{p=0}^n H_\text{dR}^p(M)$. 

\begin{align}
&H_\text{dR}^p(M) &\times &H_\text{dR}^q(M) &\rightarrow H_\text{dR}^{p+q}(M) \\
&[\eta] = [\eta+d\alpha] &\quad &[\eta'] = [\eta+d\alpha'] &\mapsto [\eta \wedge \eta']
\end{align}

\subsection{Poincare duality}

\begin{align}
f^* : & H_\text{dR}^*(M') \rightarrow H_\text{dR}^*(M) \\
&[\eta'] \mapsto [f^* \eta']
\end{align}

\begin{align}
&H_\text{dR}^p(M) &\times &H_\text{dR}^{n-p}(M) &\rightarrow \mathbb{R} \\
&[\eta] = [\eta+d\alpha] &\quad &[\eta'] = [\eta'] &\mapsto \int_M \eta \wedge \eta'
\end{align}


\subsection{Betti numbers}

\begin{equation}
H_\text{dR}^{n-}(M, \mathbb{R}) \cong H_\text{dR}^p(M, \mathbb{R}) \cong H_\text{dR}^p(M, \mathbb{R}) \implies b_p(M) = b_{n-p}(M)
\end{equation}
where
\begin{equation}
b_p(M) = \opname{dim}_\mathbb{R} H_\text{dR}^p(M, \mathbb{R})
\end{equation}
is the $p$th Betti number of $M$.

Euler number: $\chi(M) = \sum_{p=0}^n (-1)^p b_p(M) = \int_M^{2n} e(TM)$ by Gauss-Bonnet, where $e(TM)$ is the Euler characteristic class.

\textbf{Corollary}: $\opname{dim}_\mathbb{R} M = \text{odd} \implies \chi(M) = 0$

Conjecture (Chern): Let $M$ be an affine-flat compact closed manifold. Then $\chi(M) = 0$. 






\begin{comment}
\begin{figure}
\centering
\includegraphics{3a.pdf}
\caption{Half the diagrams for photon-photon scattering.}
\label{fig:3a}
\end{figure}
\end{comment}



\end{document}
