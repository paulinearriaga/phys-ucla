%%%%%%%%%%%%%%%%%%
%% CONTENTS
%%%%%%%%%%%%%%%%%%
% Packages
% Commands
% Environments
% More commands: Resizable delimiters
% More commands: Derivatives
% Useful templates
% Notes
%%%%%%%%%%%%%%%%%%
%%%%%%%%%%%%%%%%%%





%%%%%%%%%%%%%%%%%%
%% PACKAGES
%%%%%%%%%%%%%%%%%%


% standard and structural packages
\usepackage{graphicx} % support the \includegraphics command and options
\usepackage[font=footnotesize,labelfont=bf,labelsep=space]{caption} 
	% makes figure captions better, more configurable.
	% specific options make captions smaller and the label bold.
\usepackage{subcaption} % allows for subfigures and subcaptions
	% can add usual caption options to change format for subcaptions
\usepackage{epstopdf} % allows complier to convert eps files to pdf files for inclusion in document
\usepackage{amsmath} % for nice math commands and environments
\usepackage{mathtools} % extends amsmath with bug fixes and useful commands, e.g.
	% \shortintertext for short interjections in align environment,
	% \prescript{t}{b}{X} for simple, nicely aligned math pre-(super/sub)scripts
\usepackage{amsthm} % for theorem and proof environments
\usepackage[pdfencoding=auto,psdextra]{hyperref} % adds hyperlinks and outline to PDF documents
	% otpions enable enhanced unicode and math support in PDF outlines [causes conflict with \C command?]

% font packages
\usepackage{amssymb} % for \mathbb, \mathfrak fonts
\usepackage{mathrsfs} % for \mathscr font
\usepackage{dsfont} % best option for non-alphabetical doublestruck characters
\usepackage{bbm} % alternative option for non-alphabetical doublestruck characters
\DeclareMathAlphabet{\mathpzc}{OT1}{pzc}{m}{it} % defines \mathpzc for Zapf Chancery (standard postscript) font
	% basically another calligraphic font, alternative to mathcal

% other packages
\usepackage{datetime} % allows easy formatting of dates, e.g. \formatdate{dd}{mm}{yyyy}
\usepackage[inline]{enumitem} % allows for custom labels on enumerated lists
	% e.g. \begin{enumerate}[label=\textbf{(\alph*)}]
	% inline option creates '*' versions of enumerate, itemize, description 
		% which can be inlined within the text of a paragraph. 
\usepackage[squaren]{SIunits} % for nice units formatting e.g. \unit{50}{\kilo\gram}
\usepackage{cancel} % for crossing out terms with \cancel
\usepackage{verbatim} % for verbatim and comment environments
\usepackage{tensor} % for \indices e.g. M\indices{^a_b^{cd}_e}, and \tensor e.g. \tensor[^a_b^c_d]{M}{^a_b^c_d}
\usepackage{mhchem} % for writing chemical formulas with \ce, e.g. \ce{AgCl2-} or \ce{^{227}_{90}Th+}
\usepackage{feynmp-auto} % for Feynman diagrams. 
\usepackage{pgfplots} % for plotting in tikzpicture environment
\usepackage{commath} % for some nice standardized syntax stuff. 
	% \dif, \Dif, \od, \pd, \md, \(abs | envert), \(norm | enVert), \(set | cbr), \sbr, \eval, \int(o | c)(o | c), etc
\usepackage{slashed} % provides a command \slashed[1] for Feynman slash notation
\usepackage{listings} % for typesetting source code
\usepackage{algorithmicx} % for typesetting algorithms and pseudocode

\usepackage{rotating} 
	% provides environments for rotating arbitrary objects, e.g. sideways, turn[ang], rotate[ang]
	% also provides macro \turnbox{ang}{stuff}
%\newcommand{\sideways}[1]{\begin{sideways} #1 \end{sideways}} % turn things 90 degrees CCW
%\newcommand{\turn}[2][]{\begin{turn}{#2} #1 \end{turn}} % \turn[ang]{stuff} turns things arbitrary +/- angle




%%%%%%%%%%%%%%%%%%
%% COMMANDS
%%%%%%%%%%%%%%%%%%


% functions and operators
\newcommand{\opname}[1]{\operatorname{#1}} % custom operator names
\newcommand{\tr}{\opname{tr}} % for trace
\newcommand{\Tr}{\opname{Tr}} % for Trace
\newcommand{\rank}{\opname{rank}} % for rank
\newcommand{\diag}{\opname{diag}} % i.e. \diag(\lambda_1, \dots, \lambda_n)
\let \fancyre = \Re
\let \fancyim = \Im
\renewcommand{\Re}{\opname{Re}}
\renewcommand{\Im}{\opname{Im}}
\newcommand{\Res}{\opname{Res}\limits} % for residue function -- change to put limits on bottom
\newcommand{\inv}{^{-1}}
\newcommand{\sech}{\opname{sech}}
\newcommand{\csch}{\opname{csch}}


% vector and other notational commands
\let \vaccent = \v % rename builtin command \v{} to \vaccent{}
\renewcommand{\v}[1]{\ensuremath{\boldsymbol{\mathbf{#1}}}} % for vectors
\newcommand{\uv}[1]{\ensuremath{\boldsymbol{\mathbf{\widehat{#1}}}}} % for unit vectors
\newcommand{\vd}[1]{\v{\dot{#1}}} % for dotted vectors
\newcommand{\vdd}[1]{\v{\ddot{#1}}} % for ddotted vectors
\newcommand{\vddd}[1]{\v{\dddot{#1}}} % for dddotted vectors
\newcommand{\vdddd}[1]{\v{\ddddot{#1}}} % for ddddotted vectors
\renewcommand{\bbar}[1]{\bar{\bar{#1}}} % for barring things twice -- use \cbar or \zbar instead of original \bbar
\newcommand{\bbbar}[1]{\bar{\bbar{#1}}}
\newcommand{\bbbbar}[1]{\bar{\bbbar{#1}}}

% deprecated by commath package
%\newcommand{\abs}[1]{\left| #1 \right|} % for absolute value ||x||
%\newcommand{\mag}{\abs} % magnitude, just another name for \abs
%\newcommand{\norm}[1]{\left\Vert #1 \right\Vert} % for norm ||v||


% more stuff
\newcommand{\grad}[1]{\v{\nabla} #1} % for gradient
\let \divsymb = \div % rename builtin command \div to \divsymb
\renewcommand{\div}[1]{\v{\nabla} \cdot #1} % for divergence
\newcommand{\curl}[1]{\v{\nabla} \times #1} % for curl
\let \baraccent = \= % rename builtin command \= to \baraccent
\renewcommand{\=}[1]{\stackrel{#1}{=}} % for putting numbers above =


% notational shortcuts
\newcommand{\bigO}{\mathcal{O}} % big O notation
\let \bigo = \bigO % deprecated version. keeping for now because need to update instances in older files
\newcommand{\Lag}{\mathcal{L}} % fancy Lagrangian
\newcommand{\Ham}{\mathcal{H}} % fancy Hamiltonian
% fields and spaces
\newcommand{\reals}{\mathbb{R}} % real numbers
\newcommand{\complexes}{\mathbb{C}} % complex numbers
\newcommand{\ints}{\mathbb{Z}} % integers
\newcommand{\nats}{\mathbb{N}} % natural numbers
\newcommand{\irrats}{\mathbb{Q}} % irrationals
\newcommand{\quats}{\mathbb{H}} % quaternions (a la Hamilton)
\newcommand{\euclids}{\mathbb{E}} % Euclidean space
\newcommand{\R}{\reals}
\renewcommand{\C}{\complexes}
\newcommand{\Z}{\ints}
\newcommand{\Q}{\irrats}
\newcommand{\N}{\nats}
\newcommand{\E}{\euclids}
\newcommand{\RP}{\mathbb{RP}} % real projective space
\newcommand{\CP}{\mathbb{CP}} % complex projective space
% Lie groups
\newcommand{\SO}{\mathrm{SO}}
\newcommand{\SU}{\mathrm{SU}}
% etc


% matrix shortcuts! only use for small vectors/matrices and such
\newcommand{\pmat}[1]{\begin{pmatrix} #1 \end{pmatrix}}
\newcommand{\bmat}[1]{\begin{bmatrix} #1 \end{bmatrix}}
\newcommand{\Bmat}[1]{\begin{Bmatrix} #1 \end{Bmatrix}}
\newcommand{\vmat}[1]{\begin{vmatrix} #1 \end{vmatrix}}
\newcommand{\Vmat}[1]{\begin{Vmatrix} #1 \end{Vmatrix}}






%%%%%%%%%%%%%%%%%%
%% ENVIRONMENTS
%%%%%%%%%%%%%%%%%%


\newcommand{\beg}{\begin} % a few letters less for beginning environments
\newenvironment{eqn}{\begin{equation}}{\end{equation}} % a lot fewer letter for equation environment

\newenvironment{enumproblem}{
	\begin{enumerate}[label=\textbf{(\alph*)},listparindent=\parindent]
	}{\end{enumerate}}
	% for easily enumerating letters in problems
	

% theorem-style environments. note amsthm builtin proof environment: \begin{proof}[title]
% appending [section] resets counter and prepends section number
% use \setcounter{counter}{0} to reset counter
% typical use cases:
% plain: Theorem, Lemma, Corollary, Proposition, Conjecture, Criterion, Algorithm
% definition: Definition, Condition, Problem, Example
% remark: Remark, Note, Notation, Claim, Summary, Acknowledgment, Case, Conclusion
\theoremstyle{plain} % default
\newtheorem{theorem}{Theorem}[section]
\newtheorem{lemma}[theorem]{Lemma}
\newtheorem{corollary}[theorem]{Corollary}
\newtheorem{proposition}[theorem]{Proposition}
\newtheorem{conjecture}[theorem]{Conjecture}
% definition style
\theoremstyle{definition}
\newtheorem{definition}{Definition}
\newtheorem{problem}{Problem}
\newtheorem{exercise}{Exercise}
\newtheorem{example}{Example}
% remark style
\theoremstyle{remark}
\newtheorem{remark}{Remark}
\newtheorem{note}{Note}
\newtheorem{claim}{Claim}
\newtheorem{conclusion}{Conclusion}
% to-do: add problem/subproblem/answer environments for homeworks





%%%%%%%%%%%%%%%%%%
% RESIZEABLE DELIMITERS
%%%%%%%%%%%%%%%%%%


% the value for the optional argument ranges from -1 to 4 with higher values resulting in larger delimiters
% default value for #1 is -1 which results in automatic size for the delimeters
\providecommand{\bra}[2][-1]{
\ensuremath{\mathinner{
\ifthenelse{\equal{#1}{-1}}{ % if
\left< #2 \right| }{}
\ifthenelse{\equal{#1}{0}}{ % if
\langle #2 | }{}
\ifthenelse{\equal{#1}{1}}{ % if
\bigl< #2 \bigr| }{}
\ifthenelse{\equal{#1}{2}}{ % if
\Bigl< #2 \Bigr| }{}
\ifthenelse{\equal{#1}{3}}{ % if
\biggl< #2 \biggr| }{}
\ifthenelse{\equal{#1}{4}}{ % if
\Biggl< #2 \Biggr| }{}
}} % \ensuremath{\mathinner{
}  % bra - Dirac bra <x|. the optional argument describes the size of the delimiters, no argument results in automatic size

% the value for the optional argument ranges from -1 to 4 with higher values resulting in larger delimiters
% default value for #1 is -1 which results in automatic size for the delimeters
\providecommand{\ket}[2][-1]{
\ensuremath{\mathinner{
\ifthenelse{\equal{#1}{-1}}{ % if
\left| #2 \right> }{}
\ifthenelse{\equal{#1}{0}}{ % if
| #2 \rangle }{}
\ifthenelse{\equal{#1}{1}}{ % if
\bigl| #2 \bigr> }{}
\ifthenelse{\equal{#1}{2}}{ % if
\Bigl| #2 \Bigr> }{}
\ifthenelse{\equal{#1}{3}}{ % if
\biggl| #2 \biggr> }{}
\ifthenelse{\equal{#1}{4}}{ % if
\Biggl| #2 \Biggr> }{}
}} % \ensuremath{\mathinner{
}  % bra - Dirac ket |x>. the optional argument describes the size of the delimiters, no argument results in automatic size

% the value for the optional argument ranges from -1 to 4 with higher values resulting in larger delimiters
% default value for #1 is -1 which results in automatic size for the delimeters
\providecommand{\braket}[3][-1]{
\ensuremath{\mathinner{
\ifthenelse{\equal{#1}{-1}}{ % if
\left< #2 \vphantom{#3} \right| \left. #3 \vphantom{#2} \right> }{}
\ifthenelse{\equal{#1}{0}}{ % if
\langle #2 | #3 \rangle }{}
\ifthenelse{\equal{#1}{1}}{ % if
\bigl< #2 \vphantom{#3} \bigr| \bigl. #3 \vphantom{#2} \bigr> }{}
\ifthenelse{\equal{#1}{2}}{ % if
\Bigl< #2 \vphantom{#3} \Bigr| \Bigl. #3 \vphantom{#2} \Bigr> }{}
\ifthenelse{\equal{#1}{3}}{ % if
\biggl< #2 \vphantom{#3} \biggr| \biggl. #3 \vphantom{#2} \biggr> }{}
\ifthenelse{\equal{#1}{4}}{ % if
\Biggl< #2 \vphantom{#3} \Biggr| \Biggl. #3 \vphantom{#2} \Biggr> }{}
}} % \ensuremath{\mathinner{
}  % braket - Dirac bra-ket <x|y>. the optional argument describes the size of the delimiters, no argument results in automatic size

% the value for the optional argument ranges from -1 to 4 with higher values resulting in larger delimiters
% default value for #1 is -1 which results in automatic size for the delimeters
\providecommand{\matrixel}[4][-1]{
\ensuremath{\mathinner{
\ifthenelse{\equal{#1}{-1}}{ % if
\left< #2 \vphantom{#3#4} \right| #3 \left| #4 \vphantom{#2#3} \right> }{}
\ifthenelse{\equal{#1}{0}}{ % if
\langle #2 \vphantom{#3#4} | #3 | #4 \vphantom{#2#3} \rangle }{}
\ifthenelse{\equal{#1}{1}}{ % if
\bigl< #2 \vphantom{#3#4} \bigr| #3 \bigl| #4 \vphantom{#2#3} \bigr> }{}
\ifthenelse{\equal{#1}{2}}{ % if
\Bigl< #2 \vphantom{#3#4} \Bigr| #3 \Bigl| #4 \vphantom{#2#3} \Bigr> }{}
\ifthenelse{\equal{#1}{3}}{ % if
\biggl< #2 \vphantom{#3#4} \biggr| #3 \biggl| #4 \vphantom{#2#3} \biggr> }{}
\ifthenelse{\equal{#1}{4}}{ % if
\Biggl< #2 \vphantom{#3#4} \Biggr| #3 \Biggl| #4 \vphantom{#2#3} \Biggr> }{}
}} % \ensuremath{\mathinner{
}  % matrixel - matrix element <x|A|y>. the optional argument describes the size of the delimiters, no argument results in automatic size

% the value for the optional argument ranges from -1 to 4 with higher values resulting in larger delimiters
% default value for #1 is -1 which results in automatic size for the delimeters
\providecommand{\avg}[2][-1]{
\ensuremath{\mathinner{
\ifthenelse{\equal{#1}{-1}}{ % if
\left< #2 \right> }{}
\ifthenelse{\equal{#1}{0}}{ % if
\langle #2 \rangle }{}
\ifthenelse{\equal{#1}{1}}{ % if
\bigl< #2 \bigr> }{}
\ifthenelse{\equal{#1}{2}}{ % if
\Bigl< #2 \Bigr> }{}
\ifthenelse{\equal{#1}{3}}{ % if
\biggl< #2 \biggr> }{}
\ifthenelse{\equal{#1}{4}}{ % if
\Biggl< #2 \Biggr> }{}
}} % \ensuremath{\mathinner{
}  % avg - average value <x>. the optional argument describes the size of the delimiters, no argument results in automatic size

% the value for the optional argument ranges from -1 to 4 with higher values resulting in larger delimiters
% default value for #1 is -1 which results in automatic size for the delimeters
\providecommand{\inner}[3][-1]{
\ensuremath{\mathinner{
\ifthenelse{\equal{#1}{-1}}{ % if
\left< #2 , #3 \right> }{}
\ifthenelse{\equal{#1}{0}}{ % if
\langle #2 , #3 \rangle }{}
\ifthenelse{\equal{#1}{1}}{ % if
\bigl< #2 , #3 \bigr> }{}
\ifthenelse{\equal{#1}{2}}{ % if
\Bigl< #2 , #3 \Bigr> }{}
\ifthenelse{\equal{#1}{3}}{ % if
\biggl< #2 , #3 \biggr> }{}
\ifthenelse{\equal{#1}{4}}{ % if
\Biggl< #2 , #3 \Biggr> }{}
}} % \ensuremath{\mathinner{
}  % inner - inner product <x,y>. the optional argument describes the size of the delimiters, no argument results in automatic size





%%%%%%%%%%%%%%%%%%
%% DERIVATIVES
%%%%%%%%%%%%%%%%%%



% BUG: derivatives revert to text mode often when in smaller environments in math mode?
% FIXED: below by setting defaults to display
% BUG: superscripts in bottom of derivative impose double superscript error 
% -- requires embedding within brackets, i.e. \pd{x^\mu}{{y^\nu}}
% -- redefine \pd and other derivatives?


% Command for functional derivatives. The first argument denotes the function and the second argument denotes the variable with respect to which the derivative is taken. The optional argument denotes the order of differentiation. The style (text style/display style) is determined automatically
\providecommand{\fd}[3][]{\ensuremath{
\ifinner
\tfrac{\delta{^{#1}}#2}{\delta{#3^{#1}}}
\else
\dfrac{\delta{^{#1}}#2}{\delta{#3^{#1}}}
\fi
}}

% \tfd[2]{f}{k} denotes the second functional derivative of f with respect to k
% The first letter t means "text style"
\providecommand{\tfd}[3][]{\ensuremath{\mathinner{
\tfrac{\delta{^{#1}}#2}{\delta{#3^{#1}}}
}}}
% \dfd[2]{f}{k} denotes the second functional derivative of f with respect to k
% The first letter d means "display style"
\providecommand{\dfd}[3][]{\ensuremath{\mathinner{
\dfrac{\delta{^{#1}}#2}{\delta{#3^{#1}}}
}}}

% mixed functional derivative - analogous to the functional derivative command
% \mfd{F}{5}{x}{2}{y}{3}
\providecommand{\mfd}[6]{\ensuremath{
\ifinner
\tfrac{\delta{^{#2}}#1}{\delta{#3^{#4}}\delta{#5^{#6}}}
\else
\dfrac{\delta{^{#2}}#1}{\delta{#3^{#4}}\delta{#5^{#6}}}
\fi
}}


% Command for thermodynamic (chemistry?) partial derivatives. The first argument denotes the function and the second argument denotes the variable with respect to which the derivative is taken. The optional argument denotes the order of differentiation. The style (text style/display style) is determined automatically
\providecommand{\pdc}[4][]{\ensuremath{
\ifinner
\left( \tfrac{\partial{^{#1}}#2}{\partial{#3^{#1}}} \right)_{#4}
\else
\left( \dfrac{\partial{^{#1}}#2}{\partial{#3^{#1}}} \right)_{#4}
\fi
}}

% \tpd[2]{f}{k} denotes the second thermo partial derivative of f with respect to k
% The first letter t means "text style"
\providecommand{\tpdc}[4][]{\ensuremath{\mathinner{
\left( \tfrac{\partial{^{#1}}#2}{\partial{#3^{#1}}} \right)_{#4}
}}}
% \dpd[2]{f}{k} denotes the second thermo partial derivative of f with respect to k
% The first letter d means "display style"
\providecommand{\dpdc}[4][]{\ensuremath{\mathinner{
\left( \dfrac{\partial{^{#1}}#2}{\partial{#3^{#1}}} \right)_{#4}
}}}



% some shorter synonyms for commath
\let \underdot = \d % rename builtin command \d{} to \underdot{}
\let \d = \od % for derivatives
\let \pd = \dpd % redefine \pd as \dpd because the ambiguity is annoying
\let \fd = \dfd % likewise for these
\let \od = \dod
\let \pdc = \dpdc

% fixes for commath's differentials, which look bad for powers, e.g. $\dif^3 x$
\renewcommand{\dif}{\mathrm{d}} % \opname{d} better maybe?
\renewcommand{\Dif}{\mathrm{D}}










%%%%%%%%%%%%%%%%%%
%% USEFUL TEMPLATES
%%%%%%%%%%%%%%%%%%

\begin{comment}

% template for figures
\begin{figure}
\centering
\includegraphics[width=0.8\textwidth]{myfile.png}
\caption{This is a caption}
\label{fig:myfigure}
\end{figure}

% template for subfigures
\begin{figure}
	\centering
	\begin{subfigure}[b]{0.3\textwidth}
		\includegraphics[width=\textwidth]{gull}
		\caption{A gull}
		\label{fig:gull}
	\end{subfigure}%
	~ %add desired spacing between images, e. g. ~, \quad, \qquad, \hfill etc.
		%(or a blank line to force the subfigure onto a new line)
	\begin{subfigure}[b]{0.3\textwidth}
		\includegraphics[width=\textwidth]{tiger}
		\caption{A tiger}
		\label{fig:tiger}
	\end{subfigure}
	~ %add desired spacing between images, e. g. ~, \quad, \qquad, \hfill etc.
	%(or a blank line to force the subfigure onto a new line)
	\begin{subfigure}[b]{0.3\textwidth}
		\includegraphics[width=\textwidth]{mouse}
		\caption{A mouse}
		\label{fig:mouse}
	\end{subfigure}
	\caption{Pictures of animals}\label{fig:animals}
\end{figure}

% template for Feynman diagrams using feynmf/feynmp
\begin{fmfgraph*}(40,25)
\fmfleft{em,ep}
\fmf{fermion}{em,Zee,ep}
\fmf{photon,label=$Z$}{Zee,Zff}
\fmf{fermion}{fb,Zff,f}
\fmfright{fb,f}
\fmfdot{Zee,Zff}
\end{fmfgraph*}

% template for drawing plots with pgfplot
\pgfplotsset{compat=1.3,compat/path replacement=1.5.1}
\begin{tikzpicture}
\begin{axis}[
extra x ticks={-2,2},
extra y ticks={-2,2},
extra tick style={grid=major}]
\addplot {x};
\draw (axis cs:0,0) circle[radius=2];
\end{axis}
\end{tikzpicture}

%% find package for easily drawing mapping / algebraic / commutative diagrams..

\end{comment}
%%%%%%%%%%%%%%%%%%





%%%%%%%%%%%%%%%%%%
%% NOTES
%%%%%%%%%%%%%%%%%%


%% Note on math environments
% 'equation' and 'align' are basic standards 
	% use '&' to for alignment in 'align'
	% multi-line environments all end lines with '\\' (except for last line)
	% generally all lines are tagged except in 'split' and '*' versions of environments
	% add \label{txt} after line to give LaTeX-label, \tag{txt} to change textual label
% use 'gather' to display multiple center-aligned lines (no manual alignment)
	% tags/labels every line
% use 'multline' to left-align first line, right-aline last line, and center-align all lines in between (no manual alignment)
	% only tags/labels last line
% use 'split' within 'equation' or 'align' to give single middle-aligned tag/label to the contained lines
	% use '&' for alignment within 'split'
	% no '*' version

%% Note on spacing
% 5) \qquad
% 4) \quad
% 3) \thickspace = \;
% 2) \medspace = \:
% 1) \thinspace = \,
% -1) \negthinspace = \!
% -2) \negmedspace
% -3) \negthickspace

%% Note on useful synonyms
% \rightarrow = \to
% \leftarrow = \gets
% \ni = \owns = backwards \in
% \cup = union, \cap = intersect
% \sqcup / \sqcap for disjoint varieties
% \gg = >>, \ll = <<
% \vert = | (pipe, e.g. absolute value delimiter), 
% \Vert = || (double pipes, e.g. norm delimiter)

