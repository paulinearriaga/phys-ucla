% declare document class and geometry
\documentclass[12pt]{article} % use larger type; default would be 10pt
\usepackage[english]{babel} % for hyphenation dictionary
%\setdefaultlanguage{english} % polyglossia command for use with XeTeX / LuaTeX
\usepackage[margin=1in]{geometry} % handle page geometry

% import packages and commands
% standard packages
\usepackage{graphicx} % support the \includegraphics command and options
\usepackage{amsmath} % for nice math commands and environments
\usepackage{mathtools} % extends amsmath with bug fixes and useful commands e.g. \shortintertext
\usepackage{amsthm} % for theorem and proof environments

% font packages
\usepackage{amssymb} % for \mathbb, \mathfrak fonts
\usepackage{mathrsfs} % for \mathscr font
\DeclareMathAlphabet{\mathpzc}{OT1}{pzc}{m}{it} % defines \mathpzc for Zapf Chancery (standard postscript) font

% other packages
\usepackage{datetime} % allows easy formatting of dates, e.g. \formatdate{dd}{mm}{yyyy}
\usepackage{caption} % makes figure captions better, more configurable
\usepackage{enumitem} % allows for custom labels on enumerated lists, e.g. \begin{enumerate}[label=\textbf{(\alph*)}]
\usepackage[squaren]{SIunits} % for nice units formatting e.g. \unit{50}{\kilo\gram}
\usepackage{cancel} % for crossing out terms with \cancel
\usepackage{verbatim} % for verbatim and comment environments
\usepackage{tensor} % for \indices e.g. M\indices{^a_b^{cd}_e}, and \tensor e.g. \tensor[^a_b^c_d]{M}{^a_b^c_d}
\usepackage{feynmp-auto} % for Feynman diagrams. 
\usepackage{pgfplots} % for plotting in tikzpicture environment
\usepackage{commath} % for some nice standardized syntax stuff. \dif, \Dif \od, \pd, \md, \(abs | envert), \(norm | enVert), \(set | cbr), \sbr, \eval, \int(o | c)(o | c), etc
\usepackage{slashed} % provides a command \slashed[1] for Feynman slash notation
%\newcommand{\fslash}[1]{#1\!\!\!/} % feynman slash
%\newcommand{\fsl}[1]{\ensuremath{\mathrlap{\!\not{\phantom{#1}}}#1}}% \fsl{<symbol>}
	% alternative feynman slash

% new commands
\newcommand{\beg}{\begin} % a few letters less for beginning environments
\newenvironment{eqn}{\begin{equation}}{\end{equation}} % a lot fewer letter for equation environment

% rotate stuff
\usepackage{rotating}
	% provides environments for rotating arbitrary objects, e.g. sideways, turn[ang], rotate[ang]
	% also provides macro \turnbox{ang}{stuff}
%\newcommand{\sideways}[1]{\begin{sideways} #1 \end{sideways}} % turn things 90 degrees CCW
%\newcommand{\turn}[2][]{\begin{turn}{#2} #1 \end{turn}} % \turn[ang]{stuff} turns things arbitrary +/- angle

% notational commands
\newcommand{\opname}[1]{\operatorname{#1}} % custom operator names
%\newcommand{\pd}{\partial} % partial differential shortcut
\newcommand{\ket}[1]{\left| #1 \right>} % for Dirac kets
%\newcommand{\ket}[1]{| #1 \rangle}
\newcommand{\bra}[1]{\left< #1 \right|} % for Dirac bras
%\newcommand{\bra}[1]{\langle #1 |}
\newcommand{\braket}[2]{\left< #1 \vphantom{#2} \right| \left. #2 \vphantom{#1} \right>} 
	% for Dirac bra-kets \braket{bra}{ket}
%\newcommand{\braket}[2]{\langle #1 | #2 \rangle} 
\newcommand{\matrixel}[3]{\left< #1 \vphantom{#2#3} \right| #2 \left| #3 \vphantom{#1#2} \right>} 
	% for Dirac matrix elements \matrixel{bra}{op}{ket}
%\newcommand{\matrixel}[3]{\langle #1 | #2 | #3 \rangle} 

%\newcommand{\pd}[2]{\frac{\partial #1}{\partial #2}} % for partial derivatives
%\newcommand{\fd}[2]{\frac{\delta #1}{\delta #2}} % for functional derivatives
\let \vaccent = \v % rename builtin command \v{} to \vaccent{}
%\renewcommand{\v}[1]{\ensuremath{\mathbf{#1}}} % for vectors
\renewcommand{\v}[1]{\ensuremath{\boldsymbol{\mathbf{#1}}}} % for vectors
%\newcommand{\gv}[1]{\ensurmath{\mbox{\boldmath$ #1 $}}} % for vectors of Greek letters
\newcommand{\uv}[1]{\ensuremath{\boldsymbol{\mathbf{\widehat{#1}}}}} % for unit vectors
%\newcommand{\abs}[1]{\left| #1 \right|} % for absolute value ||x||
%\newcommand{\mag}{\abs} % magnitude, just another name for \abs
%\newcommand{\norm}[1]{\left\Vert #1 \right\Vert} % for norm ||v||
\newcommand{\vd}[1]{\v{\dot{#1}}} % for dotted vectors
\newcommand{\vdd}[1]{\v{\ddot{#1}}} % for ddotted vectors
\newcommand{\vddd}[1]{\v{\dddot{#1}}} % for dddotted vectors
\newcommand{\vdddd}[1]{\v{\ddddot{#1}}} % for ddddotted vectors
\newcommand{\avg}[1]{\left< #1 \right>} % for average <x>
\newcommand{\inner}[2]{\left< #1, #2 \right>} % for inner product <x,y>
%\newcommand{\set}[1]{ \left\{ #1 \right\} } % for sets {a,b,c,...}
\newcommand{\tr}{\opname{tr}} % for trace
\newcommand{\Tr}{\opname{Tr}} % for Trace
\newcommand{\rank}{\opname{rank}} % for rank
\let \fancyre = \Re
\let \fancyim = \Im
\newcommand{\Res}{\opname{Res}\limits} % for residue function -- change to put limits on bottom
\renewcommand{\Re}{\opname{Re}}
\renewcommand{\Im}{\opname{Im}}
\renewcommand{\bbar}[1]{\bar{\bar{#1}}} 
	% for barring things twice -- use \cbar or \zbar instead of original \bbar
\newcommand{\bbbar}[1]{\bar{\bbar{#1}}}
\newcommand{\bbbbar}[1]{\bar{\bbbar{#1}}}

\newcommand{\inv}{^{-1}}

% temporary fixes -- commath's versions are bad for powers, like $\dif^3 x$
\renewcommand{\dif}{\mathrm{d}} % \opname{d} better maybe?
\renewcommand{\Dif}{\mathrm{D}}

% notational shortcuts
\newcommand{\bigO}{\mathcal{O}} % big O notation
\let \bigo = \bigO % keep for now, need to update instances in older files
\newcommand{\Lag}{\mathcal{L}} % fancy Lagrangian
\newcommand{\Ham}{\mathcal{H}} % fancy Hamiltonian
\newcommand{\reals}{\mathbb{R}} % real numbers
\newcommand{\complexes}{\mathbb{C}} % complex numbers
\newcommand{\ints}{\mathbb{Z}} % integers
\newcommand{\nats}{\mathbb{N}} % natural numbers
\newcommand{\irrats}{\mathbb{Q}} % irrationals
\newcommand{\quats}{\mathbb{H}} % quaternions (a la Hamilton)
\newcommand{\euclids}{\mathbb{E}} % Euclidean space
\newcommand{\R}{\reals}
\newcommand{\C}{\complexes}
\newcommand{\Z}{\ints}
\newcommand{\Q}{\irrats}
\newcommand{\N}{\nats}
\newcommand{\E}{\euclids}
\newcommand{\RP}{\mathbb{RP}} % real projective space
\newcommand{\CP}{\mathbb{CP}} % complex projective space

% matrix shortcuts!
\newcommand{\pmat}[1]{\begin{pmatrix} #1 \end{pmatrix}}
\newcommand{\bmat}[1]{\begin{bmatrix} #1 \end{bmatrix}}
\newcommand{\Bmat}[1]{\begin{Bmatrix} #1 \end{Bmatrix}}
\newcommand{\vmat}[1]{\begin{vmatrix} #1 \end{vmatrix}}
\newcommand{\Vmat}[1]{\begin{Vmatrix} #1 \end{Vmatrix}}


% more stuff
\newenvironment{enumproblem}{\begin{enumerate}[label=\textbf{(\alph*)}]}{\end{enumerate}}
	% for easily enumerating letters in problems
\newcommand{\grad}[1]{\v{\nabla} #1} % for gradient
\let \divsymb = \div % rename builtin command \div to \divsymb
\renewcommand{\div}[1]{\v{\nabla} \cdot #1} % for divergence
\newcommand{\curl}[1]{\v{\nabla} \times #1} % for curl
\let \baraccent = \= % rename builtin command \= to \baraccent
\renewcommand{\=}[1]{\stackrel{#1}{=}} % for putting numbers above =


% theorem-style environments. note amsthm builtin proof environment: \begin{proof}[title]
% appending [section] resets counter and prepends section number
% use \setcounter{counter}{0} to reset counter
% typical use cases:
% plain: Theorem, Lemma, Corollary, Proposition, Conjecture, Criterion, Algorithm
% definition: Definition, Condition, Problem, Example
% remark: Remark, Note, Notation, Claim, Summary, Acknowledgment, Case, Conclusion
\theoremstyle{plain} % default
\newtheorem{theorem}{Theorem}[section]
\newtheorem{lemma}[theorem]{Lemma}
\newtheorem{corollary}[theorem]{Corollary}
\newtheorem{proposition}[theorem]{Proposition}
\newtheorem{conjecture}[theorem]{Conjecture}
% definition style
\theoremstyle{definition}
\newtheorem{definition}{Definition}
\newtheorem{problem}{Problem}
\newtheorem{exercise}{Exercise}
\newtheorem{example}{Example}
% remark style
\theoremstyle{remark}
\newtheorem{remark}{Remark}
\newtheorem{note}{Note}
\newtheorem{claim}{Claim}
\newtheorem{conclusion}{Conclusion}
% to-do: add problem/subproblem/answer environments for homeworks









%%%%% derivatives


\let \underdot = \d % rename builtin command \d{} to \underdot{}
\let \d = \od % for derivatives

% BUG: derivatives revert to text mode often when in smaller environments in math mode?


% Command for functional derivatives. The first argument denotes the function and the second argument denotes the variable with respect to which the derivative is taken. The optional argument denotes the order of differentiation. The style (text style/display style) is determined automatically
\providecommand{\fd}[3][]{\ensuremath{
\ifinner
\tfrac{\delta{^{#1}}#2}{\delta{#3^{#1}}}
\else
\dfrac{\delta{^{#1}}#2}{\delta{#3^{#1}}}
\fi
}}

% \tfd[2]{f}{k} denotes the second functional derivative of f with respect to k
% The first letter t means "text style"
\providecommand{\tfd}[3][]{\ensuremath{\mathinner{
\tfrac{\delta{^{#1}}#2}{\delta{#3^{#1}}}
}}}
% \dfd[2]{f}{k} denotes the second functional derivative of f with respect to k
% The first letter d means "display style"
\providecommand{\dfd}[3][]{\ensuremath{\mathinner{
\dfrac{\delta{^{#1}}#2}{\delta{#3^{#1}}}
}}}

% mixed functional derivative - analogous to the functional derivative command
% \mfd{F}{5}{x}{2}{y}{3}
\providecommand{\mfd}[6]{\ensuremath{
\ifinner
\tfrac{\delta{^{#2}}#1}{\delta{#3^{#4}}\delta{#5^{#6}}}
\else
\dfrac{\delta{^{#2}}#1}{\delta{#3^{#4}}\delta{#5^{#6}}}
\fi
}}


% Command for thermodynamic (chemistry?) partial derivatives. The first argument denotes the function and the second argument denotes the variable with respect to which the derivative is taken. The optional argument denotes the order of differentiation. The style (text style/display style) is determined automatically
\providecommand{\pdc}[4][]{\ensuremath{
\ifinner
\left( \tfrac{\partial{^{#1}}#2}{\partial{#3^{#1}}} \right)_{#4}
\else
\left( \dfrac{\partial{^{#1}}#2}{\partial{#3^{#1}}} \right)_{#4}
\fi
}}

% \tpd[2]{f}{k} denotes the second thermo partial derivative of f with respect to k
% The first letter t means "text style"
\providecommand{\tpdc}[4][]{\ensuremath{\mathinner{
\left( \tfrac{\partial{^{#1}}#2}{\partial{#3^{#1}}} \right)_{#4}
}}}
% \dpd[2]{f}{k} denotes the second thermo partial derivative of f with respect to k
% The first letter d means "display style"
\providecommand{\dpdc}[4][]{\ensuremath{\mathinner{
\left( \dfrac{\partial{^{#1}}#2}{\partial{#3^{#1}}} \right)_{#4}
}}}


%%%%%%





%%%%%%%%%%%%%%%%%%%
% some templates for various things
\begin{comment}

% template for figures
\begin{figure}
\centering
\includegraphics{myfile.png}
\caption{This is a caption}
\label{fig:myfigure}
\end{figure}

% template for Feynman diagrams using feynmf/feynmp
\begin{fmfgraph*}(40,25)
\fmfleft{em,ep}
\fmf{fermion}{em,Zee,ep}
\fmf{photon,label=$Z$}{Zee,Zff}
\fmf{fermion}{fb,Zff,f}
\fmfright{fb,f}
\fmfdot{Zee,Zff}
\end{fmfgraph*}

% template for drawing plots with pgfplot
\pgfplotsset{compat=1.3,compat/path replacement=1.5.1}
\begin{tikzpicture}
\begin{axis}[
extra x ticks={-2,2},
extra y ticks={-2,2},
extra tick style={grid=major}]
\addplot {x};
\draw (axis cs:0,0) circle[radius=2];
\end{axis}
\end{tikzpicture}

%% find package for easily drawing mapping / algebraic / commutative diagrams..

\end{comment}
%%%%%%%%%%%%%%%%%%%



%%%%% A note on spacing
% 5) \qquad
% 4) \quad
% 3) \thickspace = \;
% 2) \medspace = \:
% 1) \thinspace = \,
% -1) \negthinspace = \!
% -2) \negmedspace
% -3) \negthickspace




% title information
\title{Phys 221A -- Quantum Mechanics -- Lec16}
\author{UCLA, Fall 2014}
\date{\formatdate{26}{11}{2014}} % Activate to display a given date or no date (if empty),
         % otherwise the current date is printed 

\begin{document}
\maketitle


\section{Rotations and Angular Momentum in Quantum Mechanics}

\subsection{Group theory}

Orthogonal transformations are Euclidean transformations which do not change distances, i.e. rotations and reflections. Suppose we have a transformation $R$ taking
\begin{eqn}
\v x \mapsto \v x' = R \cdot \v x.
\end{eqn}
For the transformation to be orthogonal, we must have
\begin{eqn}
\norm{\v x} = \sqrt{x_i x_i} = \norm{\v x'} = \sqrt{R_{ij} x_j R_{ij'} x_{j'}},
\end{eqn}
which implies that
\begin{eqn}
R_{ij} R_{ij'} = \delta_{jj'}.
\end{eqn}
In other words, we have
\begin{eqn}
R^\top R = 1 \qquad
\implies \qquad
R^\top = R\inv,
\end{eqn}
so $R$ is an orthogonal matrix. Thus rotations and reflections are parametrized by orthogonal matrices. The set of all orthogonal matrices in 3 dimensions is a \textit{group} called $O(3)$. 

\begin{definition}
A \textit{group} is a set $G = \set{g_\alpha}$ with an operation $*$ with the following properties. 
\begin{itemize}
\item (Identity) There is an identity $1 \in G$ such that for all $g_\alpha \in G$ we have $g_\alpha * 1 = 1 * g_\alpha = g_\alpha$. 
\item (Closure) For each $g_\alpha, g_\beta \in G$, we have $g_\alpha * g_\beta \in G$.
\item (Invertibility) For each $g_\alpha \in G$ we have an inverse $g_\alpha\inv \in G$ such that $g_\alpha * g_\alpha\inv = g_\alpha\inv * g_\alpha = 1$.
\item (Associativity) For $g_\alpha, g_\beta, g_\gamma \in G$, we have $(g_\alpha * g_\beta) * g_\gamma = g_\alpha * (g_\beta * g_\gamma)$.
\end{itemize}
In general, $g_\alpha * g_\beta \neq g_\beta * g_\alpha$, i.e. group elements must not necessarily commute. If all elements commute, the group is called \textit{Abelian}. A subset of a group $G$ that is itself a group is called a \textit{subgroup} of $G$. 
\end{definition}

Notice that for an orthogonal matrix $R$, we have $\det R = \pm 1$. The subgroup of $O(3)$ with determinant 1 is the group of proper rotations $SO(3)$. If $R \in O(3)$ and $\det R = -1$ then we call $R$ an improper rotation. If furthermore $R^2 = 1$, then $R$ is a reflection. Note that in three dimensions, parity is an improper rotation, but in two dimensions, parity is a proper rotation---it's just a rotation by $\pi$. In general, parity is a proper rotation in even dimensions and improper in odd dimensions. 

For now we will consider rotations in three dimensions. A rotation $R_z (\phi)$ about the $z$-axis can be written
\begin{eqn}
R_z (\phi) = 
\begin{pmatrix}
\cos\phi & -\sin\phi & 0 \\
\sin\phi & \cos\phi & 0 \\
0 & 0 & 1
\end{pmatrix}.
\end{eqn}
For an infinitesimal angle $\epsilon$ we have
\begin{eqn}
R_z (\epsilon) = 
\begin{pmatrix}
1 - \epsilon^2 / 2 & -\epsilon & 0 \\
\epsilon & 1 - \epsilon^2 / 2 & 0 \\
0 & 0 & 1
\end{pmatrix}
+ \bigO(\epsilon^2).
\end{eqn}
Similarly for the other axes the infinitesimal rotations are (to quadratic order)
\begin{eqn}
R_x (\epsilon) = 
\begin{pmatrix}
1 & 0 & 0 \\
0 & 1 - \epsilon^2 / 2 & -\epsilon \\
0 & \epsilon & 1 - \epsilon^2 / 2 \\
\end{pmatrix}, \qquad
R_y (\epsilon) =
\begin{pmatrix}
1 - \epsilon^2 / 2 & 0 & \epsilon \\
0 & 1 & 0 \\
-\epsilon & 0 & 1 - \epsilon^2 / 2 \\
\end{pmatrix}.
\end{eqn}
If we look at the commutators of the rotations about the axes, we find
\begin{eqn}
[R_x(\epsilon), R_y(\epsilon)] = R_z(\epsilon^2) - 1 = 
\begin{pmatrix}
0 & -\epsilon^2 & 0 \\
\epsilon^2 & 0 & 0 \\
0 & 0 & 0
\end{pmatrix}.
\end{eqn}
One can generalize to other permutations of the elementary rotations by cyclic permutation of $(xyz)$. These commutator relations will be useful in our applications to quantum mechanics. 


\subsection{Angular momentum in QM}

Recall that the infinitesimal evolution operator can be written
\begin{eqn}
U(\epsilon) = 1 - \frac{i}{\hbar} \epsilon H + \bigO(\epsilon^2),
\end{eqn}
so we call $H$ the \textit{generator of time evolution}. From now on we will assume that $H$ is time-independent. We can recover the full time evolution by repeatedly composing infinitesimal time evolutions
\begin{eqn}
U(t) = \lim_{n \rightarrow \infty} \left[ U(t/n) \right]^n.
\end{eqn}
Now, recall the identity
\begin{eqn}
e^\phi = \lim_{n \rightarrow \infty} \left( 1 + \frac{\phi}{n} \right)^n,
\end{eqn}
which also works for operators so that we have
\begin{eqn}
U(t) = \lim_{n \rightarrow \infty} \left( 1 - \frac{i}{\hbar} \, \frac{t}{n} H \right)^n = e^{i t H / \hbar}.
\end{eqn}
Thus we have
\begin{eqn}
H = i \hbar \partial_t U(t).
\end{eqn}

Similarly, momentum is the generator of translations,
\begin{eqn}
T (\v \epsilon) = 1 - \frac{i}{\hbar} \v \epsilon \cdot \v p = 1 - \v \epsilon \cdot \v \nabla,
\end{eqn}
since $\v p = - i \hbar \v \nabla$. Thus similarly we have
\begin{eqn}
\v p = i \hbar \v \nabla_{\v x} T(\v x).
\end{eqn}
We can now define the angular momentum operator $\v J$ in quantum mechanics in a similar way, as the generator of rotations
\begin{eqn}
\v J = i \hbar \v \nabla_{\v \phi} R(\v \phi)
\end{eqn}
where $R(\v \phi)$ is a rotation about $\uv \phi$ by angle $\abs{\v \phi}$. 

Quantum mechanically, the general state of a particle in some rotation state is described by a ket
\begin{eqn}
\ket{\alpha} = \ket{\sigma} \otimes \ket{\psi},
\end{eqn}
where $\ket{\psi}$ is the usual $L_2$ Euclidean wavefunction giving dynamics in real space, while $\ket{\sigma}$ is a spin ket, or spinor, giving us dynamics in ``spin space''. In the absence of the spinor component we just have the Euclidean wavefunction. Furthermore, since $\v L = \v r \times \v p$ and $\v p = -i\hbar \v \nabla$, an infinitesimal rotation can be written
\begin{eqn}
R(\v \epsilon) = 1 - \frac{i}{\hbar} \v \epsilon \cdot \v L = 1 - \v \epsilon \cdot (\v r \times \v \nabla).
\end{eqn}
In general, when we include the spin structure the angular momentum $\v J$ is decomposed
\begin{eqn}
\v J = \v L + \v S,
\end{eqn}
where $\v L$ is the ``orbital'' angular momentum and $\v S$ is the ``spin'' angular momentum. 

The only remaining task is to construct the (irreducible) representations of $SO(3)$. We define a \textit{representation} $D(g)$ of a group element $g \in G$ if $D$ is a homomorphism, i.e. given $g, h \in G$ we have
\begin{eqn}
D(g*h) = D(g) * D(h).
\end{eqn}
In the case of spin rotations, given a spin ket $\ket{\sigma}$ we have under rotations
\begin{eqn}
\ket{\sigma} \rightarrow D(R) \ket{\sigma}
\end{eqn}
It turns out that the irreducible representations of $D(R)$ are $N \times N$ matrices with $N = 2s + 1$ where $s = 0, \frac{1}{2}, 1, \frac{3}{2}, \dots$ is the spin of the particle. In terms of elementary particles, bosons have integer-valued spin while fermions have half-integer-valued spin. 








\end{document}
