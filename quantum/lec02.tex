% declare document class and geometry
\documentclass[12pt]{article} % use larger type; default would be 10pt
\usepackage[margin=1in]{geometry} % handle page geometry

% import packages and commands
% standard packages
\usepackage{graphicx} % support the \includegraphics command and options
\usepackage{amsmath} % for nice math commands and environments

% font packages
\usepackage{amssymb} % for \mathbb, \mathfrak fonts
\usepackage{mathrsfs} % for \mathscr font
\DeclareMathAlphabet{\mathpzc}{OT1}{pzc}{m}{it} % defines \mathpzc for Zapf Chancery (standard postscript) font

% other packages
\usepackage{datetime} % allows easy formatting of dates, e.g. \formatdate{dd}{mm}{yyyy}
\usepackage{caption} % makes figure captions better, more configurable
\usepackage{enumitem} % allows for custom labels on enumerated lists, e.g. \begin{enumerate}[label=\textbf{(\alph*)}]
\usepackage[squaren]{SIunits} % for nice units formatting e.g. \unit{50}{\kilo\gram}
\usepackage{cancel} % for crossing out terms with \cancel
\usepackage{verbatim} % for verbatim and comment environments
\usepackage{tensor} % for \indices e.g. M\indices{^a_b^{cd}_e}, and \tensor e.g. \tensor[^a_b^c_d]{M}{^a_b^c_d}
\usepackage{feynmp-auto} % for Feynman diagrams. 
\usepackage{pgfplots} % for plotting in tikzpicture environment

% new commands
\newcommand{\beg}{\begin} % a few letters less for beginning environments
\newenvironment{eqn}{\begin{equation}}{\end{equation}} % a lot fewer letter for equation environment

% notational commands
\newcommand{\opname}[1]{\operatorname{#1}} % custom operator names
\newcommand{\fslash}[1]{#1\!\!\!/} % feynman slash
\newcommand{\pd}{\partial} % partial differential shortcut
\newcommand{\ket}[1]{\left| #1 \right>} % for Dirac kets
\newcommand{\bra}[1]{\left< #1 \right|} % for Dirac bras
\newcommand{\braket}[2]{\left< #1 \vphantom{#2} \right| 
	\left. #2 \vphantom{#1} \right>} % for Dirac brackets
%\let\underdot=\d % rename builtin command \d{} to \underdot{}
%\renewcommand{\d}[2]{\frac{d #1}{d #2}} % for derivatives
%\newcommand{\pd}[2]{\frac{\partial #1}{\partial #2}} % for partial derivatives
%\newcommand{\fd}[2]{\frac{\delta #1}{\delta #2}} % for functional derivatives
\let\vaccent=\v % rename builtin command \v{} to \vaccent{}
%\renewcommand{\v}[1]{\ensuremath{\mathbf{#1}}} % for vectors
\renewcommand{\v}[1]{\ensuremath{\boldsymbol{\mathbf{#1}}}} % for vectors
%\newcommand{\gv}[1]{\ensurmath{\mbox{\boldmath$ #1 $}}} % for vectors of Greek letters
\newcommand{\uv}[1]{\ensuremath{\boldsymbol{\mathbf{\widehat{#1}}}}} % for unit vectors
\newcommand{\abs}[1]{\left| #1 \right|} % for absolute value ||x||
%\newcommand{\mag}{\abs} % magnitude, just another name for \abs
\newcommand{\norm}[1]{\left\Vert #1 \right\Vert} % for norm ||v||
\newcommand{\avg}[1]{\left< #1 \right>} % for average <x>
\newcommand{\inner}[2]{\left< #1, #2 \right>} % for inner product <x,y>
\newcommand{\set}[1]{ \left\{ #1 \right\} } % for sets {a,b,c,...}
\newcommand{\tr}{\opname{tr}} % for trace
\newcommand{\Tr}{\opname{Tr}} % for Trace

% notational shortcuts
\newcommand{\reals}{\mathbb{R}} % real numbers
\newcommand{\complexes}{\mathbb{C}} % complex numbers
\newcommand{\nats}{\mathbb{N}} % natural numbers
\newcommand{\irrats}{\mathbb{Q}} % irrationals
\newcommand{\quats}{\mathbb{H}} % quaternions (a la Hamilton)
\newcommand{\euclids}{\mathbb{E}} % Euclidean space
\newcommand{\bigo}{\mathcal{O}} % big O notation
\newcommand{\Lag}{\mathcal{L}} % fancy Lagrangian
\newcommand{\Ham}{\mathcal{H}} % fancy Hamiltonian





%%%%%%%%%%%%%%%%%%%
% some templates for various things
\begin{comment}

% template for figures
\begin{figure}
\centering
\includegraphics{myfile.png}
\caption{This is a caption}
\label{fig:myfigure}
\end{figure}

% template for Feynman diagrams using feynmf/feynmp
\begin{fmfgraph*}(40,25)
\fmfleft{em,ep}
\fmf{fermion}{em,Zee,ep}
\fmf{photon,label=$Z$}{Zee,Zff}
\fmf{fermion}{fb,Zff,f}
\fmfright{fb,f}
\fmfdot{Zee,Zff}
\end{fmfgraph*}

% template for drawing plots with pgfplot
\pgfplotsset{compat=1.3,compat/path replacement=1.5.1}
\begin{tikzpicture}
\begin{axis}[
extra x ticks={-2,2},
extra y ticks={-2,2},
extra tick style={grid=major}]
\addplot {x};
\draw (axis cs:0,0) circle[radius=2];
\end{axis}
\end{tikzpicture}

\end{comment}
%%%%%%%%%%%%%%%%%%%


\title{Phys 221A -- Quantum Mechanics -- Lec02}
\author{UCLA, Fall 2014}
\date{\formatdate{08}{10}{2014}} % Activate to display a given date or no date (if empty),
         % otherwise the current date is printed 

\begin{document}
\maketitle


\section{Introduction}

Preliminary notes
\begin{itemize}
\item TA (Shahriar) office hours changed: Wednesday 3-4 PM in Knudsen 3-111.
\end{itemize}

\subsection{More on Hilbert spaces}

From now on we will be living in Hilbert space $G$ with vectors, for example $\psi$ or $\phi$. Recall that in the bra-ket notation we have $\psi = | \psi \rangle$ in the Hilbert space, a ``ket''. The inner product is given by
\begin{equation}
\begin{matrix}
G \times G & \rightarrow & \mathbb{C} \\
\ket{\phi}, \ket{\psi} & \mapsto & \braket{\phi}{\psi}. 
\end{matrix}
\end{equation}
We write elements in the dual space as $\bra{\psi}$, a ``bra''. This is a linear map taking $G \rightarrow \mathbb{C}$ and mapping $\ket{\phi} \mapsto \braket{\psi}{\phi}$. 

Consider an operator $O : G \rightarrow G$ mapping $\ket{\psi} \mapsto O \ket{\psi}$. For example, $O = \ket{\phi} \bra{\psi}$ for some $\phi, \psi \in G$. Then 
\begin{equation}
O \ket{\xi} = (\ket{\phi} \bra{\psi}) \ket{\xi} = \ket{\phi} \braket{\psi}{\xi}.
\end{equation}
Given operators $A, B, C$ we have 
\begin{equation}
(AB) \ket{\psi} = A (B \ket{\psi}), \qquad ABC = A(BC) = (AB)C.
\end{equation}
Recall the closure relation
\begin{equation}
1 = \sum_i \ket{\alpha_i} \bra{\alpha_i}
\end{equation}
where $\set{ \ket{\alpha_i} }$ forms an orthonormal basis (so $\braket{\alpha_i}{\alpha_j} = \delta_{ij}$).

Given an operator $O$, we have that $O$ is linear iff 
\begin{equation}
O (c_1 \ket{\psi_1} + c_2 \ket{\psi_2}) = c_1 O \ket{\psi_1} + c_2 O \ket{\psi_2}
\end{equation}
for all $c_i \in \mathbb{C}$, $\ket{\psi_i} \in G$. The operator $O^\dagger$ is the Hermitian conjugate or adjoint of $O$ iff 
\begin{equation}
\braket{\psi}{O\phi} = \braket{O^\dagger \psi}{\phi}
\end{equation}
for all $\psi, \phi \in G$. If $O^\dagger$ is the adjoint then $(O^\dagger)^\dagger = O$ and furthermore $\braket{O\phi}{\psi} = \braket{\phi}{O^\dagger \psi}$. 

If $O^\dagger = O$ then $O$ is Hermitian or self-adjoint. If $UU^\dagger = 1$ (i.e. $U^\dagger = U^{-1}$) then $U$ is called unitary. Note that we have 
\begin{equation}
\bra{O\psi} = \bra{\psi} O^\dagger,
\end{equation}
or in other words
\begin{equation}
\braket{O\psi}{\phi} = \bra{\psi} O^\dagger \ket{\phi} = \braket{\psi}{O^\dagger \phi}.
\end{equation}

A physical observable, i.e. a dynamical variable, must be represented by a linear operator $O$. Given a physical state $\psi$, where $\braket{\psi}{\psi} = 1$, the ``expectation value'' of $O$ in the state $\ket{\psi}$ is given by
\begin{equation}
\avg{O} = \braket{\psi}{O\psi}.
\end{equation}
Note that if $O$ is Hermitian, i.e. $O = O^\dagger$, then $\avg{O} \in \mathbb{R}$ for any $\psi$. 


\subsection{Representation in a basis}

In an orthonormal basis $\ket{\alpha_i}$ we have 
\begin{equation}
\ket{\psi} = \sum_i \ket{\alpha_i} \braket{\alpha_i}{\psi} = \sum_i c_i \ket{\alpha_i} 
\end{equation}
so we can write
\begin{align}
O \ket{\psi} &= O \sum_i c_i \ket{\alpha_i} = \sum_i c_i O \ket{\alpha_i} \\
	&= \sum_{ij} c_i \ket{\alpha_j} \bra{\alpha_j} O \ket{\alpha_i}.
\end{align}
Thus we can write $O\ket{\psi}$ in the basis as $\sum_j b_j \ket{\alpha_j}$ where $b_j = \sum_i O_{ji} c_i$ and
\begin{equation}
O_{ij} = \bra{\alpha_i} O \ket{\alpha_j}
\end{equation}
is called the matrix element. 

There is a one-to-one map from vectors $\psi$ to column vectors $\v{c} = (c_1, \dots, c_n)^\top$ (where $n = \opname{dim} G$) such that $\psi = \sum_i c_i \ket{\alpha_i}$, as well as a map from operators $O$ to $n \times n$ matrices $\hat{O} = (O_{ij})$. Thus we can write $G \cong \mathbb{C}^n$ (note that $n$ is not necessarily finite). Note that given a vector $\phi$ represented by $\v{b}$, we can write $\braket{\phi}{\psi} = \v{b}^\dagger \cdot \v{c}$. Furthermore if $\ket{\phi} = O \ket{\psi}$ then $\v{b} = \hat{O} \v{c}$. 

A Hermitian operator $O$ corresponds to a Hermitian matrix $\hat{O}$, and similarly a unitary operator $U$ corresponds to a unitary matrix $\hat{U}$. Thus given Hermitian $O$ and unitary $U$ we have $O_{ij} = O_{ji}^*$ and $\hat{U} \hat{U}^\dagger = \hat{1}$. Given two orthonormal bases $\set{ \ket{\alpha_i} }$ and $\set{ \ket{\alpha'_i} }$ we can always write $\hat{O}' = \hat{U}^\dagger \hat{O} \hat{U}$ where $U_{ij} = \braket{\alpha_i}{\alpha_j'}$ and happens to be unitary. We can show this last statement:
\begin{equation}
(\hat{U} \hat{U}^\dagger)_{ij} = \sum_k U_{ik} U_{jk}^* = \sum_k \braket{\alpha_i}{\alpha'_k} \braket{\alpha'_k}{\alpha_j} = \braket{\alpha_i}{\alpha_j} = \delta_{ij}.
\end{equation}

A little more on operators. We define
\begin{align}
[A,B] &= AB - BA, \qquad \text{the commutator, and} \\
\set{ A,B } &= AB + BA, \qquad \text{the anti-commutator}.
\end{align}
Then we have
\begin{align}
AB &= \frac{1}{2} [A,B] + \frac{1}{2} \set{A,B} \\
	&= BA = -\frac{1}{2} [A,B] + \frac{1}{2} \set{A,B}
\end{align}
Notice that $\set{A,B}$ is Hermitian and $[A,B]$ is anti-Hermitian ($O^\dagger = -O$). Furthermore, if $A$ and $B$ are Hermitian then 
\begin{equation}
(AB)^\dagger = B^\dagger A^\dagger = BA \neq AB.
\end{equation}
Finally we can split any operator $O$ into a Hermitian part $\zeta_+$ and an anti-Hermitian part $\zeta_-$:
\begin{equation}
O = \frac{1}{2} (\zeta_+ + \zeta_i), \quad \zeta_+ = O + O^\dagger, \quad \zeta_- = O - O^\dagger.
\end{equation}


\subsection{Onward to physics}

The eigenvectors of a Hermitian operator yield a complete orthonormal basis. If $A$ and $B$ are Hermitian operators and $[A,B] = 0$ then the operators yield a complete set of simultaneous eigenvectors (which form an orthonormal basis). When this happens we call $A, B$ compatible observables. 

In the hydrogen atom we have quantum numbers:
\begin{itemize}
\item principle quantum number $n = 1, 2, \dots$ where $E \sim -1/n^2$, eigenvalues of the Hamiltonian $H$,
\item angular momentum quantum number $\ell = 0, 1, \dots, n-1$, eigenvalues of $L^2$,
\item magnetic quantum number $m = -\ell, \dots, \ell$, eigenvalues of $L_z$,
\item spin quantum number $s = \pm 1/2$, eigenvalues of $S_z$. 
\end{itemize}
We can show that 
\begin{equation}
[H, L^2] = [L^2, L_z] = [H, L_z] = 0
\end{equation}
so that $H, L^2, L_z$, and $S_z$ are compatible observables uniquely identified by their quantum numbers $(n,\ell,m,s)$. 

Generally, a complete set of compatible observables whose eigenvalues, or quantum numbers, uniquely label their common eigenvectors (resolving all degeneracies). Thus each $\ket{\alpha_i}$ in the corresponding orthonormal basis is labelled by the set of quantum numbers. Typically this complete set is provided by $H$ and corresponding symmetry generators. 







\begin{comment}
\begin{figure}
\centering
\includegraphics{3a.pdf}
\caption{Half the diagrams for photon-photon scattering.}
\label{fig:3a}
\end{figure}
\end{comment}



\end{document}
