% declare document class and geometry
\documentclass[12pt]{article} % use larger type; default would be 10pt
\usepackage[margin=1in]{geometry} % handle page geometry

% import packages and commands
% standard packages
\usepackage{graphicx} % support the \includegraphics command and options
\usepackage{amsmath} % for nice math commands and environments

% font packages
\usepackage{amssymb} % for \mathbb, \mathfrak fonts
\usepackage{mathrsfs} % for \mathscr font
\DeclareMathAlphabet{\mathpzc}{OT1}{pzc}{m}{it} % defines \mathpzc for Zapf Chancery (standard postscript) font

% other packages
\usepackage{datetime} % allows easy formatting of dates, e.g. \formatdate{dd}{mm}{yyyy}
\usepackage{caption} % makes figure captions better, more configurable
\usepackage{enumitem} % allows for custom labels on enumerated lists, e.g. \begin{enumerate}[label=\textbf{(\alph*)}]
\usepackage[squaren]{SIunits} % for nice units formatting e.g. \unit{50}{\kilo\gram}
\usepackage{cancel} % for crossing out terms with \cancel
\usepackage{verbatim} % for verbatim and comment environments
\usepackage{tensor} % for \indices e.g. M\indices{^a_b^{cd}_e}, and \tensor e.g. \tensor[^a_b^c_d]{M}{^a_b^c_d}
\usepackage{feynmp-auto} % for Feynman diagrams. 
\usepackage{pgfplots} % for plotting in tikzpicture environment

% new commands
\newcommand{\beg}{\begin} % a few letters less for beginning environments
\newenvironment{eqn}{\begin{equation}}{\end{equation}} % a lot fewer letter for equation environment

% notational commands
\newcommand{\opname}[1]{\operatorname{#1}} % custom operator names
\newcommand{\fslash}[1]{#1\!\!\!/} % feynman slash
\newcommand{\pd}{\partial} % partial differential shortcut
\newcommand{\ket}[1]{\left| #1 \right>} % for Dirac kets
\newcommand{\bra}[1]{\left< #1 \right|} % for Dirac bras
\newcommand{\braket}[2]{\left< #1 \vphantom{#2} \right| 
	\left. #2 \vphantom{#1} \right>} % for Dirac brackets
%\let\underdot=\d % rename builtin command \d{} to \underdot{}
%\renewcommand{\d}[2]{\frac{d #1}{d #2}} % for derivatives
%\newcommand{\pd}[2]{\frac{\partial #1}{\partial #2}} % for partial derivatives
%\newcommand{\fd}[2]{\frac{\delta #1}{\delta #2}} % for functional derivatives
\let\vaccent=\v % rename builtin command \v{} to \vaccent{}
%\renewcommand{\v}[1]{\ensuremath{\mathbf{#1}}} % for vectors
\renewcommand{\v}[1]{\ensuremath{\boldsymbol{\mathbf{#1}}}} % for vectors
%\newcommand{\gv}[1]{\ensurmath{\mbox{\boldmath$ #1 $}}} % for vectors of Greek letters
\newcommand{\uv}[1]{\ensuremath{\boldsymbol{\mathbf{\widehat{#1}}}}} % for unit vectors
\newcommand{\abs}[1]{\left| #1 \right|} % for absolute value ||x||
%\newcommand{\mag}{\abs} % magnitude, just another name for \abs
\newcommand{\norm}[1]{\left\Vert #1 \right\Vert} % for norm ||v||
\newcommand{\avg}[1]{\left< #1 \right>} % for average <x>
\newcommand{\inner}[2]{\left< #1, #2 \right>} % for inner product <x,y>
\newcommand{\set}[1]{ \left\{ #1 \right\} } % for sets {a,b,c,...}
\newcommand{\tr}{\opname{tr}} % for trace
\newcommand{\Tr}{\opname{Tr}} % for Trace

% notational shortcuts
\newcommand{\reals}{\mathbb{R}} % real numbers
\newcommand{\complexes}{\mathbb{C}} % complex numbers
\newcommand{\nats}{\mathbb{N}} % natural numbers
\newcommand{\irrats}{\mathbb{Q}} % irrationals
\newcommand{\quats}{\mathbb{H}} % quaternions (a la Hamilton)
\newcommand{\euclids}{\mathbb{E}} % Euclidean space
\newcommand{\bigo}{\mathcal{O}} % big O notation
\newcommand{\Lag}{\mathcal{L}} % fancy Lagrangian
\newcommand{\Ham}{\mathcal{H}} % fancy Hamiltonian





%%%%%%%%%%%%%%%%%%%
% some templates for various things
\begin{comment}

% template for figures
\begin{figure}
\centering
\includegraphics{myfile.png}
\caption{This is a caption}
\label{fig:myfigure}
\end{figure}

% template for Feynman diagrams using feynmf/feynmp
\begin{fmfgraph*}(40,25)
\fmfleft{em,ep}
\fmf{fermion}{em,Zee,ep}
\fmf{photon,label=$Z$}{Zee,Zff}
\fmf{fermion}{fb,Zff,f}
\fmfright{fb,f}
\fmfdot{Zee,Zff}
\end{fmfgraph*}

% template for drawing plots with pgfplot
\pgfplotsset{compat=1.3,compat/path replacement=1.5.1}
\begin{tikzpicture}
\begin{axis}[
extra x ticks={-2,2},
extra y ticks={-2,2},
extra tick style={grid=major}]
\addplot {x};
\draw (axis cs:0,0) circle[radius=2];
\end{axis}
\end{tikzpicture}

\end{comment}
%%%%%%%%%%%%%%%%%%%



\title{Phys 221A -- Quantum Mechanics -- Lec01}
\author{UCLA, Fall 2014}
\date{\formatdate{06}{10}{2014}} % Activate to display a given date or no date (if empty),
         % otherwise the current date is printed 

\begin{document}
\maketitle


\section{Introduction}

Basic info
\begin{itemize}
\item Professor: Yaroslav Tserkovnyak, condensed matter (experimentalist?). Email: yaroslav@physics.ucla.edu. Office hours: Thursday 11-12, Knudsen 6-137C. No midterm. Grade: 35\% HW, 65\% final. Homeworks will be assigned Wednesdays, due Wednesday the next week. 
\item TA: Shahriar, office hours Tuesday 1-2 at Knudsen 3-111. 
\end{itemize}

What the course is \textbf{not}:
\begin{itemize}
\item Will not motivate QM. You should have motivation by now.
\item Will not be overly mathematically rigorous. Only need a certain level of physical rigor. 
\item Will not give historical or philosophical perspective. 
\end{itemize}

What the course will cover:
\begin{enumerate}
\item Intro
\begin{enumerate}
\item Hilbert spaces
\item Quantum numbers
\item \underline{Uncertainty principle}
\item Correspondence principle --- establishes link between QM and classical mechanics
\end{enumerate}
\item Quantum dynamics
\begin{enumerate}
\item Schroedinger vs Heisenberg picture
\item Feynman path integrals --- establishes connection with principle of least action, Lagrangian mechanics
\item Gauge transformations --- establishes connection with E\&M
\end{enumerate}
\item Rotations
\begin{enumerate}
\item Spin and orbital angular momentum
\item Addition of angular momentum
\item Bell inequalities --- relatively recent. Show that there are measurements that prove classical description is incomplete. 
\end{enumerate}
\end{enumerate}

221B will cover symmetries, approximations, scattering theory. 221C will cover applications with many (identical) particles and fundamentals needed for quantum field theory. 


\subsection{Hilbert spaces}

A Hilbert space $G$ is a complete inner product space. One example is Euclidean space $\mathbb{E}^n$, that is, $\mathbb{R}^n$ with the standard inner product $\braket{x}{y} = \v{x} \cdot \v{y} = \sum_{i=1}^n x_i y_i$. Of course $\mathbb{R}^n$ itself is a vector space. We will define what completeness means later. 

We should define the inner product. The inner product $\braket{x}{y}$ between vectors $x$ and $y$ in $G$ is a complex number such that: 
\begin{enumerate}
\item $\braket{x}{y} = \braket{y}{x}^*$,
\item Linearity: $\braket{x}{\alpha_1 y_1 + \alpha_2 y_2} = \alpha_1 \braket{x}{y_1} + \alpha_2 \braket{x}{y_2}$ for any scalars $\alpha_1, \alpha_2 \in \mathbb{C}$, and
\item Positivity: $\braket{x}{x} \geq 0$ with equality iff $x = 0$.
\end{enumerate}

Define the distance 
\begin{equation}
d(x, y) = d(y, x) = \norm{x - y} = \sqrt{\braket{x - y}{x - y}}.
\end{equation}
Need to prove this is a good distance function. In particular, we need the triangle inequality: $d(x, z) \leq d(x, y) + d(y, z)$. Mathematicians would call this a metric space. Can prove the triangle inequality with the Cauchy-Schwartz inequality:
\begin{equation}
\abs{\braket{x}{y}} \leq \norm{x} \cdot \norm{y}.
\end{equation}
We can prove this as follows. Define 
\begin{equation}
z = x - y \frac{\braket{y}{x}}{\braket{y}{y}},
\end{equation}
then 
\begin{equation}
x = z + y \frac{\braket{y}{x}}{\braket{y}{y}}
\end{equation}
so $\norm{x}^2 = \norm{z}^2 + \norm{y}^2 (\dots)^2$ by Pythagorean theorem, thus
\begin{equation}
\norm{x}^2 \norm{y}^2 = \norm{z}^2 \norm{y}^2 + \abs{\braket{y}{x}}^2
\end{equation}
and we see equality occurs iff $z = 0$ i.e. $x \parallel y$

Now we can define completeness of a metric space $G$. Define a Cauchy sequence $x_i \in G$, $i = 1,2,\dots$ as a sequence such that for any $\delta > 0$ there exists an integer $N > 0$ so that $\norm{x_n - x_m} < \delta$ for all $m, n > N$. The space $G$ is complete if every Cauchy sequence converges. Good, now we've defined a Hilbert space. 

Now, the universe itself is a Hilbert space. Or rather, physical realizations of the universe are elements of a Hilbert space. A quantum mechanical state $\psi$ is represented by a vector in a Hilbert space. We will be interested in the preparation, measurement, and evolution of this state. 

As $\psi(t)$ evolves, it must remain in $G$. Furthermore, the physical state $\psi$ must be properly normalized: $\braket{\psi}{\psi} = 1$ at all times. Dirac's bracket notation provides us with a good way of dealing with elements of the Hilbert space and the dual space. 
\begin{itemize}
\item $\ket{\psi} \in G$ is the ket, the representation in the Hilbert space
\item $\bra{\psi} \in G$ is the bra, the representation in the dual space. The bra is a linear map $\bra{\psi} : G \rightarrow \mathbb{C}$ mapping kets $\ket{\phi} \mapsto \braket{\psi}{\phi}$. 
\end{itemize}
For each Hilbert space $G$ there is a dual (Hilbert) space $G^*$. Bras, the elements of $G^*$, are defined by their action on the kets in $G$. Note that we have
\begin{equation}
\braket{c_1 \psi_1 + c_2 \psi_2}{\phi} = c_1^* \braket{\psi_1}{\phi} + c_2^* \braket{\psi_2}{\phi},
\end{equation}
so that
\begin{equation}
\bra{c_1 \psi_1 + c_2 \psi_2} = c_1^* \bra{\psi_1} + c_2^* \bra{\psi_2}. 
\end{equation}

A Hilbert space can be spanned by an orthonormal basis $\ket{\alpha_i}$, $i = 1,2, \dots$, i.e. $\braket{\alpha_i}{\alpha_j} = \delta_{ij}$. We can find an orthonormal basis via the Gram-Schmidt method as follows. Suppose we have a complete set of linearly independent states $\ket{u_i}$. Define a normal vector $e_1 = u_1 / \norm{u_1}$. Next define $v_2 = u_2 - P_{e_1}(u_2)$ where we use the projection operator $P_e(v) = \ket{e} \braket{e}{v}$, and finally $e_2 = v_2 / \norm{v_2}$. We can see that $e_1$ and $e_2$ form an orthonormal set so we continue this process ad infinitum to obtain the full orthonormal basis, i.e. $v_{i+1} = u_{i+1} - P_{e_i}(u_{i+1})$ and $e_{i+1} = v_{i+1} / \norm{v_{i+1}}$ for all $i > 1$. 

Any state $\psi$ can be written in the basis: $\ket{\psi} = \sum_i c_i \ket{\alpha_i}$. Note that the coefficients can be obtained by taking the inner product with the corresponding orthonormal basis vector: $\braket{\alpha_i}{\psi} = \sum_j c_j \delta_{ij} = c_i$. Since $c_i = \braket{\alpha_i}{\psi}$, we can write $\ket{\psi} = \sum_i \ket{\alpha_i} \braket{\alpha_i}{\psi}$, thus we find
\begin{equation}
\sum_i \ket{\alpha_i} \bra{\alpha_i} = 1.
\end{equation}
We call this the closure relation. 



\begin{comment}
\begin{figure}
\centering
\includegraphics{3a.pdf}
\caption{Half the diagrams for photon-photon scattering.}
\label{fig:3a}
\end{figure}
\end{comment}



\end{document}
