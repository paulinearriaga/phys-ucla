% declare document class and geometry
\documentclass[12pt]{article} % use larger type; default would be 10pt
\usepackage[margin=1in]{geometry} % handle page geometry

% import packages and commands
% standard packages
\usepackage{graphicx} % support the \includegraphics command and options
\usepackage{amsmath} % for nice math commands and environments

% font packages
\usepackage{amssymb} % for \mathbb, \mathfrak fonts
\usepackage{mathrsfs} % for \mathscr font
\DeclareMathAlphabet{\mathpzc}{OT1}{pzc}{m}{it} % defines \mathpzc for Zapf Chancery (standard postscript) font

% other packages
\usepackage{datetime} % allows easy formatting of dates, e.g. \formatdate{dd}{mm}{yyyy}
\usepackage{caption} % makes figure captions better, more configurable
\usepackage{enumitem} % allows for custom labels on enumerated lists, e.g. \begin{enumerate}[label=\textbf{(\alph*)}]
\usepackage[squaren]{SIunits} % for nice units formatting e.g. \unit{50}{\kilo\gram}
\usepackage{cancel} % for crossing out terms with \cancel
\usepackage{verbatim} % for verbatim and comment environments
\usepackage{tensor} % for \indices e.g. M\indices{^a_b^{cd}_e}, and \tensor e.g. \tensor[^a_b^c_d]{M}{^a_b^c_d}
\usepackage{feynmp-auto} % for Feynman diagrams. 
\usepackage{pgfplots} % for plotting in tikzpicture environment

% new commands
\newcommand{\beg}{\begin} % a few letters less for beginning environments
\newenvironment{eqn}{\begin{equation}}{\end{equation}} % a lot fewer letter for equation environment

% notational commands
\newcommand{\opname}[1]{\operatorname{#1}} % custom operator names
\newcommand{\fslash}[1]{#1\!\!\!/} % feynman slash
\newcommand{\pd}{\partial} % partial differential shortcut
\newcommand{\ket}[1]{\left| #1 \right>} % for Dirac kets
\newcommand{\bra}[1]{\left< #1 \right|} % for Dirac bras
\newcommand{\braket}[2]{\left< #1 \vphantom{#2} \right| 
	\left. #2 \vphantom{#1} \right>} % for Dirac brackets
%\let\underdot=\d % rename builtin command \d{} to \underdot{}
%\renewcommand{\d}[2]{\frac{d #1}{d #2}} % for derivatives
%\newcommand{\pd}[2]{\frac{\partial #1}{\partial #2}} % for partial derivatives
%\newcommand{\fd}[2]{\frac{\delta #1}{\delta #2}} % for functional derivatives
\let\vaccent=\v % rename builtin command \v{} to \vaccent{}
%\renewcommand{\v}[1]{\ensuremath{\mathbf{#1}}} % for vectors
\renewcommand{\v}[1]{\ensuremath{\boldsymbol{\mathbf{#1}}}} % for vectors
%\newcommand{\gv}[1]{\ensurmath{\mbox{\boldmath$ #1 $}}} % for vectors of Greek letters
\newcommand{\uv}[1]{\ensuremath{\boldsymbol{\mathbf{\widehat{#1}}}}} % for unit vectors
\newcommand{\abs}[1]{\left| #1 \right|} % for absolute value ||x||
%\newcommand{\mag}{\abs} % magnitude, just another name for \abs
\newcommand{\norm}[1]{\left\Vert #1 \right\Vert} % for norm ||v||
\newcommand{\avg}[1]{\left< #1 \right>} % for average <x>
\newcommand{\inner}[2]{\left< #1, #2 \right>} % for inner product <x,y>
\newcommand{\set}[1]{ \left\{ #1 \right\} } % for sets {a,b,c,...}
\newcommand{\tr}{\opname{tr}} % for trace
\newcommand{\Tr}{\opname{Tr}} % for Trace

% notational shortcuts
\newcommand{\reals}{\mathbb{R}} % real numbers
\newcommand{\complexes}{\mathbb{C}} % complex numbers
\newcommand{\nats}{\mathbb{N}} % natural numbers
\newcommand{\irrats}{\mathbb{Q}} % irrationals
\newcommand{\quats}{\mathbb{H}} % quaternions (a la Hamilton)
\newcommand{\euclids}{\mathbb{E}} % Euclidean space
\newcommand{\bigo}{\mathcal{O}} % big O notation
\newcommand{\Lag}{\mathcal{L}} % fancy Lagrangian
\newcommand{\Ham}{\mathcal{H}} % fancy Hamiltonian





%%%%%%%%%%%%%%%%%%%
% some templates for various things
\begin{comment}

% template for figures
\begin{figure}
\centering
\includegraphics{myfile.png}
\caption{This is a caption}
\label{fig:myfigure}
\end{figure}

% template for Feynman diagrams using feynmf/feynmp
\begin{fmfgraph*}(40,25)
\fmfleft{em,ep}
\fmf{fermion}{em,Zee,ep}
\fmf{photon,label=$Z$}{Zee,Zff}
\fmf{fermion}{fb,Zff,f}
\fmfright{fb,f}
\fmfdot{Zee,Zff}
\end{fmfgraph*}

% template for drawing plots with pgfplot
\pgfplotsset{compat=1.3,compat/path replacement=1.5.1}
\begin{tikzpicture}
\begin{axis}[
extra x ticks={-2,2},
extra y ticks={-2,2},
extra tick style={grid=major}]
\addplot {x};
\draw (axis cs:0,0) circle[radius=2];
\end{axis}
\end{tikzpicture}

\end{comment}
%%%%%%%%%%%%%%%%%%%


\title{Phys 221A -- Quantum Mechanics -- Lec03}
\author{UCLA, Fall 2014}
\date{\formatdate{13}{10}{2014}} % Activate to display a given date or no date (if empty),
         % otherwise the current date is printed 

\begin{document}
\maketitle


\section{Introduction}

Any theory we consider will have a complete set of compatible observables which we call an operator basis. That is, given operators $\set{A, B, C, \dots}$, we will have 
\begin{equation}
0 = [A,B] = [A,C] = [B,C] = \dots.
\end{equation}
These operators will have associated quantum numbers $\set{\alpha, \beta, \gamma, \dots}$ such that a subset of them is in one-to-one correspondence with a basis vectors in the associated orthonormal basis. Typically these operators will be the Hamiltonian $H$ along with symmetry generators. 

\subsection{Examples}

\subsubsection{Particle in a centro-symmetric potential}

In this system the Hamiltonian is given by
\begin{equation}
H = \frac{\v{p}^2}{2m} + V(\abs{\v{r}}) + H_\text{so}
\end{equation}
where $H_\text{so} \propto \v{L} \cdot \v{S}$. Here the Hilbert space is $L_2$. If we neglect spin-orbit coupling the operator basis is given by $\set{H_0, L^2, L_z, S_z}$. Otherwise we will need to use the total angular momentum $\v{J}$, which gives us operators $\set{H, J^2, J_z, L^2}$. 

\subsubsection{Free particle}

Here the Hamiltonian is given by $H = \v{p}^2 / 2m$. This is the minimal Hamiltonian, and its operator basis is given by $\set{p_x, p_y, p_z, S_z}$. Notice that $H \propto p_x^2 + p_y^2 + p_z^2$. 

\subsubsection{Harmonic oscillator in 3D}

Here the Hamiltonian is given by
\begin{equation}
H = \frac{\v{p}^2}{2m} + \frac{1}{2} k \v{x}^2
\end{equation}
and the operator basis is given by $\set{H, S_z}$. 


\subsection{More on operators and shit}

Consider a Hermitian operator $O = O^\dagger$ and normal vector $\ket{\psi} \in G$, so $\braket{\psi}{\psi} = 1$. Then the expectation value of $O$ in the state $\psi$ is 
\begin{equation}
\avg{O}_\psi = \bra{\psi} O \ket{\psi}.
\end{equation}
Define the corresponding \textit{deviation operator} $\Delta_O$ as
\begin{equation}
\Delta_O = O - \avg{O}_\psi.
\end{equation}
Then the quantity $\avg{ (\Delta_O)^2 }_\psi$ characterizes quantum fluctuations of $O$ in the state $\psi$. Note of course that the operator $(\Delta_O)^2$ is Hermitian. 


\subsection{Generalized uncertainty principle}

The generalized uncertainty principle states that, given any $\psi \in G$ and two self-adjoint operators $A,B$, we have 
\begin{equation}
\avg{ (\Delta_A)^2 }_\psi \avg{ (\Delta_B)^2 }_\psi \geq \frac{1}{4} \abs{ \avg{ [A,B] }_\psi }^2.
\end{equation}
Let's prove this. Consider the states $\ket{\alpha} = \Delta_A \ket{\psi}$, $\ket{\beta} = \Delta_B \ket{\psi}$. Then the Cauchy inequality tells us that
\begin{equation}
\braket{\alpha}{\alpha} \braket{\beta}{\beta} \geq \abs{\braket{\alpha}{\beta}}^2
\end{equation}
so that we have
\begin{equation}
\avg{(\Delta_A)^2}_\psi \avg{(\Delta_B)^2}_\psi \geq \abs{\avg{\Delta_A \Delta_B}_\psi }^2.
\end{equation}
We can decompose 
\begin{equation}
\Delta_A \Delta_B = \underbrace{\frac{1}{2} \{ \Delta_A, \Delta_B \}}_\text{Hermitian} + \underbrace{\frac{1}{2} [\Delta_A, \Delta_B]}_\text{anti-Hermitian}
\end{equation}
and find that
\begin{equation}
\avg{\Delta_A \Delta_B} = \underbrace{\frac{1}{2} \avg{ \{ \Delta_A, \Delta_B \} }}_\text{real} + \underbrace{\frac{1}{2} \avg{ [\Delta_A, \Delta_B] } }_\text{imaginary}.
\end{equation}
Thus from the inequality before we have
\begin{align}
\avg{(\Delta_A)^2} \avg{(\Delta_B)^2} &\geq \abs{\avg{\Delta_A \Delta_B} }^2 \\
	&\geq \frac{1}{4} \abs{ \avg{ \{ \Delta_A, \Delta_B \} }}^2 + \frac{1}{4} \abs{ \avg{ [\Delta_A, \Delta_B] } }^2 \\
	&\geq \frac{1}{4} \abs{ \avg{ [\Delta_A, \Delta_B] } }^2. 
\end{align}
We drop the anti-commutator term because it turns out to not be very important. 


\subsection{Copenhagen interpretation}

Given a (pure) quantum state $\ket{\psi} \in G$ normalized $\braket{\psi}{\psi} = 1$, we can measure a dynamical variable $A = A^\dagger$ with eigenstates $\ket{\alpha_i}$ and eigenvalues $a_i$. If $\ket{\psi} = \sum_i c_i \ket{\alpha_i}$ then the measurement apparatus yields a measurement $a_i$ with probability $P_i = \abs{c_i}^2$ and the physical state ``collapses'' to the state $\ket{\alpha_i}$. 

After measurement we are left with a mixed state. After ``infinitely'' many measurements we find the density matrix 
\begin{equation}
\rho = \sum_i P_i \ket{\alpha_i} \bra{\alpha_i}
\end{equation}
where $\sum_i P_i = 1$ since $\braket{\psi}{\psi} = 1$. More generally the density matrix is given by
\begin{equation}
\rho = \sum_i P_i \ket{\psi_i} \bra{\psi_i}.
\end{equation}
For any operator $O$ that is evaluated over all possible outcomes we have
\begin{equation}
\avg{O} = \sum_i P_i \avg{O}_{\alpha_i} = \opname{tr} (O\rho).
\end{equation}

For an operator $A$ define the trace
\begin{equation}
\opname{tr} A = \sum_i \bra{\alpha_i} A \ket{\alpha_i}
\end{equation}
where $\set{\ket{\alpha_i}}$ is an orthonormal basis. It's independent of basis: given another orthonormal basis $\set{\beta_i}$ we have
\begin{align}
\opname{tr} A &= \sum_i \bra{\alpha_i} A \ket{\alpha_i} \\
	&= \sum_{ij} \braket{\beta_j}{\alpha_i} \bra{\alpha_i} A \ket{\beta_j} \\
	&= \sum_j \bra{\beta_j} A \ket{\beta_j}.
\end{align}

The trace blah blah blah [blah blah]

Averaging over outcomes we have
\begin{align}
\sum_i P_i a_i &= \avg{A}_\psi = \opname{tr} (A\rho) \\
\sum_i P_i (a_i - \avg{A})^2 &= \avg{(\Delta_A)^2}_\psi = \opname{tr} (\Delta_A^2 \rho).
\end{align}

The density matrix is just represents the projection operator combined with the statistical outcomes of measurements. 

If we measure $A$ in state $\ket{\psi}$, we get the density matrix
\begin{equation}
\rho = \sum_i P_i \ket{\alpha_i} \bra{\alpha_i}.
\end{equation}
If we measure $B$ relative to the outcomes of $A$ we get
\begin{equation}
\avg{B} = \opname{tr} (B\rho), \qquad \avg{(\Delta_B)^2} = \opname{tr} (\Delta_B^2 \rho).
\end{equation}

[He rambled a bit more here about some stuff but I stopped paying attention.]




\end{document}
