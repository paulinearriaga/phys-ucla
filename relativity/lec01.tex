% declare document class and geometry
%\documentclass[12pt]{article} % use larger type; default would be 10pt
% ***********************************************************
% ******************* PHYSICS HEADER ************************
% ***********************************************************
% Version 2
\documentclass[12pt]{article} 





\usepackage{datetime} % allows easy formatting of dates, e.g. \formatdate{dd}{mm}{yyyy}

\usepackage{amsmath} % AMS Math Package
\usepackage{amsthm} % Theorem Formatting
\usepackage{amssymb}	% Math symbols such as \mathbb
\usepackage{graphicx} % Allows for eps images
\usepackage{multicol} % Allows for multiple columns
\usepackage[dvips,letterpaper,margin=1in,bottom=1in]{geometry}
 % Sets margins and page size
\pagestyle{empty} % Removes page numbers
\makeatletter % Need for anything that contains an @ command 
%\renewcommand{\maketitle} % Redefine maketitle to conserve space
%{ \begingroup \vskip 10pt \begin{center} \large {\bf \@title}
%	\vskip 10pt \large \@author \hskip 20pt \@date \end{center}
%  \vskip 10pt \endgroup \setcounter{footnote}{0} }
\makeatother % End of region containing @ commands
\renewcommand{\labelenumi}{(\alph{enumi})} % Use letters for enumerate
% \DeclareMathOperator{\Sample}{Sample}
\let\vaccent=\v % rename builtin command \v{} to \vaccent{}
\renewcommand{\v}[1]{\ensuremath{\mathbf{#1}}} % for vectors
\newcommand{\gv}[1]{\ensuremath{\mbox{\boldmath$ #1 $}}} 
% for vectors of Greek letters
\newcommand{\vx}{\ensuremath{\v{x}}} 
% for vectors of Greek letters
\newcommand{\vy}{\ensuremath{\v{y}}} 
% for vectors of Greek letters
\newcommand{\xdot}{\ensuremath{\dot{x}}} 
% for vectors of Greek letters

\newcommand{\ydot}{\ensuremath{\dot{y}}} 
% for vectors of Greek letters
\usepackage{commath} % for some nice standardized syntax stuff. 
	% \dif, \Dif, \od, \pd, \md, \(abs | envert), \(norm | enVert), \(set | cbr), \sbr, \eval, \int(o | c)(o | c), etc
\newcommand{\bbar}[1]{\bar{\bar{#1}}} % for barring things twice -- use \cbar or \zbar instead of original \bbar

\newcommand{\uv}[1]{\ensuremath{\mathbf{\hat{#1}}}} % for unit vector
%\newcommand{\abs}[1]{\left| #1 \right|} % for absolute value
\newcommand{\avg}[1]{\left< #1 \right>} % for average
\let\underdot=\d % rename builtin command \d{} to \underdot{}
\renewcommand{\d}[2]{\frac{d #1}{d #2}} % for derivatives
\newcommand{\dd}[2]{\frac{d^2 #1}{d #2^2}} % for double derivatives
%\newcommand{\pd}[2]{\frac{\partial #1}{\partial #2}} 
% for partial derivatives
\newcommand{\fd}[2]{\frac{\delta #1}{\delta #2}} 
% for functional derivatives

\newcommand{\pdd}[2]{\frac{\partial^2 #1}{\partial #2^2}} 
% for double partial derivatives
\newcommand{\pdc}[3]{\left( \frac{\partial #1}{\partial #2}
 \right)_{#3}} % for thermodynamic partial derivatives
\newcommand{\ket}[1]{\left| #1 \right>} % for Dirac bras
\newcommand{\bra}[1]{\left< #1 \right|} % for Dirac kets
\newcommand{\braket}[2]{\left< #1 \vphantom{#2} \right|
 \left. #2 \vphantom{#1} \right>} % for Dirac brackets
\newcommand{\matrixel}[3]{\left< #1 \vphantom{#2#3} \right|
 #2 \left| #3 \vphantom{#1#2} \right>} % for Dirac matrix elements
\newcommand{\grad}[1]{\gv{\nabla} #1} % for gradient
\let\divsymb=\div % rename builtin command \div to \divsymb
\renewcommand{\div}[1]{\gv{\nabla} \cdot #1} % for divergence
\newcommand{\curl}[1]{\gv{\nabla} \times #1} % for curl
\let\baraccent=\= % rename builtin command \= to \baraccent
\renewcommand{\=}[1]{\stackrel{#1}{=}} % for putting numbers above =
\newtheorem{prop}{Proposition}
\newtheorem{thm}{Theorem}[section]
\newtheorem{lem}[thm]{Lemma}
\theoremstyle{definition}
\newtheorem{dfn}{Definition}
\theoremstyle{remark}
\newtheorem*{rmk}{Remark}
\newcommand{\bigO}{\mathcal{O}} % big O notation
\let \bigo = \bigO % deprecated version. keeping for now because need to update instances in older files










\makeatletter
% À droite
\renewcommand\subsection{\@startsection {subsection}{1}{\z@}%
                                   {-3.5ex \@plus -1ex \@minus -.2ex}%
                                   {2.3ex \@plus.2ex}%
                                   {\raggedright\normalfont\Large\bfseries}}
\makeatother


\makeatletter
\def\section{\@ifstar\unnumberedsection\numberedsection}
\def\numberedsection{\@ifnextchar[%]
  \numberedsectionwithtwoarguments\numberedsectionwithoneargument}
\def\unnumberedsection{\@ifnextchar[%]
  \unnumberedsectionwithtwoarguments\unnumberedsectionwithoneargument}
\def\numberedsectionwithoneargument#1{\numberedsectionwithtwoarguments[#1]{#1}}
\def\unnumberedsectionwithoneargument#1{\unnumberedsectionwithtwoarguments[#1]{#1}}
\def\numberedsectionwithtwoarguments[#1]#2{%
  \ifhmode\par\fi
  \removelastskip
  \vskip 5ex\goodbreak
  \refstepcounter{section}%
  \hbox to \hsize{\vbox{%
      \noindent
      \leavevmode
      \begingroup
      \Large\bfseries\raggedleft
      \thesection.\ 
      #2\par
      \endgroup
      \vskip -2ex
      \noindent\hrulefill
      \vskip -2.2ex\nobreak
      \noindent\hrulefill
      }}\nobreak
  \vskip 2ex\nobreak
  \addcontentsline{toc}{section}{%
    \protect\numberline{\thesection}%
    #1}%
  }
\def\unnumberedsectionwithtwoarguments[#1]#2{%
  \ifhmode\par\fi
  \removelastskip
  \vskip 5ex\goodbreak
%  \refstepcounter{section}%
  \hbox to \hsize{\vbox{%
      \noindent
      \leavevmode
      \begingroup
      \Large\bfseries\raggedleft
%      \thesection.\ 
      #2\par
      \endgroup
      \vskip -2ex
      \noindent\hrulefill
      \vskip -2.2ex\nobreak
      \noindent\hrulefill
      }}\nobreak
  \vskip 2ex\nobreak
  \addcontentsline{toc}{section}{%
%    \protect\numberline{\thesection}%
    #1}%
  }
\makeatother
\pagestyle{empty}




% ***********************************************************
% ********************** END HEADER *************************
% ***********************************************************

\usepackage[margin=1in]{geometry} % handle page geometry





\title{Phys 220A -- Relativity -- Lec01}
\author{UCLA, Winter 2014}
\date{\formatdate{6}{1}{2014}} % Activate to display a given date or no date (if empty),
         % otherwise the current date is printed 

\begin{document}
\setlength{\unitlength}{1mm}
\maketitle

\section{Introduction}
Gravity is the most familiar force but mysterious. We can compare it to the electric force
\begin{equation}
\frac{1}{\alpha} \frac{m_p^2}{M_p^2} = .8 \times 10^{-36}
\end{equation}
Where $M_p$ is the Planck mass
\begin{equation}
M_p = \left(\frac{\hbar c}{G_N}\right)^{1/2} = 1/2 \times 10^19 GeV/c^2  \approx 10^{-5} g
\end{equation}
and the mass of the proton is
\begin{equation}
m_p \approx 1 GeV/c^2 \approx 10^{24} g
\end{equation}
So gravitational corrections to atomic physics are suppressed by a factor of
\begin{equation}
\alpha^x \left(\frac{m}{M_p}\right)^x
\end{equation}
which is small for any values of m and x. 

The minimum distance at which gravity has been tested is to the order
of $10^{-2} cm$. What does it mean to test? We can't observe the
exchange of gravitons like we can observe photons. All we can do is
measure the interactions between huge numbers of particles. There's
also another interaction typical of gravity: dark energy. 
\begin{equation}
\rho \approx M_{vac}^4
\end{equation}
where
\begin{equation}
M_{vac} \approx 10^{-3} eV
\end{equation}
If we translate this energy scale to the length scale, it's on the
order of magnitude of $10^{-2}cm$. Because of the enormous
discrepancy between these mass scales, extrapolating what we know about
gravity experimentally down to Planck mass scales may be incorrect. It
may be different at particle energy scales. 

\section{Pre-relativity Physics}

Forces given by Newton's laws. Coordinates given by the Cartesian
coordinate vector. This law is invariant under transformations between frames of a
  given type: changing to a coordinate system in which 
\begin{equation}
x' = R * \v{x} + \v{v} t + \v{d}
\end{equation}
\begin{equation}
t' = t + \tau
\end{equation}
Where R is a 3 dimensional rotation matrix, v is a velocity between
frames, d is an arbitrary displacement of the origin and t is a time
shift.  These transformations are called the Galilean group of
transformation. This is a ten-parameter group. These frames are
inertial frames.

What distinguishes inertial from non-inertial frames? We have inertial
frames because there is absolute space and absolute time. The laws
have a simple distinguished form if you're either at rest in absolute
space or moving uniformly in absolute time. Then Maxwell theory came
along and Maxwell theory has a velocity in it, the velocity of light
which is part of the equations. The constancy of the speed of light in
all frames was inconsistent with the
framework of Galilean boosts. Lorentz and Poincare found the set of
transformations leaving Maxwell theory invariant.

Einstein then postulated that all inertial frames are equivalent
(relativity principle) and that the velocity of light is constant in
all frames. If you take these two postulates then by very simple
algebra we can deduce that the set of transformations that satisfy
these two are the Lorentz transformations which replaces the Galilean
transformations. However the Lorentz
transformations mix space and time nontrivially. In special relativity
the inert ail frames are special and if you go to a non inertial frame
your laws of physics will change. Now what characterizes an inertial
frame? Unlike the Newtonian formalism where we have absolute space and
absolute time, we have absolute spacetime





\section{Lorentz Transformations}
We now take the Cartesian coordinates plus time and put it into a
four-vector $x^\alpha = (x^1, x^2, x^3, t)$ where 
$\alpha = 1, 2, 3, 0$. In this classes c is 1 and $\hbar=1$. We will
also use the summation convention. Let's define Lorentz
transformations in the way we want to use them. What is a Lorentz
transformation? It can be generalized as follows
\begin{equation}
x^\beta \rightarrow x'^\alpha = \Lambda^\alpha _\beta x^\beta
\end{equation}
Where $\Lambda^\alpha_\beta$ satisfies
\begin{equation}\label{blah}
\Lambda^\alpha_\gamma \Lambda^\beta_\delta \eta_{\alpha \beta} =
\eta_{\gamma \delta}
\end{equation}
where
\begin{equation}
\eta_{\alpha \beta} = diag(1, 1, 1, -1)
\end{equation}
only $\Lambda$ satisfying this are Lorentz transformations. The
\textbf{Minkowski metric} is one where $\eta$ is defined by $diag(1,
1, 1, -1)$.

The fundamental property of the Minkowski metric is that it leaves the
proper time interval invariant
\begin{equation}
d\tau^2 = -\eta_{\alpha\beta} dx^\alpha dx^\beta
\end{equation}
We prove this in the homework. For light propagation
\begin{equation}
d\tau = 0
\end{equation}

The homogeneous Lorentz group (in contrast to the Poincare group) is
characterize simply by a Lorentz boost. The Poincare group is
characterized by 10 parameters, like the Gallilean
group. From~\ref{blah} one can prove that (homework)
\begin{equation}
det \Lambda = \pm 1
\end{equation}
and that $\Lambda_0^0$ is either greater than or equal to 1 or less
than or equal to -1. The basic subset of the Galilean group is where
\begin{equation}
\det \Lambda = 1
\end{equation}
\begin{equation}
\Lambda^0_0 \le 1
\end{equation}
Then this defines the proper orthochronous (preserving the direction
of time) Lorentz subgroup.

The other icase is to get the matrices with
\begin{equation}
\det \Lambda = -1
\end{equation}
\begin{equation}
\Lambda^0_0 \le 1
\end{equation}
by multiplying a member of the first group by the matrix $\Lambda = diag (-1, -1 ,-1, 1)$. This
is a space inversion matrix. We have the subgroup where
\begin{equation}
\det \Lambda = -1
\end{equation}
\begin{equation}
\Lambda^0_0 \ge -1
\end{equation}
multiply by $\Lambda = 1, 1, 1, -1$ which is a time inversion. Final
case is where
\begin{equation}
\det \Lambda = 1
\end{equation}
\begin{equation}
\Lambda^0_0 > -1
\end{equation}
multiply by $\Lambda = (-1, -1, -1, -1)$ which is an inversion of
space and time. 

Let's concertrate on the proper orthochronous Lorentz group. Ordinary
three dimensional rotations are contained in the POLG. Suppose we take
a matrix in which the space component 
\begin{equation}
\Lambda^i_j = R^i_j, \Lambda^0_0 = 1, \Lambda^0_i = 0
\end{equation}
in other words is simply a rotation matrix. So R satisfies $det R = 1$
and the rotation is an orthogonal matrixwhich means that if we take $R
R^T = 1$. In addition to these rotations, the group will also contain
the Lorentz boosts. Rotations can in general be written in terms of 3
angles
\begin{equation}
R(\v{\theta}) = e^{\theta_i J_i}
\end{equation}
where J are the generators of the rotation, or in a quantum mechanical
setting are the angular momentum operators. These rotations are
physically represented if they satisfy the commutation relations i.e.
\begin{equation}
[J_i, J_j] = \epsilon_{ijk} J_k
\end{equation}
This defines the SO(3)



\end{document}
